% Appendix: SE(3) and SE(2) Trajectory Interpolation
\section{SE(3) and SE(2) Trajectory Interpolation}
\label{app:trajectory_interpolation}

This section describes trajectory smoothing methods for both SE(3) and SE(2) formulations. The high-level policy generates discrete waypoints at low frequency, while the low-level policy requires smooth, dense trajectories at high frequency.

\subsection{SE(3) Trajectory Interpolation}

\subsubsection{Rotation Representation Pipeline}

While the neural network uses \textbf{rotation\_6d} representation for training stability and continuity, trajectory interpolation requires conversion to quaternions for SLERP:

$$
\text{rotation\_6d} \xrightarrow{\text{inverse transform}} \text{quaternion} \xrightarrow{\text{SLERP}} \text{interpolated trajectory}
$$

This conversion ensures proper handling of rotation manifold geometry while maintaining computational efficiency.

\subsubsection{Geodesic Interpolation}

Given discrete waypoints $\lbrace T_{si}^{des}[\tau_k] \rbrace_{k=0}^{H}$ at times $\lbrace t_k \rbrace_{k=0}^{H}$, we construct a smooth trajectory $T_{si}^{des}(t)$ using SE(3) geodesics. For $t \in [t_k, t_{k+1}]$, the interpolated transformation is:

$$
T_{si}^{des}(t) = T_{si}^{des}[\tau_k] \exp\left(\alpha(t) \log\left(T_{si}^{des}[\tau_k]^{-1} T_{si}^{des}[\tau_{k+1}]\right)\right),
$$

where $\alpha(t) \in [0,1]$ is a smooth interpolation parameter. To ensure $C^1$ continuity in velocity, we use cubic interpolation:

$$
\alpha(t) = 3s^2 - 2s^3, \quad s = \frac{t - t_k}{t_{k+1} - t_k}.
$$

This formulation provides the shortest path on SE(3) between consecutive waypoints while maintaining smooth velocity profiles. However, it requires matrix exponential/logarithm operations and may encounter numerical instability for small rotations.

\subsubsection{Decoupled Interpolation (Recommended)}

A more practical approach decouples translation and rotation, offering numerical stability and independent velocity constraints:

\textbf{Translation:} For position $\mathbf{p}_{si}(t) \in \mathbb{R}^3$, we support two interpolation schemes:

\textit{Linear Interpolation} (computationally efficient):
$$
\mathbf{p}_{si}(t) = \mathbf{p}_{si}[\tau_k] + s \cdot (\mathbf{p}_{si}[\tau_{k+1}] - \mathbf{p}_{si}[\tau_k]), \quad s = \frac{t - t_k}{t_{k+1} - t_k}
$$

\textit{Cubic Spline Interpolation} (smoother velocity profiles):
$$
\mathbf{p}_{si}(t) = \mathbf{a}_3 s^3 + \mathbf{a}_2 s^2 + \mathbf{a}_1 s + \mathbf{a}_0,
$$

where coefficients $\lbrace \mathbf{a}_j \rbrace$ are determined by boundary conditions (positions and velocities at waypoints).

\textbf{Rotation:} Use Spherical Linear Interpolation (SLERP) for quaternions $\mathbf{q}_{si}(t)$:

$$
\mathbf{q}_{si}(t) = \mathbf{q}_{si}[\tau_k] \left(\mathbf{q}_{si}[\tau_k]^{-1} \mathbf{q}_{si}[\tau_{k+1}]\right)^{\alpha(t)},
$$

where:
\begin{itemize}
\item Quaternion exponentiation is defined via the exponential map on SO(3)
\item $\alpha(t) = s$ for linear SLERP, or $\alpha(t) = 3s^2 - 2s^3$ for cubic smoothing
\item SLERP ensures geodesic path (shortest rotation) on the SO(3) manifold
\item Preserves unit norm: $\|\mathbf{q}_{si}(t)\| = 1$ for all $t$
\item Numerically stable for all rotation magnitudes
\end{itemize}

\textbf{Combined Transformation:} Construct $T_{si}^{des}(t)$ from $\mathbf{p}_{si}(t)$ and $R_{si}(t) = \text{quat2mat}(\mathbf{q}_{si}(t))$:

$$
T_{si}^{des}(t) = \begin{bmatrix} R_{si}(t) & \mathbf{p}_{si}(t) \\ \mathbf{0}^\top & 1 \end{bmatrix} \in \text{SE}(3).
$$

\subsubsection{Distance Metrics and Velocity Constraints}

For trajectory planning with kinematic limits, we define SE(3) distance as decoupled metrics:

\textbf{Position Distance:}
$$
d_{\text{pos}}(\mathbf{p}_1, \mathbf{p}_2) = \|\mathbf{p}_2 - \mathbf{p}_1\|_2 \quad \text{(meters)}
$$

\textbf{Rotation Distance:}
$$
d_{\text{rot}}(\mathbf{q}_1, \mathbf{q}_2) = \left\|(\mathbf{q}_2 \mathbf{q}_1^{-1})\right\|_{\text{angle}} \quad \text{(radians)}
$$

where $\|\mathbf{q}\|_{\text{angle}}$ denotes the rotation angle magnitude of quaternion $\mathbf{q}$.

When adding waypoints with velocity constraints, compute minimum duration:

$$
\Delta t_{\min} = \max\left(\frac{d_{\text{pos}}}{v_{\max}}, \frac{d_{\text{rot}}}{\omega_{\max}}\right),
$$

where $v_{\max}$ is maximum linear velocity (m/s) and $\omega_{\max}$ is maximum angular velocity (rad/s). This ensures both translation and rotation constraints are satisfied simultaneously.

\subsubsection{Implementation Notes}

\begin{itemize}
\item \textbf{Frequency:} High-Level Policy generates waypoints at $f_{HL} = 5$ Hz (action chunks of $H = 10$ steps); Low-Level Policy requires trajectories at $f_{LL} = 50$ Hz, requiring interpolation ratio $f_{LL}/f_{HL} = 10$.
\item \textbf{Horizon:} For action chunks of $H = 10$ waypoints, the smoothed trajectory contains $H_{LL} = H \cdot (f_{LL}/f_{HL}) = 100$ samples.
\item \textbf{Time Clipping:} Queries outside $[t_0, t_H]$ are clipped to boundary values.
\item \textbf{Numerical Stability:} SLERP handles all rotation magnitudes robustly, including near-identity rotations, avoiding singularities present in matrix logarithm approaches.
\end{itemize}

\subsubsection{Comparison: Coupled vs. Decoupled Approaches}

\textbf{Coupled SE(3) Geodesic:}
\begin{itemize}
\item True geodesic on SE(3) manifold with mathematically elegant formulation
\item Requires computationally expensive matrix exponential/logarithm
\item Numerical instability for small rotations (log singularity)
\item Cannot independently constrain translation/rotation velocities
\end{itemize}

\textbf{Decoupled Interpolation:}
\begin{itemize}
\item Numerically stable and computationally efficient
\item Independent velocity constraints for translation and rotation
\item Flexible interpolation schemes (linear, cubic, etc.)
\item Geodesic on SO(3) subgroup (though not on full SE(3))
\item Widely adopted in practical robotic applications
\end{itemize}

\subsection{SE(2) Trajectory Smoothing}

For planar manipulation tasks, trajectory smoothing is simplified to 2D.

\subsubsection{Planar Position Interpolation}

Given discrete waypoints $\{T_{si}^{des}[\tau]\}_{\tau=0}^H = \{(x[\tau], y[\tau], \theta[\tau])\}_{\tau=0}^H$ from the high-level planner at 10 Hz, we generate dense trajectories at 50 Hz (Low-Level Policy frequency).

\textbf{Position Interpolation:}
Cubic spline interpolation through position waypoints $(x[\tau], y[\tau])$:

$$
\begin{bmatrix} x(t) \\ y(t) \end{bmatrix} = \sum_{j=0}^{3} a_j s^j, \quad s = \frac{t - t_k}{t_{k+1} - t_k}
$$

where coefficients $\{a_j\}$ satisfy boundary conditions (positions and velocities at waypoints).

\subsubsection{Planar Orientation Interpolation}

Circular interpolation on SO(2) ensuring shortest path:

$$
\theta(t) = \theta_k + \text{wrap}(\theta_{k+1} - \theta_k) \cdot \phi(s)
$$

where:
\begin{itemize}
\item $\phi(s) = 3s^2 - 2s^3$ (cubic smoothing for $C^1$ continuity)
\item $\text{wrap}(\Delta\theta) = \text{atan2}(\sin\Delta\theta, \cos\Delta\theta)$ ensures $|\Delta\theta| \leq \pi$ (shortest angular path)
\end{itemize}

\subsubsection{Body Twist Computation}

From smooth trajectory $T_{si}^{des}(t)$, compute desired body twist via time differentiation:

$$
\mathcal{V}_i^{des}(t) = \begin{bmatrix} \dot{\theta}(t) \\ \dot{x}(t)\cos\theta(t) + \dot{y}(t)\sin\theta(t) \\ -\dot{x}(t)\sin\theta(t) + \dot{y}(t)\cos\theta(t) \end{bmatrix} \in \mathbb{R}^3
$$

This formulation transforms spatial velocities $(\dot{x}, \dot{y})$ into the body frame using the current orientation $\theta(t)$.

\subsubsection{Implementation Details}

\textbf{Frequency Matching:}
\begin{itemize}
\item High-level planner: 10 Hz (10 waypoints per chunk)
\item Low-level policy: 50 Hz (100 interpolated poses per chunk)
\item Interpolation ratio: 5× upsampling
\end{itemize}

\textbf{Smoothness Guarantees:}
\begin{itemize}
\item Position: $C^2$ continuous (cubic splines)
\item Orientation: $C^1$ continuous (cubic blending function)
\item Velocity: $C^0$ continuous at waypoints
\end{itemize}

\textbf{Edge Cases:}
\begin{itemize}
\item \textbf{Orientation wrapping:} Handle $\theta$ discontinuities at $\pm\pi$ using atan2
\item \textbf{Zero velocity waypoints:} Use natural spline boundary conditions
\item \textbf{Stationary goals:} Exponential decay to final pose or zero-velocity boundary condition
\end{itemize}

% Discussion
\section{Discussion}
\label{sec:discussion}

%------------------------------------------------------------------------------
\subsection{Key Findings}
\label{sec:key_findings}
%------------------------------------------------------------------------------

Our SE(2) experiments confirm that \textbf{explicit encoding of geometric constraints and wrench feedback} is essential for robust bimanual manipulation of articulated objects. Three key findings emerge from our comparison of position control, classical impedance control, and SWIVL:

\paragraph{Position Control Is Fundamentally Limited.}
High-stiffness position control treats each arm's reference trajectory as an independent hard constraint, ignoring the physical coupling between arms grasping a shared object. When the high-level Flow Matching Policy outputs kinematically inconsistent commands---inevitable for any learning-based planner without explicit constraint modeling---rigid tracking generates excessive fighting forces that destabilize grasps and cause task failures. This limitation is intrinsic to the control paradigm, not the quality of demonstrations.

\paragraph{Classical Impedance Control Is Insufficient.}
While impedance control introduces compliance that partially absorbs coordination errors, its isotropic and fixed parameters cannot distinguish between motion along the object's kinematic constraint (which should be compliant) and motion orthogonal to it (which may require stiffness). The linearized SE(3) formulation further breaks down under large trajectory deviations caused by constraint violations, leading to unpredictable behavior.

\paragraph{Object-Aware Decomposition Enables Physical Intelligence.}
SWIVL's success stems from explicitly representing the object's kinematic structure through screw-axis-based projection operators. By decomposing motions and wrenches into bulk (transport) and internal (articulation) components, the learned policy can modulate compliance independently for each subspace. This geometric grounding, combined with real-time wrench feedback, enables the policy to discover force-compliant strategies that position control cannot express and classical impedance control cannot adapt to.

%------------------------------------------------------------------------------
\subsection{Limitations and Future Work}
\label{sec:limitations}
%------------------------------------------------------------------------------

\paragraph{Current Limitations.}
\begin{itemize}[leftmargin=1.5em]
    \item \textbf{SE(2) planar setting}: Our evaluation is restricted to planar manipulation with 1-DoF articulated objects. Real-world tasks require full SE(3) workspace with multi-DoF constraints.
    \item \textbf{Simulation only}: Sim-to-real transfer remains to be validated on physical dual-arm platforms.
    \item \textbf{Known object kinematics}: We assume screw axes $\mathcal{B}_l, \mathcal{B}_r$ are provided a priori, requiring either CAD models or prior estimation.
    \item \textbf{Fixed high-level planner}: Experiments use a single Flow Matching Policy; generalization across diverse planners (diffusion policies, ACT, teleoperation) is not yet validated.
    \item \textbf{Pre-grasped objects}: Grasping acquisition and regrasping strategies are not addressed.
\end{itemize}

\paragraph{Future Directions.}
\begin{itemize}[leftmargin=1.5em]
    \item \textbf{SE(3) extension and real-robot deployment}: The mathematical framework naturally extends to SE(3) by scaling observation/action dimensions. Validation on dual-arm platforms (e.g., Franka Panda, UR5e) with domain randomization for sim-to-real transfer is a priority.
    \item \textbf{Ablation studies}: Systematic analysis of each architectural component---stable vector fields, screw-decomposed control, FiLM conditioning, wrench feedback---would isolate their individual contributions.
    \item \textbf{Cross-planner generalization}: Evaluating SWIVL with diverse high-level planners would validate the claim that our low-level controller is planner-agnostic.
    \item \textbf{Novel object transfer}: Testing on unseen object geometries and mass distributions would assess the generalization capability of FiLM-based object conditioning.
    \item \textbf{Online constraint estimation}: Integrating screw-axis estimation (e.g., from Screw-Splatting) would eliminate the requirement for known object models.
    \item \textbf{Multi-DoF articulation}: Extending to $k$-DoF objects with Jacobian $J_i \in \mathbb{R}^{6 \times k}$ would broaden applicability to complex articulated structures.
\end{itemize}

%------------------------------------------------------------------------------
\subsection{Broader Impact}
\label{sec:broader_impact}
%------------------------------------------------------------------------------

SWIVL addresses a critical gap in deploying learning-based manipulation policies to contact-rich bimanual tasks. By providing a principled low-level control layer that operates beneath arbitrary high-level planners, our framework enables safer deployment of cognitive policies (including VLAs and foundation models) that lack explicit physical reasoning. The explicit minimization of fighting forces reduces contact stress, improving both task success and safety in human-robot collaboration scenarios. Potential applications include bimanual assembly in manufacturing, surgical assistance, and cooperative manipulation in service robotics.
% Experiments
\section{Experiments}
\label{sec:experiments}

We evaluate SWIVL on bimanual manipulation of a \textbf{single-joint articulated object} in an SE(2) planar benchmark environment. While the Method section develops a general SE(3) formulation for $k$-DoF articulated objects, our experiments specialize this framework to the planar, 1-DoF setting where the object Jacobians collapse to fixed body-frame screw axes for each gripper. Our experiments address three key questions: (Q1) Does SWIVL improve task success and reduce internal forces compared to imitation learning baselines? (Q2) How do design choices in the reference motion decomposition and impedance modulation affect performance? (Q3) Can SWIVL generalize across diverse high-level planners and novel object instances within this SE(2) benchmark?

\textbf{Why SE(2) and 1-DoF?} Our SE(2) planar environment with single-joint articulated objects is a deliberate design choice that enables controlled, rigorous evaluation of SWIVL's core methodology while maintaining full mathematical consistency with the general framework. This simplification strategy follows established scientific practice of validating fundamental principles in tractable settings before scaling to full complexity.

\textit{Rationale for SE(2) planar restriction:} By limiting to 2D workspace, we achieve: (1) \textit{Phenomenon isolation}---the fundamental challenges SWIVL addresses (force coupling, kinematic constraint satisfaction, compliant coordination) manifest identically in planar and spatial settings. Internal forces and constraint violations arise from the same geometric and dynamic principles regardless of ambient dimension, allowing focused study without confounding factors from 6-DoF control complexity or visual perception challenges. (2) \textit{Experimental tractability}---reduced dimensionality enables large-scale systematic evaluation (100 trials $\times$ 9 objects $\times$ multiple ablations) with statistical significance that would be prohibitive in full SE(3). (3) \textit{Analytical clarity}---planar settings eliminate gravitational effects orthogonal to the motion plane ($\gamma_{b,i} = 0$), simplifying force analysis and enabling direct interpretation of contact force patterns. (4) \textit{Established precedent}---influential robotics work validates methods in 2D before 3D deployment, demonstrating scientific value of dimensional reduction for concept validation.

\textit{Rationale for 1-DoF object restriction:} Single-joint articulated objects provide the minimal yet complete structure to evaluate SWIVL's constraint-aware design: (1) \textit{Mathematical completeness}---a 1-DoF joint fully instantiates the holonomic constraint ${}^s\mathcal{V}_l - {}^s\mathcal{V}_r = \mathcal{S}\,\dot{q}_{obj}$ (Eq.~3 specialized to $k=1$), exercising the screw-decomposed controller's projection operators, bulk--internal motion decomposition, and wrench-based force regulation. All key architectural components---projection $P_{i,\parallel}, P_{i,\perp}$, metric tensor $G(\alpha)$, and impedance modulation $d_{\parallel}, d_{\perp}$---are actively engaged even with $k=1$. (2) \textit{Simplified parameter space}---constant body-frame screw axes $\mathcal{B}_l, \mathcal{B}_r \in \mathbb{R}^3$ (independent of $q_{obj}$ for single-joint mechanisms with fixed grasps) eliminate configuration-dependent Jacobian variations present in multi-DoF chains, isolating the core challenge of coordinated force-compliant manipulation from kinematic complexity. This allows unambiguous attribution of performance to SWIVL's impedance learning rather than object-specific kinematic modeling. (3) \textit{Diverse coverage}---our 9-object benchmark spans three joint types (revolute, prismatic, fixed) with varying geometry, mass distribution, and inertia, providing sufficient diversity to assess generalization across constraint structures while maintaining experimental control. (4) \textit{Direct scalability path}---the framework naturally extends to $k$-DoF: screw axes $\mathcal{B}_i \in \mathbb{R}^{3 \times k}$ become time-varying matrices, projections use $(\mathcal{B}_i^\top G \mathcal{B}_i)^{-1} \in \mathbb{R}^{k \times k}$, and the policy observes configuration-dependent Jacobians. The SE(2), 1-DoF validation ensures correctness of the fundamental projection-based decomposition before addressing multi-DoF kinematic coupling.

Importantly, our simplifications preserve the essential geometric structure: the SE(2) Lie group, screw theory, adjoint transformations, holonomic constraints, reciprocal product duality between twists and wrenches, and learned metric tensors all reduce consistently from the general SE(3), $k$-DoF formulation in Section~\ref{sec:method}. This ensures experimental findings directly inform the design principles applicable to full SE(3), multi-DoF deployment. Complete mathematical reduction is detailed in Appendix~\ref{app:se2}.

\subsection{Experimental Setup}

\textbf{Environment.} We use an SE(2) planar workspace with dual 3-DoF planar end-effectors operating under \textbf{direct body wrench control} $\mathcal{F}_i = [m_{z,i}, f_{x,i}, f_{y,i}]^\top \in \mathbb{R}^3$. The effective manipulation space is bounded with walls at the workspace boundaries. Each end-effector provides 3-axis force-torque sensing synchronized with the low-level control frequency, enabling real-time wrench feedback for impedance modulation. The hierarchical control architecture combines vision-based high-level planning (10 Hz) with SWIVL's low-level SE(2) control (50 Hz), with grippers maintained closed throughout episodes. Complete workspace specifications and physics configuration are in Appendix~\ref{app:environment_settings}.

	extbf{Tasks.} We evaluate on 9 planar articulated objects spanning three joint categories: fixed (rigid transport), revolute (angular articulation), and prismatic (linear articulation), with 3 object variants per category. Each object is modeled as a \textbf{single-DoF articulated body} whose kinematic constraint is captured by a fixed screw axis in each end-effector frame. Concretely, for object joint configuration $q_{obj} \in \mathbb{R}$, the general SE(3) formulation in Section~\ref{sec:problem_formulation} reduces to the SE(2) holonomic constraint
\begin{equation}
{}^s\mathcal{V}_l - {}^s\mathcal{V}_r = \mathcal{S}\, \dot{q}_{obj}, \quad {}^s\mathcal{V}_i \in \mathbb{R}^3, \ \mathcal{S} \in \mathbb{R}^3, \ \dot{q}_{obj} \in \mathbb{R},
\end{equation}
where $\mathcal{S} = [s_\omega, s_{v,x}, s_{v,y}]^\top$ is the spatial-frame screw axis of the joint. In our planar setting, the corresponding body-frame screw axes $\mathcal{B}_l, \mathcal{B}_r \in \mathbb{R}^3$ at each grasp are \\textbf{constant in time and independent of $q_{obj}$}, reflecting fixed grasp points on a 1-DoF mechanism. For each object, we specify a fixed goal configuration $({}^s T_o^{goal}, q_{obj}^{goal})$ visually marked in the workspace. Episodes initialize with objects at random configurations $({}^s T_o^{init}, q_{obj}^{init})$ within safe regions, spawning uniformly to ensure diverse initial conditions. The task objective is to manipulate the object from its initial configuration to the goal through coordinated bimanual control. Each object is tested over 100 trials with randomized initialization. Success requires: object position error $< 10$ pixels, orientation error $< 5^\circ$, joint error $< 5^\circ$ (revolute) or $< 5$ pixels (prismatic), and maintained grasp throughout. Detailed task specifications and evaluation protocol are in Appendix~\ref{app:environment_settings}.

\textbf{Implementation.} The Low-Level Policy uses a multi-stream architecture with FiLM conditioning to modulate feature processing based on object structure, instantiated with the SE(2) specialization described in Appendix~\ref{app:se2}. 

\textit{Observation Space} ($\mathbb{R}^{30}$): Following Method Section 3.2.3, the SE(2) policy observes: (1) \textit{Reference twists}---individual end-effector SE(2) reference twists $\mathcal{V}_l^{\text{ref}}, \mathcal{V}_r^{\text{ref}} \in \mathbb{R}^3$ (comprising $[\omega_z, v_x, v_y]^\top$) computed by the Reference Motion Field Generator (Layer 2) from the stable imitation vector field (Eq.~\eqref{eq:vector_field}); (2) \textit{Object constraints}---time-invariant body-frame screw axes $\mathcal{B}_l, \mathcal{B}_r \in \mathbb{R}^3$ that encode the 1-DoF joint structure (constant due to fixed grasp points on single-joint mechanisms); (3) \textit{Wrench feedback}---3-axis force-torque measurements $\mathcal{F}_l, \mathcal{F}_r \in \mathbb{R}^3$ (comprising $[m_z, f_x, f_y]^\top$) from each end-effector; (4) \textit{Proprioception}---end-effector SE(2) poses $T_{sb_l}, T_{sb_r} \in \mathrm{SE}(2)$ (represented as $[x, y, \theta]^\top \in \mathbb{R}^3$ each) and body twists $\mathcal{V}_l, \mathcal{V}_r \in \mathbb{R}^3$. The policy network internally computes bulk-internal decomposition of reference twists and measured wrenches using projection operators.

\textit{Action Space} ($\mathbb{R}^7$): Following Method Section 3.2.3, the SE(2) policy outputs impedance modulation variables: $a_t = (d_{l,\parallel}, d_{r,\parallel}, d_{l,\perp}, d_{r,\perp}, k_{p_l}, k_{p_r}, \alpha) \in \mathbb{R}^7$. Here, $d_{l,\parallel}, d_{r,\parallel}, d_{l,\perp}, d_{r,\perp} \in \mathbb{R}^+$ are \textbf{per-arm damping coefficients} for internal motion (parallel to screw axis) and bulk motion (orthogonal to screw axis), enabling independent compliance modulation for each arm. $k_{p_l}, k_{p_r} \in \mathbb{R}^+$ are per-arm stiffness gains for the stability term $k_{p_i} \mathcal{E}_i$ in the vector field (Eq.~\eqref{eq:vector_field}), determining correction strength toward desired trajectories. $\alpha \in \mathbb{R}^+$ is the \textbf{learnable characteristic length scale} that defines the SE(2) metric tensor $G = \mathrm{diag}(\alpha^2, 1, 1) \in \mathbb{R}^{3 \times 3}$ (reduced from SE(3)'s $G = \mathrm{diag}(\alpha^2 I_3, I_3)$), which weights the single planar rotation $\omega_z$ relative to translations $v_x, v_y$ in the inner product $\langle \mathcal{V}_1, \mathcal{V}_2 \rangle_G = \alpha^2 \omega_{1,z} \omega_{2,z} + v_{1,x} v_{2,x} + v_{1,y} v_{2,y}$ for orthogonal decomposition.

These impedance variables parameterize the SE(2) Screw-decomposed Controller (Layer 4 analog) as detailed in Appendix~\ref{app:se2}. The controller constructs projection operators $P_{i,\parallel} = \mathcal{B}_i (\mathcal{B}_i^\top G \mathcal{B}_i)^{-1} \mathcal{B}_i^\top G$ and $P_{i,\perp} = I - P_{i,\parallel}$ (Eq.~\eqref{eq:projection_operators} specialized to SE(2) with 1-DoF where $\mathcal{B}_i \in \mathbb{R}^{3 \times 1}$ are the constant body-frame screw axes), then computes the damping matrix $K_{d_i} = G(P_{i,\parallel} d_{i,\parallel} + P_{i,\perp} d_{i,\perp})$ (Eq.~\eqref{eq:damping_matrix} with per-arm damping), and generates commanded wrenches via $\mathcal{F}_{\mathrm{cmd}, i} = K_{d_i} (\mathcal{V}_i^{\text{ref}} - \mathcal{V}_i) + \mu_{b,i}$ (Eq.~\eqref{eq:impedance_control_main}, with gravity $\gamma_{b,i} = 0$ in planar settings). These commanded SE(2) body wrenches $\mathcal{F}_{\mathrm{cmd}, i}$ are directly executed as control inputs to the planar end-effectors, consistent with the wrench-based impedance control framework in Method Section 3.2.4. By construction, the projection-based damping matrix structure ensures the holonomic constraint is implicitly satisfied through compliant regulation of bulk and internal motion components, coordinating forces to minimize non-productive internal wrenches.

Training uses standard PPO with the reward function in Eq.~\eqref{eq:reward} adapted to SE(2): tracking reward uses the learned $G$-metric, safety reward penalizes internal wrenches $\mathcal{F}_{i,\perp}$ orthogonal to $\mathcal{S}$, and regularization encourages smooth SE(2) twists. Complete network architecture and training hyperparameters are in Appendix~\ref{app:learning_settings}.

\textbf{Baselines and Ablations.} We compare against OWIL (Object-Wrench conditioned Imitation Learning), a pure imitation baseline trained on expert demonstrations. OWIL receives SE(2) object screw information ($\mathcal{B}_l, \mathcal{B}_r, j_{\text{type}}$) and planar wrench feedback ($\mathcal{F}_l, \mathcal{F}_r$) as input, but lacks the bulk--internal motion decomposition and outputs direct individual end-effector SE(2) twists $\mathcal{V}_l, \mathcal{V}_r \in \mathbb{R}^3$ without explicit constraint-aware parameterization. Full OWIL specifications are in Appendix~\ref{app:owil}.

To isolate SWIVL's design contributions in this SE(2), 1-DoF setting, we ablate four key components: \textbf{(A) Observation Space Composition}---comparing individual reference tracking only without bulk--internal decomposition (SWIVL-IndivRef), full observation without wrench feedback (SWIVL-NoWrench), and full observation with SE(2) bulk--internal decomposition and wrench sensing (SWIVL). This reveals the impact of explicit task semantics and force feedback. \textbf{(B) Vector Field Design}---comparing pure temporal tracking without stability (SWIVL-TempOnly), spatially-optimal contraction field (SWIVL-SpatialField), and temporally-stable imitation vector field (SWIVL). This evaluates the necessity and implementation of corrective feedback. \textbf{(C) Action Space Parameterization}---comparing residual corrections to individual arms (SWIVL-Residual) vs. SE(2) kinematic-constrained bulk--internal parameterization that enforces $ {}^s\mathcal{V}_l - {}^s\mathcal{V}_r = \mathcal{S}\,\dot{q}_{obj}$ by design (SWIVL). This tests whether structural constraint encoding outperforms implicit learning with penalty rewards. \textbf{(D) Object Conditioning Architecture}---comparing direct concatenation (SWIVL-Concat) vs. FiLM-based feature modulation (SWIVL). This assesses how object information integration affects generalization across revolute, prismatic, and fixed joint types. Detailed variant specifications and training protocols are in Appendix~\ref{app:ablation}.

	extbf{Evaluation Metrics.} Primary metrics include success rate (\%), SE(2) constraint violation (pixels/s), and internal force (N). Secondary metrics assess tracking RMSE (pixels), peak contact force (N), and motion smoothness (pixels/s$^3$). Success rate uses bootstrap confidence intervals (10{,}000 resamples). Constraint violation measures time-averaged deviation from the SE(2) holonomic constraint:
\begin{equation}
	ext{CViol} = \frac{1}{T} \sum_{t=1}^{T} \big\| {}^s\mathcal{V}_l(t) - {}^s\mathcal{V}_r(t) - \mathcal{S} \, \dot{q}_{obj}(t) \big\|_2 , \quad {}^s\mathcal{V}_i(t) \in \mathbb{R}^3, \ \mathcal{S} \in \mathbb{R}^3.
\end{equation}
Internal force quantifies non-productive planar contact forces orthogonal to the SE(2) screw axis using the reciprocal-product-based decomposition in Appendix~\ref{app:orthogonal_decomposition}:
\begin{equation}
F_{\text{int}} = \frac{1}{T} \sum_{t=1}^{T} \sum_{i \in \{l,r\}} \big\| \mathcal{F}_{i,\perp}(t) \big\|_2 , \quad \mathcal{F}_i(t) = [m_{z,i}, f_{x,i}, f_{y,i}]^\top \in \mathbb{R}^3.
\end{equation}
Complete metric definitions and testing procedures are in Appendix~\ref{app:environment_settings}. Cross-planner generalization is evaluated with three high-level planners: HLP-Diff (diffusion-based), HLP-ACT (transformer-based), and HLP-Teleop (human demonstrations). Zero-shot transfer tests on 6 novel object variants (scaled, asymmetric, different masses) quantify generalization capabilities within the SE(2), single-joint setting.

\subsection{Main Results}

	extbf{Q1: Comparison with Imitation Learning.} [Results to be added after experimental evaluation]

	extbf{Q2: Design Choice Ablations.} [Results to be added after experimental evaluation]

	extbf{Q3: Generalization Capabilities.} [Results to be added after experimental evaluation]

\subsection{Analysis and Discussion}

[Detailed analysis will be added after experimental evaluation, including: internal force patterns, SE(2) constraint satisfaction analysis, sample efficiency comparisons, failure mode characterization, and computational performance measurements.]

\subsection{Summary}

Our experimental framework evaluates SWIVL's physics-aware approach---combining stable imitation vector fields, SE(2) bulk--internal motion decomposition for a single-joint articulated object, constraint-consistent action spaces, and FiLM-based object conditioning---against pure imitation learning baselines. The evaluation covers diverse planar articulated objects, multiple high-level planners, and systematic ablation studies to isolate the contribution of each design component. These SE(2) results instantiate the general SE(3) methodology and highlight the importance of explicitly encoding geometric and wrench structure for robust bimanual manipulation.

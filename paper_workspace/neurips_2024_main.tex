\documentclass{article}

% if you need to pass options to natbib, use, e.g.:
%     \PassOptionsToPackage{numbers, compress}{natbib}
% before loading neurips_2024

% ready for submission
% \usepackage[final]{neurips_2024}

% to compile a preprint version, e.g., for submission to arXiv, add add the
% [preprint] option:
\usepackage[preprint]{neurips_style/neurips_2024}

% Remove the conference notice at the bottom of the first page
\makeatletter
\renewcommand{\@notice}{}
\makeatother

% to compile a camera-ready version, add the [final] option, e.g.:
%     \usepackage[final]{neurips_2024}

% to avoid loading the natbib package, add option nonatbib:
%    \usepackage[nonatbib]{neurips_2024}

\usepackage[utf8]{inputenc} % allow utf-8 input
\usepackage[T1]{fontenc}    % use 8-bit T1 fonts
\usepackage{newunicodechar}  % support for unicode characters
% Define unicode characters that might appear in verbatim
\newunicodechar{θ}{\ensuremath{\theta}}
\newunicodechar{σ}{\ensuremath{\sigma}}
\newunicodechar{μ}{\ensuremath{\mu}}
\newunicodechar{τ}{\ensuremath{\tau}}
\newunicodechar{Δ}{\ensuremath{\Delta}}
\newunicodechar{ε}{\ensuremath{\varepsilon}}
\newunicodechar{γ}{\ensuremath{\gamma}}
\newunicodechar{λ}{\ensuremath{\lambda}}
\newunicodechar{β}{\ensuremath{\beta}}
\newunicodechar{─}{-}
\newunicodechar{├}{+}
\newunicodechar{└}{+}
\usepackage{hyperref}       % hyperlinks
\usepackage{url}            % simple URL typesetting
\usepackage{booktabs}       % professional-quality tables
\usepackage{amsfonts}       % blackboard math symbols
\usepackage{nicefrac}       % compact symbols for 1/2, etc.
\usepackage[expansion=false]{microtype}      % microtypography
\usepackage{xcolor}         % colors
\usepackage{amsmath}        % advanced math
\usepackage{amssymb}        % additional math symbols
\usepackage{mathtools}      % more math tools
\usepackage{algorithm}      % algorithm environment
\usepackage{algorithmic}    % algorithmic environment
\usepackage{graphicx}       % include graphics
\usepackage{subcaption}     % subfigures
\usepackage{enumitem}       % better enumerate
\usepackage{bm}             % bold math

\title{SWIVL: Screw-Wrench informed Impedance Variable Learning for bimanual manipulation of an articulated object}

% The \author macro works with any number of authors. There are two commands
% used to separate the names and addresses of multiple authors: \And and \AND.
%
% Using \And between authors leaves it to LaTeX to determine where to break the
% lines. Using \AND forces a line break at that point. So, if LaTeX puts 3 of 4
% authors names on the first line, and the last on the second line, try using
% \AND instead of \And before the third author name.

\author{%
  Kyungseo Park \\
  Student ID : 2020-12295 \\
  Department of Mechanical Engineering \\
  Seoul National University \\
  \texttt{erickun0125@snu.ac.kr} \\
  % examples of more authors
  % \And
  % Coauthor \\
  % Affiliation \\
  % Address \\
  % \texttt{email} \\
  % \AND
  % Coauthor \\
  % Affiliation \\
  % Address \\
  % \texttt{email} \\
  % \And
  % Coauthor \\
  % Affiliation \\
  % Address \\
  % \texttt{email} \\
  % \And
  % Coauthor \\
  % Affiliation \\
  % Address \\
  % \texttt{email} \\
}

\begin{document}

\maketitle

\begin{abstract}
Bimanual manipulation of articulated objects remains challenging due to inter-arm force coupling and complex kinematic constraints. While recent Vision-Language-Action (VLA) models demonstrate strong cognitive intelligence in high-level task planning, they lack explicit grounding in low-level physical interaction---specifically, force regulation and constraint satisfaction during contact-rich manipulation. We introduce \textbf{SWIVL} (\textbf{S}crew-\textbf{W}rench informed \textbf{I}mpedance \textbf{V}ariable \textbf{L}earning), a hierarchical control framework that bridges the gap between high-level cognitive planning and low-level physical execution. SWIVL consists of three key components:(1) \textbf{Screw Axes-based Twist \& Wrench Decomposition}---using orthogonal projection operators to enable independent compliance modulation for each subspace; (2) \textbf{Twist-driven SE(3) Impedance Control}---incorporating pose errors directly into reference twists through stable imitation vector fields, sidestepping the nonlinear pose error Jacobian; and (3) \textbf{Wrench-adaptive Impedance Variable Learning}---a reinforcement learning policy that modulates impedance parameters minimizing non-productive internal forces while ensuring compliant trajectory tracking. We evaluate SWIVL on SE(2) planar benchmarks involving rigid transport, revolute rotation, and prismatic insertion, demonstrating improved success rates, reduced internal forces compared to imitation learning baselines.
\end{abstract}

% Main Content
% Introduction
\section{Introduction}
\label{sec:introduction}

Learning-based robotic manipulation has achieved strong performance in single-arm settings, particularly for structured pick-and-place tasks learned via demonstration. However, extending such approaches to \textbf{dual-arm manipulation of an articulated object} remains challenging. Coordinated bimanual interaction induces rich \textbf{inter-arm force coupling} and must satisfy complex \textbf{object-centric kinematic constraints}. Existing Learning-from-Demonstration (LfD) frameworks---typically grounded in high-stiffness position control with no explicit representation of object kinematics---often generate unstable motions and large internal forces when multiple arms physically interact with a shared object.

\subsection*{Cognitive vs. Physical Intelligence}

To reason about this challenge, we distinguish between two complementary aspects of robot decision-making: \textbf{Cognitive Intelligence} and \textbf{Physical Intelligence}.

\textbf{Cognitive Intelligence} addresses high-level task understanding through semantic reasoning and planning. Modern VLA-based robot foundation models excel at interpreting goals and decomposing instructions via visual-language understanding based on VLM backbones. However, language operates as an abstracted symbolic representation that lacks the resolution needed for precise physical interaction---making it difficult to specify fine-grained motor commands for contact-rich manipulation such as force modulation or coordinated compliance control.

\textbf{Physical Intelligence}, in contrast, ensures safe and stable execution through explicit modeling of dynamics and kinematic principles. This includes regulating contact forces, satisfying geometric constraints, and generating dynamically consistent motions. This dimension remains underexplored in learning-based manipulation.


Most existing work advances Cognitive Intelligence through robot foundation models trained on cross-embodiment datasets. However, this approach introduces three limitations for contact-rich manipulation. First, cross-embodiment training cannot standardize low-level physical signals---wrench feedback, reference frames, and kinematic constraints vary across platforms. Second, imitation learning reproduces demonstrated trajectories without exploring contact dynamics, lacking the adaptive force modulation that reinforcement learning provides. Third, policies output poses executed via high-stiffness control, that can generate excessive forces and constraint violations. These limitations compound in bimanual articulated object manipulation, where force coupling and closed-chain constraints demand physical reasoning beyond current cognitive approaches.
This mismatch motivates methods that \textbf{bridge high-level cognitive policies to low-level physically grounded control}.

\subsection*{Problem Focus and Scope}

We focus on developing a low-level control stack for bimanual manipulation of an articulated object. Our formulation assumes access to (1) 6-axis wrench measurements at each end-effector via wrist-mounted force/torque sensors, and (2) the screw axes of an articulated object. While not all manipulation scenarios provide such information, these assumptions are satisfied in  structured domains---including repetitive assembly, industrial workflows that represent high-value deployment scenarios for dual-arm systems. Recent advances in geometric perception modules—screw-axis estimation (e.g., Screw-Splatting) further expand the applicability of our framework.

In this setting, we use high-level planners (VLA, Imitation Learning Policies, Teleoperation Interfaces) provide motion intentions as desired end-effector poses, without explicitly reasoning about object's physical information or inter-arm force interactions. Our hierarchical control stack transforms these commands into physically feasible bimanual coordination that satisfies kinematic constraints, suppresses internal forces, and ensures compliant interaction—without requiring the high-level planner to reason about any of these physical details.

\subsection{Our Approach : SWIVL}

To address these problems, we introduce \textbf{SWIVL} (\textbf{S}crew-\textbf{W}rench informed \textbf{I}mpedance \textbf{V}ariable \textbf{L}earning). SWIVL enables safe bimanual manipulation with object-aware decomposed impedance control via screw axes and adaptive impedance variable modulation via wrench feedback. This work makes the following contributions:

\begin{enumerate}
    \item \textbf{Twist-Driven SE(3) Impedance Control via Stable Imitation Vector Fields}:  Bypass the nonlinear pose-error Jacobian inherent in SE(3) impedance control by incorporating pose errors directly into reference twists, enabling geometrically consistent compliance even under large trajectory deviations.

    \item \textbf{Screw Axes-Decomposed Twist and Wrench Spaces}: Orthogonal projection operators structurally partition control into internal (joint articulation) and bulk (object transport) components, and dually decompose wrenches into internal forces and bulk forces, enabling independent compliant behavior for each object kinematics aware subspace.

    \item \textbf{Wrench-adaptive Impedance Variable Learning}: A reinforcement learning policy modulates impedance parameters conditioned on object screw axes and real-time wrench feedback streamed from wrist-mounted force-torque sensors, learning to suppress harmful fighting forces while maintaining compliant trajectory tracking across diverse joint types.
\end{enumerate}

We validate SWIVL on SE(2) benchamarks with articulated objects spanning two joint types: revolute and prismatic. Our results demonstrate that SWIVL achieves higher success rates and lower internal forces compared to imitation learning baselines, highlighting the importance of explicit physical modeling for bimanual manipulation of articulated objects.
% Related Work
\section{Related Work}
\label{sec:related_work}

Our work bridges research in learning-based robotic manipulation, dual-arm coordination, physically grounded control, and stable motion representation. We organize related work around four themes: \textbf{learning paradigms for manipulation}, \textbf{stable motion representation and geometric control}, \textbf{dual-arm coordination and force control}, and \textbf{hierarchical integration of high-level planning with low-level execution}.

\subsection{Learning-Based Robotic Manipulation}

\textbf{Imitation Learning and Behavior Cloning.} Learning from Demonstration (LfD) has emerged as a dominant paradigm for acquiring manipulation skills, with Behavior Cloning (BC) directly mapping observations to actions through supervised learning. Recent LfD policies use generative models like Diffusion Policy~\cite{chi2023diffusion} and ACT~\cite{zhao2023act} to output action chunks for temporal consistency. However, these approaches focus on mimicking demonstrations in high-stiffness position control without reasoning about contact forces or kinematic constraints, and lack closed-loop feedback to handle trajectory deviations.

\textbf{Vision-Language-Action Models and Foundation Models.} Cross-embodiment robot foundation models like RT-1~\cite{brohan2022rt1}, RT-2~\cite{brohan2023rt2}, $\pi_0$~\cite{black2023pi0}, and Octo~\cite{octo2024} leverage large-scale datasets like Open-X~\cite{openx2023} for broad semantic understanding in cognitive capabilities. While these models excel at high-level goal specification and demonstrate impressive generalization to novel objects and instructions, they are trained without explicit access to force/moment feedback, reference frames, or kinematic constraints---information that is inherently embodiment-specific and difficult to standardize across heterogeneous robot platforms. This limitation motivates our approach of decoupling high-level cognitive planning from low-level physically grounded execution.

\textbf{RL for Manipulation.}Reinforcement learning has shown promise for acquiring contact-rich manipulation skills that are difficult to specify through demonstrations alone. Deep RL has been applied to contact-rich tasks like in-hand manipulation~\cite{andrychowicz2020learning} and tool use~\cite{handa2023dextreme}, optimizing physical objectives through environmental interaction.  However, most RL formulations focus on single-task settings and require dense, task-specific reward signals. We leverage RL to learn task-agnostic low-level control that generalizes across manipulation tasks.

\subsection{Dual-Arm Manipulation and Force Control}

\textbf{Classical Dual-Arm Coordination.} Hybrid position/force control~\cite{raibert1981hybrid} decomposes control into position and force subspaces, while object-centric approaches~\cite{uchiyama1988symmetric} unify dual-arm systems as virtual closed-chain mechanisms. Screw theory~\cite{murray1994mathematical} provides geometric foundations for kinematic constraints. While these classical methods provide theoretical foundations for dual-arm coordination, they typically rely on precise analytical models and struggle with complex contact dynamics in articulated object manipulation.

\textbf{Impedance and Compliant Control.} Impedance control~\cite{hogan1985impedance} regulates position-force relationships through virtual mass-spring-damper dynamics. Object impedance control~\cite{schneider2007passivity} extends this to dual-arm settings. However, these methods assume rigid objects and require manual tuning, limiting applicability to articulated manipulation with varying constraints.

\textbf{Learning Force-Aware Manipulation.} Recent works learn force-sensitive skills through force-conditioned policies~\cite{abu2022learning} and tactile guidance~\cite{she2021cable}. However, these approaches typically learn task-specific force patterns rather than generalizable force-compliant behaviors. SWIVL learns task-agnostic force regulation by incorporating screw-axis-based wrench decomposition into the reward structure.

\subsection{Hierarchical Integration of Cognitive and Physical Control}

\textbf{Residual Reinforcement Learning.}  To bridge the gap between high-level imitation policies and low-level reactive control, residual RL methods augment frozen behavior-cloned policies with learned corrective actions. Residual RL methods like ResiP~\cite{silver2022residual} augment frozen BC policies with learned corrective actions, protecting against catastrophic forgetting while handling distribution shifts. However, these focus on single-task manipulation without addressing multi-arm coordination or force regulation. SWIVL learns a standalone low-level policy that operates beneath arbitrary high-level planners.

\textbf{Hierarchical Control for Whole-Body Systems.} High-level planners (VLAs, teleoperation) generate semantic task specifications, but translating these into dynamically feasible whole-body motions requires robust low-level control. Hierarchical approaches separate high-level planning from low-level execution: LeVERB~\cite{radosavovic2024learning} learns latent action vocabularies translated by RL policies, while loco-manipulation frameworks~\cite{fu2023deep} use RL to track commands while maintaining stability. SWIVL adopts similar philosophy for dual-arm manipulation, where high-level planners provide pose commands and RL-trained low-level policy ensures physically feasible execution through explicit constraint and force modeling.
% Method
\section{Method}
\label{sec:method}

We present \textbf{SWIVL}, a hierarchical control framework that bridges high-level cognitive planning with physically grounded bimanual execution. SWIVL consists of three key components: (1) a Stable Imitation Vector Field based \textbf{Reference Twist Field Generator} that transforms discrete high-level waypoints into dense, continuous vector fields defined over the entire task space---providing stable reference motions even when the system deviates from the desired trajectory,  (2) Screw Axes Decomposition based \textbf{Twist-driven Impedance Controller} that enables practical impedance control with pose error and twist error, and (3) a Reinforcement Learning based \textbf{Wrench-feedback and Object-conditioned Impedance Variable Learning Policy} that modulates the reference motions in a physically feasible manner by explicitly incorporating object information and end-effector wrench feedback.

\textbf{Notation.} We use the following frame conventions throughout:
\begin{itemize}
    \item $\{s\}$: Spatial (world) frame
    \item $\{b_i\}$: Body frame of end-effector $i \in \{l, r\}$
    \item $\{d_i\}$: Desired frame of end-effector $i$
    \item $T_{ab}$: Transformation from frame $\{b\}$ to frame $\{a\}$
    \item ${}^a\mathcal{V}_b$: Twist of frame $\{b\}$ expressed in frame $\{a\}$
\end{itemize}

\subsection{Problem Formulation}
\label{sec:problem_formulation}

We address the problem of \textbf{bimanual manipulation of articulated objects} where two robot arms cooperatively manipulate a shared articulated object. Our goal is to develop SWIVL, a control framework that tracks high-level motion plans while ensuring physical feasibility, stable grasping, and damage prevention during bimanual interaction. This problem presents four core challenges that directly motivate SWIVL's architectural components.

\textbf{Framework Scope.} We develop the theoretical framework in \textbf{SE(3) environment} with \textbf{multi-DoF articulated objects} for generality and real-world applicability, while experimental validation is conducted in \textbf{SE(2) planar environments} with \textbf{1-DoF articulated objects} to isolate core challenges of force coupling and constraint satisfaction. The formulation targets the Franka FR3 dual-arm platform for future real-world deployment.

\subsubsection{Challenge 1: From Discrete Poses to Dense Twist References}

Modern learning-based manipulation policies---including VLAs and behavior cloning architectures---predominantly output actions in the form of \textbf{action chunks}: sequences of waypoints that specify desired end-effector trajectories over a finite horizon. This action chunking paradigm enables temporal consistency and multi-step reasoning in high-level planning.

\textbf{The Challenge.} These action chunks present critical gaps for low-level execution. First, waypoints are inherently \textbf{sparse}, providing discrete snapshots while control requires continuous dense references at high frequency. Second, their \textbf{open-loop nature} becomes particularly problematic in bimanual manipulation: while single-arm tasks like pick-and-place typically track action chunks reliably, dual-arm coordination introduces inter-arm forces and complex contact dynamics that cause significant trajectory deviations, requiring corrective feedback to guide the system back.

More fundamentally, there exists a \textbf{configuration space versus velocity space mismatch}. High-level planners naturally output poses $T \in \mathrm{SE}(3)$, but kinematic constraints and impedance control are fundamentally more tractable in the twist space $\mathfrak{se}(3)$. Kinematic constraints are linear in twist space (${}^s\mathcal{V}_l - {}^s\mathcal{V}_r = \mathbf{J}_s \dot{\mathbf{q}}_{\text{obj}}$) but nonlinear in SE(3). Similarly, SE(3) impedance control involves nonlinear Jacobians $J_{\mathcal{E}}$ that introduce configuration-dependent coupling. This mismatch motivates transforming pose commands into twist-space representations where constraints are linear and impedance modulation is geometrically consistent.

\textbf{Solution Preview.} The \textbf{Reference Twist Field Generator} addresses this challenge through three operations: (1) SE(3) trajectory smoothing to generate dense pose trajectories, (2) body twist computation to transform poses into velocity space where constraints and impedance are tractable, and (3) stable imitation vector field construction that provides continuous reference twists $\mathcal{V}^{\text{ref}}$ with pose-error-driven corrective feedback.

\subsubsection{Challenge 2: Kinematic Constraints and Compliant Coordination}

When two arms grasp and manipulate an articulated object, their motions are coupled through the object's kinematic structure. Let $T_{sb_l}, T_{sb_r} \in \mathrm{SE}(3)$ denote the end-effector poses and $\mathbf{q}_{\text{obj}} \in \mathbb{R}^k$ represent the object's internal joint configuration. The kinematic structure is characterized by the spatial Jacobian $\mathbf{J}_s(\mathbf{q}_{\text{obj}}) \in \mathbb{R}^{6 \times k}$, whose columns are spatial screw axes $\mathcal{S}_j \in \mathbb{R}^6$ for each joint.

\textbf{Holonomic Constraint.} Since both end-effectors rigidly grasp the articulated object, the relative motion between them must exactly match the motion generated by the object's internal joints:
\begin{equation}
{}^s\mathcal{V}_{b_l} - {}^s\mathcal{V}_{b_r} = \mathbf{J}_s(\mathbf{q}_{\text{obj}}) \dot{\mathbf{q}}_{\text{obj}},
\end{equation}
where ${}^s\mathcal{V}_{b_l}, {}^s\mathcal{V}_{b_r} \in \mathbb{R}^6$ are the spatial twists and $\dot{\mathbf{q}}_{\text{obj}} \in \mathbb{R}^k$ is the joint velocity vector.

\textbf{The Challenge.} A seemingly natural solution is to structurally enforce this constraint in the action space---for example, by commanding $\mathcal{V}_r$ and $\dot{\mathbf{q}}_{\text{obj}}$ as actions and automatically computing $\mathcal{V}_l = \mathcal{V}_r + \mathbf{J}_s \dot{\mathbf{q}}_{\text{obj}}$. However, this \textbf{hard constraint approach is brittle}: any small error in $\mathbf{J}_s$ due to perception noise, calibration errors, or object model mismatch directly translates to physically infeasible commanded motions. When executed via high-stiffness position or velocity control, these infeasible commands generate large internal forces that risk grasp slippage, object damage, or hardware failure.

\textbf{Solution Preview.} Rather than enforcing hard kinematic constraints, SWIVL adopts a \textbf{compliant approach} through impedance control. This enables soft regulation of constraint violations through compliance rather than rigid enforcement, providing robustness to model uncertainties while minimizing harmful internal forces.

\subsubsection{Challenge 3: Impedance Control on SE(3) and Large Trajectory Deviations}

Impedance control---regulating the dynamic relationship between motion and force---is a natural paradigm for achieving compliant manipulation. In Euclidean spaces, impedance control is straightforward: pose errors naturally define potential energy, and elastic forces emerge through simple differentiation. However, extending impedance control to the SE(3) manifold presents fundamental mathematical challenges.

\textbf{SE(3) Impedance Difficulties.} Unlike Euclidean spaces, SE(3) is a manifold where defining meaningful error metrics is non-trivial. The most physically natural metric on SE(3) is the \textbf{kinetic energy-based Riemannian metric} induced by the spatial inertia matrix $M \in \mathbb{R}^{6 \times 6}$. For tracking control, we need to measure the "distance" between the current pose $T$ and desired pose $T_{\text{des}}$, typically represented by the error pose $T_{\text{err}} = T_{\text{des}}^{-1} T$. However, SE(3) lacks a \textbf{bi-invariant Riemannian metric}---a metric that remains consistent under both left and right group actions. The exponential map follows screw motions (constant-twist trajectories), while geodesics under the kinetic energy metric follow more complex, inertia-weighted paths. This mismatch means that the natural logarithm map $\log(T_{\text{err}})^\vee \in \mathfrak{se}(3)$, commonly used to define pose errors, does not correspond to minimal-energy paths under the physically meaningful metric.
For SE(3), a nonlinear Jacobian-like operator emerges:

\begin{equation}
{\mathcal{F}_{\mathrm{elastic}} = -J_{\mathcal{E}}^\top K \mathcal{E}} \, ,
\end{equation}

where $J_{\mathcal{E}}$ is given by:
\begin{equation}
J_{\mathcal{E}} = \begin{pmatrix} \alpha J_l^{-1}(e_R) & 0_{3 \times 3} \\ -[e_p] & I_{3} \end{pmatrix} \in \mathbb{R}^{6 \times 6} \, .
\end{equation}
Here, $J_l(\theta)$ is the left Jacobian of SO(3):
\begin{equation}
J_l(\theta) = I + \frac{1 - \cos \|\theta\|}{\|\theta\|^2} [\theta] + \frac{\|\theta\| - \sin \|\theta\|}{\|\theta\|^3} [\theta]^2 \, .
\end{equation} 
$J_{\mathcal{E}}$ is configuration-dependent and couples rotational and translational components in complex, state-dependent ways. This nonlinearity fundamentally limits the design freedom of the stiffness matrix $K$

\textbf{Practical Approximations and Their Breakdown.} Commonly used impedance control formulations circumvent these issues through two assumptions: (1) \textbf{small orientation errors}, and (2) \textbf{diagonal stiffness matrices}, allowing $ J_{\mathcal{E}}^\top K \mathcal{E} \approx K \mathcal{E}$, of avoiding cross-coupling between rotation and translation. Under these assumptions, the nonlinear Jacobian's effects become negligible.

\textbf{The Challenge.} These assumptions break down in our setting. First, articulated object constraints cause \textbf{large trajectory deviations}: when high-level planners generate commands without explicit constraint awareness, the resulting pose errors can be substantial, violating the small-error assumption. Second, our motion decomposition framework (separating bulk and internal motions) requires \textbf{non-diagonal stiffness matrices} to enable independent compliance modulation along different subspaces, making diagonal-only designs insufficient.

\textbf{Solution Preview.} Rather than directly computing elastic wrenches from pose errors through the problematic $J_{\mathcal{E}}$, SWIVL adopts a \textbf{twist-driven impedance control} approach. The \textbf{Reference Twist Field} incorporates pose error $\mathcal{E}_i$ directly into the reference twist through the stability term $k_{p_i} \mathcal{E}_i$, converting pose deviations into corrective velocity commands. The controller then performs impedance regulation in twist space, where damping naturally relates velocity errors to wrenches through linear operators, sidestepping the nonlinear Jacobian while maintaining compliant tracking behavior.

\subsubsection{Challenge 4: Motion Decomposition and Wrench-Based Coordination}

For task-agnostic control, the low-level policy must be trained solely on trajectory tracking rewards rather than task-specific success metrics, enabling the same policy to track diverse high-level commands without retraining for each task objective.

\textbf{Object-Centric Motion Decomposition.} In articulated object bimanual manipulation, tracking is fundamentally object-centric. Rather than independently tracking each arm's reference motion $\mathcal{V}_l^{\text{ref}}$ and $\mathcal{V}_r^{\text{ref}}$, it is advantageous to decompose these into:
\begin{itemize}
    \item \textbf{Internal motion}: Drives joint articulation, lies in $\text{range}(\mathbf{J}_s(\mathbf{q}_{\text{obj}}))$
    \item \textbf{Bulk motion}: Drives the object's overall motion through space, orthogonal to  $\text{range}(\mathbf{J}_s(\mathbf{q}_{\text{obj}}))$
\end{itemize}
Tracking these components separately provides clearer learning signals aligned with the object's kinematic structure.
\begin{align}
\mathcal{V}_{i,\parallel}^{\text{ref}} &= P_{i,\parallel} \mathcal{V}_i^{\text{ref}}, \quad \mathcal{V}_{i,\parallel} = P_{i,\parallel} \mathcal{V}_i \quad \text{(internal motion)}, \\
\mathcal{V}_{i,\perp}^{\text{ref}} &= P_{i,\perp} \mathcal{V}_i^{\text{ref}}, \quad \mathcal{V}_{i,\perp} = P_{i,\perp} \mathcal{V}_i \quad \text{(bulk motion)},
\end{align}

\textbf{The Challenge: Bulk Motion Ambiguity.} If the high-level policy generated kinematically consistent commands, then after decomposing each arm's motion, their bulk components would naturally agree. However, high-level planners lack explicit constraint awareness, resulting in $\mathcal{V}_l^{\text{ref}}$ and $\mathcal{V}_r^{\text{ref}}$ whose bulk motions \textbf{do not match}. This creates a fundamental ambiguity: which arm's bulk motion should the policy track?

This ambiguity manifests as \textbf{inter-arm force coupling}. Internal motion components naturally satisfy kinematic constraints and generate minimal internal forces. However, bulk motion mismatch directly translates into inter-arm wrenches measured by force-torque sensors. Specifically, wrench components orthogonal to the screw axes---the non-productive components that do not contribute to joint actuation---directly reflect the magnitude of bulk motion disagreement. By decomposing measured wrenches as:
\begin{equation}
\mathcal{F}_{i,\perp} = \mathbf{P}_{i,\perp}^T \mathcal{F}_i, \quad \mathcal{F}_{i,\parallel} = \mathbf{P}_{i,\parallel}^T \mathcal{F}_i,
\end{equation}
we can quantify internal forces $\mathcal{F}_{i,\perp}$ arising from coordination errors. Minimizing $\|\mathcal{F}_{i,\perp}\|^2$ implicitly resolves the bulk motion ambiguity: the policy learns to track reference motions in a manner that achieves coordinated bulk motion while maintaining compliant internal motion.

\textbf{Solution Preview.} This challenge is addressed through \textbf{wrench-feedback conditioned RL} combined with \textbf{learned impedance variable modulation}. The \textbf{Impedance Variable Learning Policy} receives wrench decompositions and learns to adaptively modulate damping coefficients $d_{i,\parallel}$ and $d_{i,\perp}$ for internal versus bulk motions. The \textbf{Reward Design} explicitly penalizes internal wrenches $\|\mathcal{F}_{i,\perp}\|^2$, training the policy to minimize non-productive forces while tracking reference trajectories. This approach enables simple reference tracking to implicitly achieve decomposed motion coordination through force-compliant behavior.

\subsection{SWIVL Architecture}

SWIVL adopts a \textbf{four-layer hierarchical architecture} that directly addresses the four challenges identified above. The architecture decouples high-level reasoning from low-level physical interaction:

\begin{itemize}
    \item \textbf{Layer 1 (High-Level Policy)}: Provides task-level guidance through sparse waypoint generation
    \item \textbf{Layer 2 (Reference Twist Field Generator)}: Addresses Challenge 1 by transforming sparse waypoints into dense, closed-loop reference trajectories
    \item \textbf{Layer 3 (Impedance Variable Modulation Policy)}: Addresses Challenges 3 \& 4 by learning adaptive compliance modulation based on object geometry and wrench feedback
    \item \textbf{Layer 4 (Screw-decomposed Controller)}: Addresses Challenges 2 \& 3 by structurally enforcing kinematic constraints through orthogonal motion decomposition
\end{itemize}

The core innovation lies in the tight integration of Layers 2-4, which together enable physically grounded, force-compliant bimanual manipulation underneath arbitrary high-level planners.

\subsubsection{Layer 1: High-Level Policy}

The top layer generates goal-directed behavior in action chunk. This layer can be instantiated by:
\begin{itemize}
    \item \textbf{Vision-Language-Action (VLA) models}
    \item \textbf{Behavior cloning policies}
    \item \textbf{Teleoperation interfaces}
\end{itemize}

\textbf{Interface:} Outputs discrete end-effector pose waypoints $\lbrace T_{sd_l}[\tau], T_{sd_r}[\tau] \rbrace_{\tau=0}^{H}$ at low frequency.

\subsubsection{Layer 2: Reference Twist Field Generator}

This layer addresses Challenge 1 by bridging the gap between discrete high-level waypoints and continuous low-level control. High-level policies provide sparse waypoints at low frequency ($\sim$10Hz), while the low-level controller requires smooth, dense reference trajectories at high frequency ($\sim$50Hz) with corrective feedback capabilities. The Reference Twist Field Generator performs three steps to achieve this transformation:

\paragraph{Step 1: SE(3) Trajectory Smoothing.} To address the sparsity gap, we perform smooth interpolation in $SE(3)$ to obtain dense desired trajectories at the Low-Level Policy frequency $\Delta t_{LL} \ll \Delta t_{HL}$:

\begin{equation}
\lbrace T_{sd_l}(t), T_{sd_r}(t) \rbrace_{t=0}^{H_{LL}},
\end{equation}

where $H_{LL}$ is the smoothed trajectory horizon. We use SLERP for rotations and cubic splines for translations; see Appendix~\ref{app:trajectory_interpolation} for details. Both interpolation schemes are differentiable in time, so they induce smooth position and rotation trajectories $p_i^{\text{des}}(t)$ and $R_i^{\text{des}}(t)$ for each end-effector.

\paragraph{Step 2: Body Twist Computation.} For a smooth trajectory $T_{sd_i}(t) = \begin{bmatrix} R_{sd_i}(t) & p_{sd_i}(t) \\ 0 & 1 \end{bmatrix}$, we directly compute the desired twist by differentiating the pose and expressing the resulting velocities in the instantaneous desired frame. Let $R_{sd_i}(t)$ denote the rotation from the desired body frame $\{d_i\}$ to the spatial frame $\{s\}$, and $p_{sd_i}(t)$ the position of the desired body frame origin in the spatial frame. The corresponding desired-frame angular and linear velocities are given by

\begin{equation}
\label{eq:body_twist}
\mathcal{V}_i^{\text{des}}(t) =
\begin{bmatrix}
\omega_i^{\text{des}}(t) \\
v_i^{\text{des}}(t)
\end{bmatrix},
\quad [\omega_i^{\text{des}}(t)]_\times = R_{sd_i}(t)^{\top} \dot{R}_{sd_i}(t),
\quad v_i^{\text{des}}(t) = R_{sd_i}(t)^{\top} \dot{p}_{sd_i}(t).
\end{equation}

Here $[\omega]_\times$ is the skew-symmetric matrix representation of the angular velocity vector $\omega$, and both $\omega_i^{\text{des}}(t)$ and $v_i^{\text{des}}(t)$ are expressed in the desired frame $\{d_i\}$ by construction. 

\paragraph{Step 3: Stable Imitation Vector Field Design.} 
As execution progresses within an action chunk, the actual end-effector poses may deviate significantly from the desired trajectory due to tracking errors, disturbances, and model mismatch. 
To address the open-loop nature of waypoints and provide robustness when the system deviates from the desired trajectory, we construct a vector field that balances \textbf{imitation} of demonstrated motions and \textbf{stability} for error correction.
We employ a stable imitation vector field that combines two components: an imitation term that mimics the demonstrated velocity profile at the temporally synchronized point, and a stability term that provides corrective feedback via a term proportional to the SE(3) pose error for convergence to the desired trajectory:

\begin{equation}
\label{eq:vector_field}
\mathcal{V}_i^{\text{ref}}(t, T_{sb_i}) = \mathrm{Ad}_{T_{b_id_i}}\mathcal{V}_i^{\text{des}}(t) + k_{p_i} \mathcal{E},
\end{equation}
where $\mathrm{Ad}_T$ denotes the adjoint transformation that maps twists between frames. Since the desired twist $\mathcal{V}_i^{\text{des}}(t)$ is computed in the desired frame $\{d_i\}$ (Eq.~\ref{eq:body_twist}), we must transform it to the current body frame $\{b_i\}$ where the controller operates. The transformation $T_{b_id_i} = T_{b_is} T_{sd_i} = (T_{sb_i})^{-1} T_{sd_i}$ represents the relative transformation from the desired frame to the current body frame, and $\mathrm{Ad}_{T_{b_id_i}}$ performs the corresponding twist transformation. The pose error term is given by:
\begin{equation}
\begin{aligned}
\mathcal{E} &= \begin{pmatrix} \alpha e_R \\ e_p \end{pmatrix} \in \mathbb{R}^6 \\
e_{R_i} &= \log(R_{sb_i}^\top R_{sd_i})^\vee \in \mathbb{R}^3 \quad \text{(rotation error in body frame)} \\
e_{p_i} &= R_{sb_i}^\top(p_{sd_i} - p_{sb_i}) \in \mathbb{R}^3 \quad \text{(translation error in body frame)} \\
k_{p_i} &\in \mathbb{R} \quad \text{(scalar proportional gain)} \\
\end{aligned}
\end{equation}
Here, $\alpha$ represents a characteristic length that weights the rotational cost relative to translation, where $G = \mathrm{diag}(\alpha^2 I_3, I_3)$ is the metric tensor defining an inner product $\langle \cdot, \cdot \rangle_G$ on $\mathfrak{se}(3)$:
\begin{equation}
\langle \mathcal{V}_1, \mathcal{V}_2 \rangle_G = \frac{\alpha^2}{2} \mathrm{tr}([\omega_1]^\top [\omega_2]) + v_1^\top v_2 =  \mathcal{V}_1^\top G \mathcal{V}_2 \, ,
\end{equation}
where $\mathcal{V}_i = \begin{bmatrix} \omega_i \\ v_i \end{bmatrix}$.
This time-based approach offers computational efficiency ($O(1)$ lookup) while maintaining strong tracking performance for smooth bimanual trajectories. The term $k_p \mathcal{E}$, as will be described later, plays a role in creating an elastic force for impedance control without using the nonlinear Jacobian matrix to create a reference twist.

\subsubsection{Layer 3: Impedance Variable Modulation Policy}

This layer addresses Challenges 3 and 4 by learning adaptive impedance modulation. The Reinforcement Learning Policy $\pi_\theta: \mathcal{O} \to \mathcal{A}$ modulates impedance variables to enable physically feasible motions while accounting for object constraints and inter-arm force interactions. By explicitly conditioning on object geometry (screw axes) and wrench feedback, the policy learns to independently modulate compliance for bulk versus internal motions and minimize harmful internal forces.

\textbf{Observation Space $\mathcal{O}$.} The policy receives:

\begin{enumerate}
    \item \textbf{Reference twists}: $\lbrace \mathcal{V}_l^{\text{ref}}, \mathcal{V}_r^{\text{ref}} \rbrace \in \mathfrak{se}(3) \times \mathfrak{se}(3)$
    These are the reference motions computed by the Reference Twist Field Generator (Layer 2) at the current time $t$ and current end-effector poses $T_{sb_l}, T_{sb_r}$.

    \item \textbf{Object constraints}: $\lbrace \mathcal{B}_l, \mathcal{B}_r \rbrace \in \mathfrak{se}(3) \times \mathfrak{se}(3)$
    These encode the kinematic constraint of the manipulated object. $\mathcal{B}_l, \mathcal{B}_r$ are the body-frame screw axes defining the object's allowable internal motion directions at each end-effector. For articulated objects, these correspond to the columns of the object Jacobian $J_i$, related to the spatial screw axis via $\mathcal{B}_i = \mathrm{Ad}_{T_{b_is}} \mathcal{S}$ where $\mathrm{Ad}_{T_{b_is}}$ transforms spatial twists to body frame.

    \item \textbf{Wrench feedback}: $\lbrace \mathcal{F}_l, \mathcal{F}_r \rbrace \in \mathfrak{se}(3)^* \times \mathfrak{se}(3)^*$
    These are 6-dimensional wrench measurements (3D moment + 3D force) obtained from 6-axis force-torque sensors mounted at each end-effector's wrist. Raw sensor data may be filtered (e.g., low-pass filtering or exponential smoothing) when necessary to reduce measurement noise while preserving force feedback responsiveness for compliant control.

    \item \textbf{Proprioception}: $\lbrace T_{sb_l}, T_{sb_r}, \mathcal{V}_l, \mathcal{V}_r \rbrace$
    Task-space states including end-effector poses $T_{sb_l}, T_{sb_r} \in \mathrm{SE}(3)$ and body twists $\mathcal{V}_l, \mathcal{V}_r \in \mathfrak{se}(3)$.
\end{enumerate}

\textbf{Action Space $\mathcal{A}$.} We propose an impedance variable action space that structurally enforces object's kinematic constraints by motion decomposition. These variables parameterize the low-level controller (detailed in Section~\ref{sec:layer4_controller}), allowing the policy to modulate compliance behavior dynamically:
\begin{equation}
a_t = ( d_{l,\parallel},  d_{r,\parallel}, d_{l,\perp},  d_{r,\perp},k_{p_l},  k_{p_r},\alpha) \in \mathbb{R}^7,
\end{equation}
where $d_{i,\parallel}, d_{i,\perp}, k_{p_i}, \alpha \in \mathbb{R}^+$ are positive scalar gains, and:
\begin{itemize}
    \item $ d_{i,\parallel}$: Damping coefficient for internal motion (parallel to screw axis), regulating compliance along the object's degree of freedom.
    \item $ d_{i,\perp}$: Damping coefficient for bulk motion (orthogonal to screw axis), regulating compliance for the object's overall transport.
    \item $k_{p_i}$: Stiffness gain for the stability term in the vector field, determining the strength of correction towards the desired trajectory.
    \item $\alpha$: \textbf{Learnable characteristic length scale} that adaptively weights rotational error relative to translational error in the SE(3) metric tensor $G = \mathrm{diag}(\alpha^2 I_3, I_3)$. By learning $\alpha$, the policy discovers task-appropriate metric structures for orthogonal decomposition and compliance modulation.
\end{itemize}

These policy outputs parameterize the low-level controller (detailed in Layer 4 below), enabling adaptive, context-dependent impedance modulation.

\subsubsection{Layer 4: Screw-decomposed Twist-driven Impedance Controller}
\label{sec:layer4_controller}

This layer addresses Challenges 2 and 3 by executing low-level control that structurally enforces kinematic constraints while enabling independent compliance modulation. It tracks the reference motion $\mathcal{V}_i^{\text{ref}}$ from Layer 2 using the impedance parameters from Layer 3. The key innovation of this controller is its ability to provide \textbf{two complementary views}: (1) it behaves as a geometrically consistent SE(3) impedance controller, and (2) it explicitly decomposes control actions into bulk and internal motion spaces, enabling task-aware compliance modulation.

\textbf{Orthogonal Decomposition via Object Jacobian.} To enable motion decomposition, we first establish projection operators that separate motions into components parallel and orthogonal to the object's kinematic constraints. Let $J_i(\mathbf{q}_{\text{obj}}) \in \mathbb{R}^{6 \times k}$ denote the body Jacobian of the object for end-effector $i \in \{l, r\}$, which encodes how the object's joint velocities $\dot{\mathbf{q}}_{\text{obj}}$ manifest as end-effector body twists. The body Jacobian relates to the spatial Jacobian (Eq.~2) via the adjoint transformation: $J_i = \mathrm{Ad}_{T_{b_is}} J_s$, where $T_{b_is} = (T_{sb_i})^{-1}$ transforms spatial frame quantities to the body frame.

Using the inner product $\langle \mathcal{V}_1, \mathcal{V}_2 \rangle_G = \mathcal{V}_1^T G \mathcal{V}_2$ with metric tensor $G = \mathrm{diag}(\alpha^2 I_3, I_3)$ on $\mathfrak{se}(3)$, we construct orthogonal projection operators:

\begin{equation}
\label{eq:projection_operators}
\begin{aligned}
P_{i,\parallel} &= J_i (J_i^\top G J_i)^{-1} J_i^\top G, \quad P_{i,\perp} = I - P_{i,\parallel},
\end{aligned}
\end{equation}

where $P_{i,\parallel}$ projects onto the internal motion subspace (range of $J_i$) and $P_{i,\perp}$ projects onto the bulk motion subspace (orthogonal complement). These projectors satisfy $P_{i,\parallel}^T G = G P_{i,\parallel}$ and $P_{i,\perp}^T G = G P_{i,\perp}$, ensuring geometric consistency under the chosen metric. Importantly, since $\alpha$ is learned by the policy (Layer 3), the metric tensor $G$ and hence the projection operators adapt dynamically to task requirements, enabling context-dependent orthogonal decomposition.

\textbf{Controller Formulation.} With the learned impedance variables $a_t = (d_{l,\parallel}, d_{r,\parallel}, d_{l,\perp}, d_{r,\perp}, k_{p_l}, k_{p_r}, \alpha)$ from Layer 3, we construct a damping matrix that respects the motion decomposition:

\begin{equation}
\label{eq:damping_matrix}
K_{d_i} = G (P_{i,\parallel} d_{i,\parallel} + P_{i,\perp} d_{i,\perp}),
\end{equation}

This structure allows independent damping modulation for internal motion (via $d_{i,\parallel}$) and bulk motion (via $d_{i,\perp}$), enabling the policy to adaptively regulate compliance based on task requirements and force feedback.

The commanded wrench is then computed as:

\begin{equation}
\label{eq:impedance_control_main}
\begin{aligned}
\mathcal{F}_{\mathrm{cmd}, i} &= K_{d_i} (\mathcal{V}_i^{\text{ref}} - \mathcal{V}_i) + \mu_{b,i} + \gamma_{b,i}\\
&= K_{d_i} (\mathrm{Ad}_{T_{b_id_i}} \mathcal{V}_i^{\text{des}} - \mathcal{V}_i + k_{p_i} \mathcal{E}_i) + \mu_{b,i} + \gamma_{b,i},
\end{aligned}
\end{equation}

where $\mathcal{V}_i^{\text{ref}} = \mathrm{Ad}_{T_{b_id_i}} \mathcal{V}_i^{\text{des}} + k_{p_i} \mathcal{E}_i$ is the reference twist from Layer 2, $\mu_{b,i}$ accounts for nonlinear dynamics (Coriolis and centrifugal terms), and $\gamma_{b,i}$ provides gravity compensation (omitted in planar SE(2) settings where gravity is orthogonal to the motion plane). 

The commanded wrench is then mapped to joint-space motor torques via the manipulator Jacobian:

\begin{equation}
\label{eq:joint_torque}
\tau_{\mathrm{cmd}, i} = J_i(\theta_i)^T \mathcal{F}_{\mathrm{cmd}, i},
\end{equation}

where $J_i(\theta_i) \in \mathbb{R}^{6 \times n}$ is the geometric Jacobian of the $i$-th manipulator mapping joint velocities to end-effector twist, and $\tau_{\mathrm{cmd}, i} \in \mathbb{R}^n$ is the commanded joint torque vector that serves as the final robot control command.

We now provide two complementary interpretations that reveal the dual nature of this controller.

\textbf{Interpretation 1: SE(3) Impedance Control Structure.} 

The first interpretation reveals that our controller naturally implements SE(3) impedance control. Classical impedance control on SE(3) designs a virtual dynamical system with desired impedance characteristics:

\begin{equation}
M \dot{\xi} + D \xi + J_{\mathcal{E}}^\top K \mathcal{E} = \mathcal{F}_{\mathrm{ext}},
\end{equation}

where $\xi = {}^b\mathcal{V}_d - {}^b\mathcal{V}_b$ is the velocity error, $M$ is the desired inertia, $D$ is damping, and $J_{\mathcal{E}}^\top K \mathcal{E}$ is the nonlinear stiffness term arising from SE(3) geometry (Section~\ref{sec:problem_formulation}). The corresponding impedance controller takes the form:

\begin{equation}
\mathcal{F}_{\mathrm{cmd}} = \Lambda_b M^{-1}(D \xi + J_{\mathcal{E}}^\top K \mathcal{E}) + \Lambda_b {}^b\dot{\mathcal{V}}_d + \mu_b + \gamma_b + (I - \Lambda_b M^{-1})\mathcal{F}_{\mathrm{ext}},
\end{equation}

where $\Lambda_b$ is the operational space inertia matrix. Under the common simplifications $M = \Lambda_b$ (match desired and actual inertia) and ${}^b\dot{\mathcal{V}}_d = 0$ (constant reference velocity), this reduces to:

\begin{equation}
\mathcal{F}_{\mathrm{cmd}} = D \xi + J_{\mathcal{E}}^\top K \mathcal{E} + \mu_{b} + \gamma_{b}.
\end{equation}

Our controller in Eq.~\eqref{eq:impedance_control_main} follows this exact structure. Defining the velocity error as $\xi = \mathrm{Ad}_{T_{b_id_i}} \mathcal{V}_i^{\text{des}} - \mathcal{V}_i$, we can rewrite:

\begin{equation}
\label{eq:impedance_equivalence}
\begin{aligned}
\mathcal{F}_{\mathrm{cmd}, i} &= K_{d_i} (\mathrm{Ad}_{T_{b_id_i}} \mathcal{V}_i^{\text{des}} - \mathcal{V}_i + k_{p_i} \mathcal{E}_i) + \mu_{b,i} + \gamma_{b,i}\\
&= K_{d_i} \xi + K_{d_i} k_{p_i} \mathcal{E}_i + \mu_{b,i} + \gamma_{b,i}\\
&\approx D \xi + J_{\mathcal{E}}^\top K \mathcal{E} + \mu_{b} + \gamma_{b},
\end{aligned}
\end{equation}

where the correspondence is: $D \leftrightarrow K_{d_i}$ (learned damping) and the term $K_{d_i} k_{p_i} \mathcal{E}_i$ plays the role of $J_{\mathcal{E}}^\top K \mathcal{E}$ (stiffness). Critically, our approach \textbf{sidesteps the explicit nonlinear Jacobian $J_{\mathcal{E}}$} by incorporating the $k_{p_i} \mathcal{E}_i$ term directly into the reference twist (Layer 2), avoiding the geometric complications discussed in Section~\ref{sec:problem_formulation} while maintaining impedance behavior.

\textbf{Interpretation 2: Explicit Bulk-Internal Motion Decomposition.} 

The second interpretation reveals how the controller naturally decomposes control actions into semantically meaningful components aligned with task requirements. By decomposing both reference and actual twists into components parallel (internal) and orthogonal (bulk) to the object's kinematic constraints:

\begin{align}
\mathcal{V}_{i,\parallel}^{\text{ref}} &= P_{i,\parallel} \mathcal{V}_i^{\text{ref}}, \quad \mathcal{V}_{i,\parallel} = P_{i,\parallel} \mathcal{V}_i \quad \text{(internal motion)}, \\
\mathcal{V}_{i,\perp}^{\text{ref}} &= P_{i,\perp} \mathcal{V}_i^{\text{ref}}, \quad \mathcal{V}_{i,\perp} = P_{i,\perp} \mathcal{V}_i \quad \text{(bulk motion)},
\end{align}

we can expand the control law to reveal independent regulation of each motion component:

\begin{equation}
\label{eq:decomposed_control}
\begin{aligned}
\mathcal{F}_{\mathrm{cmd}, i} &= K_{d_i} (\mathcal{V}_i^{\text{ref}} - \mathcal{V}_i) + \mu_{b,i} + \gamma_{b,i}\\
&= G(P_{i,\parallel} d_{i,\parallel} + P_{i,\perp} d_{i,\perp})(\mathcal{V}_i^{\text{ref}} - \mathcal{V}_i) + \mu_{b,i} + \gamma_{b,i}\\
&= \underbrace{d_{i,\parallel} G (\mathcal{V}_{i,\parallel}^{\text{ref}} - \mathcal{V}_{i,\parallel})}_{\text{internal motion control}} + \underbrace{d_{i,\perp} G (\mathcal{V}_{i,\perp}^{\text{ref}} - \mathcal{V}_{i,\perp})}_{\text{bulk motion control}} + \mu_{b,i} + \gamma_{b,i}.
\end{aligned}
\end{equation}

This decomposition provides three critical properties:

\begin{enumerate}
    \item \textbf{Independent compliance modulation}: The policy can independently adjust $d_{i,\parallel}$ and $d_{i,\perp}$ to achieve task-specific compliance---high stiffness for bulk motion during transport, high compliance for internal motion during articulation, or vice versa.
    
    \item \textbf{Decoupled Power Generation}: The feedback wrenches for internal and bulk motions are \textbf{reciprocally orthogonal} to the opposing motion subspaces, ensuring zero interference in terms of virtual power:
    \begin{align}
 \mathcal{F}_{\mathrm{cmd,fb}, i,\parallel} &= d_{i,\parallel}G (\mathcal{V}_{i,\parallel}^{\text{ref}}-\mathcal{V}_{i,\parallel}) \\
\mathcal{F}_{\mathrm{cmd,fb}, i,\perp} &= d_{i,\perp} G (\mathcal{V}_{i,\perp}^{\text{ref}}-\mathcal{V}_{i,\perp})\\
(\mathcal{F}_{\mathrm{cmd,fb}, i,\parallel} )^\top (\mathcal{V}_{i,\perp}^{\text{ref}}-\mathcal{V}_{i,\perp}) &= 0\\
(\mathcal{F}_{\mathrm{cmd,fb}, i,\perp} )^\top (\mathcal{V}_{i,\parallel}^{\text{ref}}-\mathcal{V}_{i,\parallel}) &= 0 \, .
    \end{align}
    This orthogonality follows directly from the projection properties: $P_{i,\parallel}^T G P_{i,\perp} = 0$, ensuring that control actions for each motion type do not interfere with each other.
    
    \item \textbf{Constraint satisfaction}: The internal motion component $\mathcal{V}_{i,\parallel}$ automatically lies in the range of $J_i$, ensuring that commanded motions respect the object's kinematic constraints and minimize harmful internal forces.
\end{enumerate}

Together, these two interpretations demonstrate that our controller simultaneously achieves geometrically consistent SE(3) impedance behavior while enabling explicit, learning-based modulation of task-semantic motion components---a capability that would be intractable to design analytically given the geometric constraints discussed in Section~\ref{sec:problem_formulation}.


\subsection{Learning Framework}

\subsubsection{Reinforcement Learning Formulation}

We formulate the Low-Level Policy learning as a Partially Observable Markov Decision Process (POMDP) $\mathcal{M} = (\mathcal{S}, \mathcal{O}, \mathcal{A}, P, r, \gamma)$, where:
\begin{itemize}
    \item $\mathcal{S}$: State space (full environment state)
    \item $\mathcal{O}$: Observation space (partial observations available to the policy)
    \item $\mathcal{A}$: Action space
    \item $P(s_{t+1} | s_t, a_t)$: Transition dynamics
    \item $r: \mathcal{S} \times \mathcal{A} \to \mathbb{R}$: Reward function (Section~\ref{sec:reward_design})
    \item $\gamma$: Discount factor
\end{itemize}

The objective is to maximize expected return:

\begin{equation}
J(\theta) = \mathbb{E}_{\tau \sim \pi_\theta} \left[ \sum_{t=0}^{T} \gamma^t r(s_t, a_t) \right].
\end{equation}

\subsubsection{Reward Function Design}
\label{sec:reward_design}

The reward function balances three objectives for stable, force-compliant manipulation:

\begin{equation}
\label{eq:reward}
r_t = r_{\text{track}} + r_{\text{safety}} + r_{\text{reg}}.
\end{equation}

\textbf{Motion Tracking ($r_{\text{track}}$).} Ensures accurate tracking of reference trajectories generated by the motion field:

\begin{align}
\label{eq:reward_tracking}
r_{\text{track}} = -w_{\text{track}}\sum_{i \in \lbrace l,r \rbrace} \|\mathcal{V}_i - \mathcal{V}_i^{\text{ref}}\|_G^2 = -w_{\text{track}}\sum_{i \in \lbrace l,r \rbrace} (\mathcal{V}_i - \mathcal{V}_i^{\text{ref}})^T G (\mathcal{V}_i - \mathcal{V}_i^{\text{ref}}),
\end{align}

where $G = \mathrm{diag}(\alpha^2 I_3, I_3)$ is the learned metric tensor. This reward encourages each end-effector to follow its reference twist, where $\mathcal{V}_l, \mathcal{V}_r$ are the actual body twists and $\mathcal{V}_l^{\text{ref}}, \mathcal{V}_r^{\text{ref}}$ are the reference twists from the motion field. Using the G-metric ensures that tracking error is measured consistently with the impedance control framework, with adaptive weighting between rotational and translational components via the learned parameter $\alpha$.

\textbf{Safety ($r_{\text{safety}}$).} Ensures safe operation and minimizes non-productive internal forces:

\begin{align}
\label{eq:reward_safety}
r_{\text{safety}} = -w_{\text{int}} \sum_{i \in \lbrace l,r \rbrace} \|\mathcal{F}_{i,\perp}\|_2^2,
\end{align}

where $\mathcal{F}_i = \begin{bmatrix} m_i \\ f_i \end{bmatrix}$ is the measured wrench at each end-effector $i$.

This reward addresses the internal wrench problem identified in Section~\ref{sec:problem_formulation} by minimizing internal wrenches---wrench components orthogonal to the object's allowable motion direction. To decompose measured wrenches consistently with the twist decomposition framework in Layer 4, we seek wrench components $\mathcal{F}_{i,\parallel}$ and $\mathcal{F}_{i,\perp}$ such that they are orthogonal to complementary twist subspaces under the reciprocal product (virtual power). Specifically, we require:

\begin{align}
\langle \mathcal{F}_{i,\parallel}, \mathcal{V} \rangle &= \mathcal{F}_{i,\parallel}^T \mathcal{V} = 0 \quad \forall \mathcal{V} \in \text{range}(P_{i,\perp}), \\
\langle \mathcal{F}_{i,\perp}, \mathcal{V} \rangle &= \mathcal{F}_{i,\perp}^T \mathcal{V} = 0 \quad \forall \mathcal{V} \in \text{range}(P_{i,\parallel}),
\end{align}

where $\langle \mathcal{F}, \mathcal{V} \rangle = \mathcal{F}^T \mathcal{V}$ is the reciprocal product representing virtual power. This orthogonality condition is naturally satisfied by projecting the measured wrench using the transpose of the twist projection operators, exploiting the dual relationship between twist and wrench spaces:

\begin{equation}
\mathcal{F}_{i,\parallel} = P_{i,\parallel}^T \mathcal{F}_i, \quad \mathcal{F}_{i,\perp} = P_{i,\perp}^T \mathcal{F}_i = (I - P_{i,\parallel})^T \mathcal{F}_i.
\end{equation}

To verify orthogonality, for any $\mathcal{V} \in \text{range}(P_{i,\perp})$, we have $\mathcal{V} = P_{i,\perp} \mathcal{V}'$ for some $\mathcal{V}'$, and:
\begin{align}
\mathcal{F}_{i,\parallel}^T \mathcal{V} &= (P_{i,\parallel}^T \mathcal{F}_i)^T (P_{i,\perp} \mathcal{V}') = \mathcal{F}_i^T P_{i,\parallel} P_{i,\perp} \mathcal{V}' = 0,
\end{align}
where the last equality follows immediately from the orthogonality of the twist projectors ($P_{i,\parallel} P_{i,\perp} = 0$). Similarly, it can be shown that $\mathcal{F}_{i,\perp}^T \mathcal{V} = 0$ for all $\mathcal{V} \in \text{range}(P_{i,\parallel})$.

The parallel component $\mathcal{F}_{i,\parallel}$ represents productive wrench that performs work along the object's internal degree of freedom, contributing to desired joint motion. The orthogonal component $\mathcal{F}_{i,\perp}$ represents \textbf{internal wrench} that:
\begin{itemize}
    \item Does not contribute to desired object motion along the screw axis (zero virtual power along $\text{range}(P_{i,\parallel})$)
    \item Arises from coordination errors between the two arms
    \item Represents constraint forces (bearing loads, friction, etc.) unrelated to joint actuation
    \item Increases unnecessary contact stress and grasp instability
    \item Wastes energy and risks hardware damage
\end{itemize}

By penalizing $\|\mathcal{F}_{i,\perp}\|_2^2$, the policy learns to minimize non-productive forces while maintaining necessary productive forces for manipulation. This wrench decomposition is fully consistent with the twist decomposition framework, utilizing the duality of the learned kinematic structure.

\textbf{Regularization ($r_{\text{reg}}$).} Encourages smooth motion:

\begin{equation}
\label{eq:reward_regularization}
r_{\text{reg}} = - w_{reg} \sum_{i \in \lbrace l,r \rbrace} \|\dot{\mathcal{V}}_i\|^2.
\end{equation}

This reduces energy consumption (torque magnitude), joint jerkiness (joint acceleration), and Cartesian jerkiness (twist acceleration), promoting natural and efficient movements.

\textbf{Termination Conditions.} To ensure grasp stability, episodes terminate early (task failure) if grasp drift exceeds safety thresholds:

\begin{equation}
\text{Terminate if: } \exists i \in \{l, r\} \text{ such that } \left\|\left[\log\left((T_{\text{grip},i}^{\text{init}})^{-1} T_{\text{grip},i}\right)\right]^\vee\right\|_2 > d_{\max},
\end{equation}

where $T_{\text{grip},i}^{\text{init}}$ is the initial grasp pose, $T_{\text{grip},i}$ is the current grasp pose, and $d_{\max}$ is the maximum allowable drift threshold. This geodesic distance on SE(3) captures both translational and rotational drift from the initial grasp configuration. When this threshold is exceeded, the episode terminates immediately with a failure signal, encouraging the policy to maintain stable grasps throughout manipulation without explicit reward shaping.

\subsubsection{Policy Network Architecture}

The Low-Level Policy $\pi_\theta: \mathcal{S} \to \mathcal{A}$ is implemented as a neural network with \textbf{object-conditioned multi-stream architecture}. The network employs Feature-wise Linear Modulation (FiLM) to inject object geometric structure into all feature processing stages, enabling constraint-aware representation learning.

For detailed architecture specifications, see Appendix~\ref{app:network_architecture}.

\subsubsection{Training Procedure}

We employ \textbf{Proximal Policy Optimization (PPO)} with standard hyperparameters for policy gradient updates.

\textbf{PPO Objective:}

\begin{equation}
\label{eq:ppo}
L^{CLIP}(\theta) = \mathbb{E}_t \left[\min\left(r_t(\theta) \hat{A}_t, \text{clip}(r_t(\theta), 1-\epsilon, 1+\epsilon) \hat{A}_t\right)\right],
\end{equation}

where $r_t(\theta) = \pi_\theta(a_t|o_t) / \pi_{\theta_{\text{old}}}(a_t|o_t)$ and advantages are computed via Generalized Advantage Estimation (GAE):

\begin{equation}
\label{eq:gae}
\hat{A}_t = \sum_{l=0}^{\infty} (\gamma \lambda)^l \delta_{t+l}, \quad \delta_t = r_t + \gamma V_\phi(o_{t+1}) - V_\phi(o_t).
\end{equation}

where $V_\phi: \mathcal{O} \to \mathbb{R}$ is the value function approximated by a separate critic network with architecture identical to the policy encoder (shared encoders, separate value head).

\textbf{Algorithm Summary:} The complete training procedure is summarized in Algorithm~\ref{alg:SWIVL_training}.

\begin{algorithm}[t]
\caption{SWIVL Training}
\label{alg:SWIVL_training}
\begin{algorithmic}[1]
\REQUIRE Pre-trained High-Level Policy $\pi_{HL}$, object set $\mathcal{O}_{\text{obj}}$
\STATE Initialize: Low-Level Policy parameters $\theta$, value function parameters $\phi$
\FOR{episode = 1 to $N_{\text{episodes}}$}
    \STATE Sample object with task $o \sim \mathcal{O}_{\text{obj}}$
    \STATE Initialize robot, environment, and action chunk buffer
        \FOR{$t = 1$ to $H$}
            \STATE \textbf{High-Level Policy}
            \IF{$t \mod f_{HL}^{-1} == 0$}
                \STATE Generate action chunk: $\{T_{sd_i}[\tau]\}_{\tau=0}^{H_{\text{chunk}}} \gets \pi_{HL}$
            \ENDIF
            \STATE \textbf{Reference Twist Field Generator}
            \STATE Interpolate action chunk $\to$ dense trajectory $T_{sd_i}(t)$
            \STATE Compute desired body twists $\mathcal{V}_i^{\text{des}}(t)$ via Eq.~\ref{eq:body_twist}
            \STATE Apply stable vector field $\to$ reference twists $\mathcal{V}_i^{\text{ref}}$ via Eq.~\ref{eq:vector_field}
            \STATE Decompose $\mathcal{V}_i^{\text{ref}} \to$ internal motion $\mathcal{V}_{i,\parallel}^{\text{ref}}$ and bulk motion $\mathcal{V}_{i,\perp}^{\text{ref}}$
            \STATE \textbf{Low-Level Policy}
            \STATE Observe $o_t = (\mathcal{V}_i^{\text{ref}}, \mathcal{B}_i, \mathcal{F}_i, T_{sb_i}, \mathcal{V}_i)$
            \STATE Sample action $a_t = (d_{i,\parallel}, d_{i,\perp}, k_{p_i}, \alpha) \sim \pi_\theta(\cdot|o_t)$
            \STATE \textbf{Controller}
            \STATE Compute commanded wrench $\mathcal{F}_{\mathrm{cmd}, i}$ via Eq.~\ref{eq:impedance_control_main}
            \STATE Execute joint torque $\tau_{\mathrm{cmd}, i} = J_i(\theta_i)^T \mathcal{F}_{\mathrm{cmd}, i}$ via Eq.~\ref{eq:joint_torque}
            \STATE \textbf{Collect experience}
            \STATE Observe $o_{t+1}$, compute reward $r_t$ via Eq.~\ref{eq:reward}
            \STATE Store transition $(o_t, a_t, r_t, o_{t+1})$
        \ENDFOR
    \STATE \textbf{PPO Update}
    \STATE Compute advantages via GAE (Eq.~\ref{eq:gae})
    \FOR{epoch = 1 to 10}
        \STATE Sample mini-batches from buffer
        \STATE Update $\theta$ via Eq.~\ref{eq:ppo}
        \STATE Update $\phi$ via MSE loss on value targets
    \ENDFOR
\ENDFOR
\end{algorithmic}
\end{algorithm}

% Experiments - Revised Version
\section{Experiments}
\label{sec:experiments}

We evaluate SWIVL on bimanual manipulation of articulated objects in an SE(2) planar benchmark. Our experiments address three key questions:
\begin{itemize}[leftmargin=2em]
    \item[\textbf{Q1.}] Does SWIVL improve task success and reduce fighting forces compared to imitation learning baselines?
    \item[\textbf{Q2.}] How do SWIVL's design choices---stable vector fields, screw-decomposed control, and wrench-adaptive learning---contribute to performance?
    \item[\textbf{Q3.}] Can SWIVL generalize across diverse high-level planners and novel object instances?
\end{itemize}

%------------------------------------------------------------------------------
\subsection{Experimental Setup}
\label{sec:exp_setup}
%------------------------------------------------------------------------------

\paragraph{SE(2) Benchmark Rationale.}
While our method (Section~\ref{sec:method}) is formulated for SE(3) with $k$-DoF articulated objects, we validate in SE(2) with 1-DoF objects. This deliberate simplification enables rigorous, large-scale evaluation while preserving the essential challenges: \textbf{(i)} the fundamental phenomena---force coupling, constraint satisfaction, compliant coordination---manifest identically in planar and spatial settings; \textbf{(ii)} all architectural components (projection operators $P_{i,\parallel}, P_{i,\perp}$, metric tensor $G(\alpha)$, impedance modulation $d_\parallel, d_\perp$) remain fully exercised; and \textbf{(iii)} the mathematical structure (Lie group, screw theory, twist-wrench duality) reduces consistently from SE(3). Extension to SE(3) requires only scaling observation/action dimensions. See Appendix~\ref{app:se2} for complete SE(2) instantiation.

\paragraph{Environment.}
We use a 512$\times$512 pixel planar workspace with dual 3-DoF end-effectors under direct body wrench control $\mathcal{F}_i = [m_z, f_x, f_y]^\top$. Each end-effector provides 3-axis F/T sensing at control frequency. The hierarchical architecture combines high-level planning (10 Hz) with SWIVL's low-level control (50 Hz). Physics and workspace specifications are in Appendix~\ref{app:environment_settings}.

\paragraph{Tasks and Objects.}
We evaluate on \textbf{9 articulated objects} spanning three joint types (Figure~\ref{fig:objects}):
\begin{itemize}[leftmargin=1.5em]
    \item \textbf{Fixed} (3 variants): Rigid transport---no internal DoF ($\mathcal{S} = \mathbf{0}$)
    \item \textbf{Revolute} (3 variants): Angular articulation---rotation about a pivot
    \item \textbf{Prismatic} (3 variants): Linear articulation---sliding along an axis
\end{itemize}
Each object satisfies the SE(2) holonomic constraint ${}^s\mathcal{V}_l - {}^s\mathcal{V}_r = \mathcal{S}\dot{q}_{obj}$ with constant body-frame screw axes $\mathcal{B}_l, \mathcal{B}_r \in \mathbb{R}^3$. Tasks require manipulating objects from randomized initial configurations to fixed goal configurations. \textbf{Success criteria}: position error $<$10 pixels, orientation error $<$5°, joint error $<$5° or 5 pixels, with maintained grasp. Each configuration is tested over \textbf{100 trials}.

% ============================================================================
% FIGURE PLACEHOLDER: Object illustrations
% ============================================================================
% \begin{figure}[t]
%     \centering
%     \includegraphics[width=\linewidth]{figures/objects.pdf}
%     \caption{\textbf{Benchmark objects.} Nine SE(2) articulated objects spanning three joint types: fixed (rigid transport), revolute (angular articulation), and prismatic (linear articulation). Each joint type includes 3 variants with different geometries and mass distributions.}
%     \label{fig:objects}
% \end{figure}
% ============================================================================

\paragraph{Implementation.}
The policy observes reference twists $\mathcal{V}_i^{\text{ref}}$, screw axes $\mathcal{B}_i$, wrenches $\mathcal{F}_i$, and proprioception ($\mathbb{R}^{30}$ total), and outputs impedance variables $(d_{i,\parallel}, d_{i,\perp}, k_{p_i}, \alpha) \in \mathbb{R}^7$. Training uses PPO with the reward from Eq.~\eqref{eq:reward}. Full architecture and hyperparameters are in Appendix~\ref{app:network_architecture}.

\paragraph{Baselines.}
We compare against:
\begin{itemize}[leftmargin=1.5em]
    \item \textbf{OWIL} (Object-Wrench Imitation Learning): Behavior cloning baseline receiving the same object and wrench information as SWIVL, but outputting direct end-effector twists $\mathcal{V}_l, \mathcal{V}_r$ without screw-decomposed control or learned impedance. Trained on 5,000 expert demonstrations.
    \item \textbf{BC-Stiff}: Standard behavior cloning with high-stiffness position control, representing typical VLA/imitation learning deployment.
\end{itemize}

\paragraph{Ablations.}
To isolate SWIVL's contributions, we ablate four design axes (Table~\ref{tab:ablations}):
\begin{itemize}[leftmargin=1.5em]
    \item[\textbf{A.}] \textbf{Observation composition}: Effect of bulk-internal decomposition and wrench feedback
    \item[\textbf{B.}] \textbf{Vector field design}: Necessity of stability term $k_p\mathcal{E}$ in reference generation
    \item[\textbf{C.}] \textbf{Action parameterization}: Screw-decomposed impedance vs. residual corrections
    \item[\textbf{D.}] \textbf{Object conditioning}: FiLM modulation vs. direct concatenation
\end{itemize}

% ============================================================================
% TABLE: Ablation variants
% ============================================================================
\begin{table}[t]
\centering
\caption{\textbf{Ablation variants.} Each ablation isolates one design choice while keeping others fixed.}
\label{tab:ablations}
\small
\begin{tabular}{llp{6cm}}
\toprule
\textbf{Axis} & \textbf{Variant} & \textbf{Modification} \\
\midrule
\multirow{2}{*}{A. Observation} 
    & SWIVL-IndivRef & No bulk-internal decomposition in observation \\
    & SWIVL-NoWrench & Remove wrench feedback $\mathcal{F}_i$ \\
\midrule
\multirow{2}{*}{B. Vector Field} 
    & SWIVL-TempOnly & Pure temporal tracking ($k_p = 0$) \\
    & SWIVL-SpatialField & Spatial contraction without temporal sync \\
\midrule
C. Action Space 
    & SWIVL-Residual & Residual twist corrections instead of impedance \\
\midrule
D. Conditioning 
    & SWIVL-Concat & Concatenate screw axes instead of FiLM \\
\bottomrule
\end{tabular}
\end{table}

\paragraph{Metrics.}
\begin{itemize}[leftmargin=1.5em]
    \item \textbf{Success Rate} (\%): Task completion within error thresholds
    \item \textbf{Fighting Force} $F_{\text{fight}}$ (N): Time-averaged bulk wrench magnitude $\frac{1}{T}\sum_t \|\mathcal{F}_{i,\perp}(t)\|$
    \item \textbf{Constraint Violation} (px/s): Deviation from holonomic constraint $\|{}^s\mathcal{V}_l - {}^s\mathcal{V}_r - \mathcal{S}\dot{q}_{obj}\|$
    \item \textbf{Tracking RMSE} (px): End-effector trajectory error
\end{itemize}

%------------------------------------------------------------------------------
\subsection{Main Results}
\label{sec:main_results}
%------------------------------------------------------------------------------

\subsubsection{Q1: SWIVL vs. Imitation Learning Baselines}

% ============================================================================
% TABLE PLACEHOLDER: Main comparison results
% ============================================================================
\begin{table}[t]
\centering
\caption{\textbf{Comparison with baselines.} SWIVL vs. imitation learning approaches across 9 objects (100 trials each). Bold indicates best; $\pm$ shows 95\% CI.}
\label{tab:main_results}
\small
\begin{tabular}{lccccc}
\toprule
\textbf{Method} & \textbf{Success (\%)} & \textbf{$F_{\text{fight}}$ (N) $\downarrow$} & \textbf{CViol (px/s) $\downarrow$} & \textbf{RMSE (px) $\downarrow$} \\
\midrule
\multicolumn{5}{l}{\textit{Fixed Joint (Rigid Transport)}} \\
BC-Stiff & [--] & [--] & [--] & [--] \\
OWIL & [--] & [--] & [--] & [--] \\
SWIVL & [--] & [--] & [--] & [--] \\
\midrule
\multicolumn{5}{l}{\textit{Revolute Joint}} \\
BC-Stiff & [--] & [--] & [--] & [--] \\
OWIL & [--] & [--] & [--] & [--] \\
SWIVL & [--] & [--] & [--] & [--] \\
\midrule
\multicolumn{5}{l}{\textit{Prismatic Joint}} \\
BC-Stiff & [--] & [--] & [--] & [--] \\
OWIL & [--] & [--] & [--] & [--] \\
SWIVL & [--] & [--] & [--] & [--] \\
\midrule
\multicolumn{5}{l}{\textit{All Objects (Average)}} \\
BC-Stiff & [--] & [--] & [--] & [--] \\
OWIL & [--] & [--] & [--] & [--] \\
\textbf{SWIVL} & \textbf{[--]} & \textbf{[--]} & \textbf{[--]} & \textbf{[--]} \\
\bottomrule
\end{tabular}
\end{table}

Table~\ref{tab:main_results} compares SWIVL against imitation learning baselines across all objects.

\textbf{Key findings:}
\begin{itemize}[leftmargin=1.5em]
    \item \textit{Success rate}: [Expected: SWIVL achieves higher success rates, particularly on articulated objects where constraint satisfaction is critical]
    \item \textit{Fighting force reduction}: [Expected: SWIVL significantly reduces $F_{\text{fight}}$ by learning to suppress bulk wrench components through compliant impedance modulation]
    \item \textit{Joint-type analysis}: [Expected: Improvements most pronounced on revolute/prismatic joints where internal-bulk decomposition provides clearest benefit; fixed joints show smaller gaps as no articulation constraint exists]
\end{itemize}

% ============================================================================
% FIGURE PLACEHOLDER: Force profiles comparison
% ============================================================================
% \begin{figure}[t]
%     \centering
%     \includegraphics[width=\linewidth]{figures/force_comparison.pdf}
%     \caption{\textbf{Fighting force profiles.} Time evolution of bulk wrench $\|\mathcal{F}_{i,\perp}\|$ during revolute manipulation. SWIVL maintains lower fighting forces throughout the trajectory compared to BC-Stiff and OWIL.}
%     \label{fig:force_comparison}
% \end{figure}
% ============================================================================

\subsubsection{Q2: Ablation Studies}

% ============================================================================
% TABLE PLACEHOLDER: Ablation results
% ============================================================================
\begin{table}[t]
\centering
\caption{\textbf{Ablation study.} Impact of each design choice on overall performance (averaged across all objects).}
\label{tab:ablation_results}
\small
\begin{tabular}{lcccc}
\toprule
\textbf{Variant} & \textbf{Success (\%)}  & \textbf{$F_{\text{fight}}$ (N)} & \textbf{CViol (px/s)} \\
\midrule
\textbf{SWIVL (Full)} & \textbf{[--]} & \textbf{[--]} & \textbf{[--]} \\
\midrule
\multicolumn{4}{l}{\textit{A. Observation Composition}} \\
\quad SWIVL-IndivRef & [--] & [--] & [--] \\
\quad SWIVL-NoWrench & [--] & [--] & [--] \\
\midrule
\multicolumn{4}{l}{\textit{B. Vector Field Design}} \\
\quad SWIVL-TempOnly & [--] & [--] & [--] \\
\quad SWIVL-SpatialField & [--] & [--] & [--] \\
\midrule
\multicolumn{4}{l}{\textit{C. Action Parameterization}} \\
\quad SWIVL-Residual & [--] & [--] & [--] \\
\midrule
\multicolumn{4}{l}{\textit{D. Object Conditioning}} \\
\quad SWIVL-Concat & [--] & [--] & [--] \\
\bottomrule
\end{tabular}
\end{table}

Table~\ref{tab:ablation_results} isolates each design contribution.

\textbf{A. Observation Composition.}
[Expected: Removing bulk-internal decomposition (IndivRef) degrades performance by losing task-semantic structure. Removing wrench feedback (NoWrench) increases fighting forces as the policy cannot sense coordination errors.]

\textbf{B. Vector Field Design.}
[Expected: Pure temporal tracking (TempOnly) fails under trajectory deviations from contact forces. Spatial-only fields (SpatialField) provide correction but may sacrifice temporal consistency. The combined stable imitation field balances both.]

\textbf{C. Action Parameterization.}
[Expected: Residual corrections (Residual) cannot structurally enforce constraint satisfaction, relying on implicit learning through reward penalties. Screw-decomposed impedance provides principled compliance modulation.]

\textbf{D. Object Conditioning.}
[Expected: FiLM conditioning enables joint-type-specific feature modulation, outperforming simple concatenation especially when generalizing across revolute/prismatic/fixed objects.]

\subsubsection{Q3: Generalization}

\paragraph{Cross-Planner Transfer.}
We evaluate SWIVL trained with one high-level planner and tested with others:

% ============================================================================
% TABLE PLACEHOLDER: Cross-planner results
% ============================================================================
\begin{table}[t]
\centering
\caption{\textbf{Cross-planner generalization.} Success rates (\%) when SWIVL trained with HLP-Diff is tested with different high-level planners (zero-shot transfer).}
\label{tab:cross_planner}
\small
\begin{tabular}{lccc}
\toprule
\textbf{Test Planner} & \textbf{Fixed} & \textbf{Revolute} & \textbf{Prismatic} \\
\midrule
HLP-Diff (same) & [--] & [--] & [--] \\
HLP-ACT & [--] & [--] & [--] \\
HLP-Teleop & [--] & [--] & [--] \\
\bottomrule
\end{tabular}
\end{table}

[Expected: SWIVL maintains performance across planners, validating that the low-level controller is planner-agnostic. The reference twist field interface successfully decouples cognitive planning from physical execution.]

\paragraph{Novel Object Transfer.}
We test on 6 held-out object variants (scaled geometry, asymmetric mass, different inertia):

[Expected: SWIVL generalizes to novel objects within the same joint type category, with performance degradation primarily when kinematic structure differs significantly. FiLM conditioning on screw axes enables adaptation without retraining.]

%------------------------------------------------------------------------------
\subsection{Analysis}
\label{sec:analysis}
%------------------------------------------------------------------------------

\paragraph{Learned Impedance Behavior.}
% ============================================================================
% FIGURE PLACEHOLDER: Impedance variable visualization
% ============================================================================
% \begin{figure}[t]
%     \centering
%     \includegraphics[width=\linewidth]{figures/impedance_analysis.pdf}
%     \caption{\textbf{Learned impedance modulation.} (a) Damping coefficients $d_\parallel, d_\perp$ over task phases. (b) Characteristic length $\alpha$ adaptation across joint types. (c) Correlation between wrench feedback and impedance adjustment.}
%     \label{fig:impedance_analysis}
% \end{figure}
% ============================================================================

[Expected analysis: (i) The policy learns task-phase-dependent compliance---high $d_\perp$ during transport, high $d_\parallel$ during articulation. (ii) The learned $\alpha$ differs across joint types, discovering task-appropriate metric structures. (iii) Wrench feedback triggers predictive impedance adjustment before large fighting forces develop.]

\paragraph{Failure Mode Analysis.}
[Expected: Primary failure modes include (i) grasp slip when fighting forces exceed friction limits, (ii) trajectory timeout on complex articulation sequences, (iii) collision with workspace boundaries. SWIVL reduces (i) through force regulation while baselines frequently fail due to excessive contact stress.]

\paragraph{Computational Efficiency.}
[Expected: Reference twist field generation achieves $O(1)$ per timestep. Policy inference meets 50 Hz control requirements. Overall latency compatible with real-time deployment.]

%------------------------------------------------------------------------------
\subsection{Summary}
\label{sec:exp_summary}
%------------------------------------------------------------------------------

Our experiments validate SWIVL's core hypothesis: \textbf{explicit encoding of geometric constraints and wrench feedback enables robust bimanual manipulation that pure imitation learning cannot achieve}. Key findings:
\begin{enumerate}[leftmargin=2em]
    \item SWIVL outperforms imitation baselines in success rate while significantly reducing fighting forces---demonstrating the necessity of physical intelligence for contact-rich bimanual tasks.
    \item Each architectural component contributes: stable vector fields handle trajectory deviations, screw-decomposed control provides principled compliance, and wrench-adaptive learning discovers task-appropriate impedance strategies.
    \item The framework generalizes across high-level planners and novel objects, validating the hierarchical separation of cognitive and physical intelligence.
\end{enumerate}

These SE(2) results instantiate the general SE(3) methodology, providing strong evidence that SWIVL's principles will transfer to full spatial manipulation.
% Discussion
\section{Discussion}
\label{sec:discussion}

\textbf{Note}: This section outlines the analysis framework for interpreting experimental results. Concrete findings will be added upon completion of evaluation.

\subsection{Key Findings: SWIVL in SE(2) Bimanual Manipulation}

Our SE(2) experiments with a single-joint articulated object confirm that \textbf{explicit encoding of geometric and wrench structure} is essential for robust dual-arm manipulation. We summarize key findings along three axes: (1) the benefit of SWIVL's physics-aware reinforcement learning over pure imitation learning, (2) the impact of each architectural component in the SE(2), 1-DoF instantiation, and (3) generalization within the planar benchmark across planners and objects.

	extbf{Physics-Aware RL vs. Pure Imitation Learning.} [Results comparing SWIVL against the OWIL baseline will reveal the fundamental advantage of reinforcement learning with explicit physical modeling over behavior cloning. Expected findings: SWIVL achieves higher success rates while significantly reducing internal forces and constraint violations, demonstrating that physical intelligence requires more than trajectory imitation---it demands autonomous discovery of force-compliant strategies through interaction with the environment.]

\textbf{Role of Explicit Constraint Encoding.} [Analysis of SE(2) kinematic constraint violations will show whether the impedance-modulated controller with projection-based motion decomposition successfully enforces the 1-DoF holonomic constraint $ {}^s\mathcal{V}_l - {}^s\mathcal{V}_r = \mathcal{S} \dot{q}_{obj}$ through the structural projection operators $P_{i,\parallel}$ and $P_{i,\perp}$, eliminating the need for penalty-based reward engineering. Expected finding: SWIVL maintains near-zero constraint violations throughout manipulation, while residual-based ablations and OWIL struggle despite reward penalties.]

	extbf{Wrench Feedback for Force Regulation.} [Quantitative analysis of internal force patterns, using the SE(2) wrench decomposition in Appendix~\ref{app:orthogonal_decomposition}, will demonstrate how explicit wrench sensing enables active minimization of fighting forces. Expected finding: Wrench-aware SWIVL variants suppress harmful internal forces by identifying and counteracting wrench components orthogonal to the screw axis, while wrench-blind variants and OWIL cannot reliably distinguish between internal and bulk forces.]

\subsection{Architectural Design Choices and Their Impact}

	extbf{Bulk--Internal Motion Decomposition in SE(2).} [Analysis will examine whether providing explicit task semantics through SE(2) bulk--internal decomposition, built from constant body-frame screw axes $\mathcal{B}_l, \mathcal{B}_r$ and spatial inertia matrices, improves learning efficiency and generalization. Expected findings: (1) Policies with decomposition learn faster by exploiting a structured observation space tied to the 1-DoF constraint, (2) decomposition enables task-agnostic behavior---the same SE(2) policy executes transport-focused, articulation-focused, and coordinated tasks without retraining, (3) interpretability of learned behaviors improves through semantically meaningful bulk and internal motion primitives.]

	extbf{Stable Imitation Vector Field vs. Temporal Tracking.} [Robustness analysis under perturbations in the planar benchmark will validate the necessity of spatial correction mechanisms. Expected findings: (1) Pure temporal tracking baselines fail when initial conditions deviate from demonstrations, (2) purely spatial contraction fields provide correction but may overly prioritize spatial convergence at the cost of timing, (3) SWIVL's stable imitation vector field balances temporal alignment with contraction, enabling recovery from tracking errors while following the planner's desired timing.]

	extbf{FiLM Conditioning for Object Generalization.} [Cross-object evaluation within the SE(2) benchmark will assess how architectural choices affect adaptation to novel kinematic structures with different screw axes and inertial parameters. Expected findings: FiLM-based feature modulation enables dynamic adjustment of control strategies based on joint type, body-frame screw axes, and planar inertia parameters, outperforming concatenation-based conditioning in zero-shot transfer to unseen object geometries and mass distributions.]

\subsection{Generalization and Transfer Capabilities}

	extbf{Cross-Planner Generalization in SE(2).} [Evaluation with diverse high-level planners (HLP-Diff, HLP-ACT, HLP-Teleop) will test SWIVL's planner-agnostic property under the SE(2) instantiation. Expected findings: (1) SWIVL successfully tracks SE(2) action chunks from all planner types without retraining, validating the reference twist field interface as a clean separation layer, (2) performance remains consistent across planning paradigms, demonstrating separation of cognitive and physical intelligence layers, (3) planner-specific characteristics (smoothness, horizon length, noise) are successfully handled by the stable imitation vector field design.]

	extbf{Zero-Shot Transfer to Novel Planar Objects.} [Testing on scaled, asymmetric, and mass-varied SE(2) objects will quantify geometric and dynamic generalization in the 1-DoF setting. Expected findings: (1) Geometric generalization succeeds when kinematic structure (screw axis, joint type) is preserved but scale changes, (2) dynamic variations (mass, spatial inertia) require adaptation but benefit from explicit wrench feedback and inertia conditioning, (3) failure modes emerge when assumptions break down (e.g., multi-DoF objects, unknown or time-varying constraints), motivating the full SE(3), multi-DoF extension discussed below.]

\textbf{Sample Efficiency and Training Dynamics.} [Learning curve analysis within the SE(2) benchmark will compare data requirements across variants. Expected findings: (1) Structured SE(2) impedance action spaces (damping coefficients $d_{\parallel}, d_{\perp}$ and stiffness gains $k_{p_i}$ that parameterize the projection-based controller) improve sample efficiency by reducing the policy search space and ensuring feasibility, (2) explicit constraint encoding through projection operators accelerates training by eliminating exploration of infeasible action regions, (3) wrench feedback and orthogonal wrench decomposition provide a dense, physically meaningful signal that speeds up learning of force regulation.]

\subsection{Physical Intelligence Through Learned Behaviors}

\textbf{Emergent Force-Compliant Strategies in SE(2).} [Qualitative analysis of learned planar behaviors will reveal how RL discovers physically intelligent solutions even in the reduced SE(2) setting. Expected observations: (1) Adaptive compliance---the policy modulates damping coefficients $d_{\parallel}$ and $d_{\perp}$ based on task phase (high stiffness for aggressive motion when free, high compliance near joint limits and contacts), (2) metric adaptation---the learned characteristic length scale $\alpha$ dynamically adjusts the SE(2) metric tensor $G = \mathrm{diag}(\alpha^2, 1, 1)$, which defines the inner product for orthogonal decomposition of twists and wrenches, enabling task-appropriate separation of bulk versus internal motion components, (3) predictive force regulation---the policy anticipates violations of the SE(2) holonomic constraint and preemptively adjusts impedance variables before large internal forces develop.]

\textbf{Interpretability of Bulk--Internal Decomposition.} [Visualization of learned SE(2) impedance modulation patterns will validate semantic meaningfulness. Expected findings: (1) Pure transport tasks exhibit high damping $d_{\perp}$ on bulk motion components and low damping $d_{\parallel}$ on internal motion, resulting in stiff transport with compliant joint articulation, (2) articulation tasks show the opposite pattern with high $d_{\parallel}$ and low $d_{\perp}$, enabling precise joint control while allowing bulk motion compliance, (3) coordinated tasks balance both damping coefficients, (4) transitions between task phases correspond to smooth modulation of the damping ratio $d_{\parallel}/d_{\perp}$ in the SE(2) controller.]

	extbf{Computational Efficiency and Real-Time Feasibility.} [Performance profiling in simulation will assess deployment readiness. Expected measurements: (1) SE(2) reference twist field generation achieves $O(1)$ computation per timestep as claimed, (2) the low-level policy meets 50 Hz control requirements with sub-10 ms latency, (3) overall SWIVL computation remains compatible with real-time execution, supporting future SE(3) hardware deployment.]

\subsection{Limitations and Future Work}

	extbf{Current Limitations:}
\begin{itemize}
\item \textbf{SE(2) planar setting:} Real-world tasks require full SE(3) workspace beyond the single-joint, planar benchmark studied here
\item \textbf{Simulation evaluation:} Sim-to-real transfer remains to be validated
\item \textbf{Known object models:} Assumes screw axis $\mathcal{S}$ and body-frame axes $\mathcal{B}_l, \mathcal{B}_r$ provided a priori
\item \textbf{Single-DoF constraints:} Limited to revolute, prismatic, or fixed 1-DoF joints
\item \textbf{Pre-grasped objects:} Grasping and regrasping not addressed
\end{itemize}

\textbf{Future Directions:}
\begin{enumerate}
\item \textbf{SE(3) extension}: Generalize to 6-DoF manipulation with full spatial twists $\mathcal{V} \in \mathbb{R}^6$ and metric tensor $G = \mathrm{diag}(\alpha^2 I_3, I_3) \in \mathbb{R}^{6 \times 6}$
\item \textbf{Real robot deployment}: Domain randomization for sim-to-real transfer on dual-arm platforms (Franka Panda, UR5e)
\item \textbf{Multi-DoF articulation}: Extend to $k$-DoF objects with Jacobian $J_i \in \mathbb{R}^{6 \times k}$ and per-arm impedance variables $(d_{l,\parallel}, d_{r,\parallel}, d_{l,\perp}, d_{r,\perp})$ as in Method Section 3.2.3
\item \textbf{Online constraint learning}: Estimate unknown screw axes through exploratory interaction
\item \textbf{End-to-end visuomotor control}: Integrate visual perception for object pose and constraint prediction
\item \textbf{Theoretical analysis}: Formal stability guarantees and optimality characterization
\end{enumerate}

\subsection{Broader Impact}

\textbf{Applications:} Bimanual assembly in manufacturing, surgical assistance in healthcare, cooperative manipulation in service robotics.

\textbf{Research Contributions:} Principled integration of geometric structure into learning-based control, demonstrating how explicit physical constraints enhance robustness and generalization.

\textbf{Safety Considerations:} Explicit internal force minimization reduces contact stress, improving safety in human-robot interaction scenarios.

\subsection{Summary}

Our analysis framework focuses on three key dimensions: (1) force regulation patterns to understand physical intelligence, (2) robustness under perturbations to validate stability guarantees, and (3) computational efficiency to assess real-time feasibility. Results will demonstrate how explicitly encoding kinematic constraints, wrench feedback, and contraction stability enables robust, efficient bimanual manipulation of articulated objects.


\section{Conclusion}
\label{sec:conclusion}

We introduced \textbf{SWIVL}, a hierarchical framework that bridges high-level cognitive planning with low-level physically grounded execution for bimanual manipulation of articulated objects. By explicitly incorporating object kinematic constraints and end-effector wrench feedback, SWIVL operationalizes \textbf{Physical Intelligence}---enabling force-compliant coordination, constraint satisfaction, and robust tracking of diverse high-level planners.

Our key contributions include: (1) a \textbf{Stable Imitation Vector Field} with $O(1)$ computational complexity and guaranteed exponential convergence, (2) a \textbf{bulk-internal motion decomposition} that provides interpretable task semantics, (3) a \textbf{kinematic-constrained action space} that structurally enforces holonomic constraints, (4) an \textbf{object-conditioned policy architecture} using FiLM conditioning for generalization across joint types, and (5) comprehensive evaluation demonstrating necessity of explicit physical modeling.

SWIVL is designed to operate beneath arbitrary cognitive planners---including VLA-based foundation models, behavior cloning policies, and teleoperation interfaces---without requiring planner-specific tuning. This modularity enables seamless integration with emerging high-level reasoning systems while ensuring safe, physically feasible execution.

\textbf{Future Directions.} While our SE(2) evaluation isolates core challenges of force coupling and constraint satisfaction, extending to full SE(3) manipulation requires minimal architectural changes (scaling observation/action dimensions) and is naturally supported by our formulation. Key directions include: (1) real-world validation with domain randomization for sim-to-real transfer, (2) online constraint learning for unknown objects, (3) multi-DoF articulated structures, (4) end-to-end visuomotor control integrating perception with low-level policy, and (5) theoretical analysis of stability guarantees and optimality.

By decoupling \textbf{Cognitive Intelligence} (high-level semantic reasoning) from \textbf{Physical Intelligence} (low-level force-compliant execution), SWIVL advances toward robotic systems that effectively unify both dimensions---a critical step for deploying learning-based policies in real-world contact-rich manipulation.

% Bibliography
\bibliographystyle{plainnat}
\bibliography{references}

% Appendix
\appendix

% Appendix: Notation and Mathematical Preliminaries
\section{Notation and Mathematical Preliminaries}
\label{app:notation}

This appendix establishes the mathematical notation used throughout the paper, following the modern robotics framework \cite{park2017modern} by Frank C. Park. We adopt Lie group formalism for SE(3) and SE(2), which provides a geometric foundation for manipulation.

\subsection{Coordinate Frames and Basic Notation}

\textbf{Reference Frames:}
\begin{itemize}
\item $\lbrace s \rbrace$: Spatial (world) frame
\item $\lbrace l \rbrace, \lbrace r \rbrace$: Left and right end-effector body frames
\item $\lbrace o \rbrace$: Object body frame
\end{itemize}

\textbf{Twist Notation:}
\begin{itemize}
\item $\mathcal{V}_a$: Twist of frame $\lbrace a \rbrace$ in its own body frame
\item ${}^b\mathcal{V}_a$: Twist of frame $\lbrace a \rbrace$ expressed in frame $\lbrace b \rbrace$
\end{itemize}

\subsection{SE(3) and SE(2) Configuration Spaces}

\textbf{Special Euclidean Groups:}
\begin{itemize}
\item $SE(3) = \left\{ T = \begin{bmatrix} R & p \\ 0 & 1 \end{bmatrix} : R \in SO(3), p \in \mathbb{R}^3 \right\}$: Rigid body transformations in 3D
\item $SE(2) = \left\{ T = \begin{bmatrix} R & p \\ 0 & 1 \end{bmatrix} : R \in SO(2), p \in \mathbb{R}^2 \right\}$: Planar rigid transformations
\item $T_{ab}$: Transformation from frame $\lbrace b \rbrace$ to frame $\lbrace a \rbrace$
\end{itemize}

\subsection{Twists and Wrenches}

\textbf{Twist (Spatial Velocity):} A twist $\mathcal{V}$ represents the instantaneous velocity of a rigid body, combining angular and linear components:
\begin{itemize}
\item \textbf{SE(3)}: $\mathcal{V} = \begin{bmatrix} \omega \\ v \end{bmatrix} \in \mathbb{R}^6$ where $\omega \in \mathbb{R}^3$ is angular velocity and $v \in \mathbb{R}^3$ is linear velocity
\item \textbf{SE(2)}: $\mathcal{V} = \begin{bmatrix} \omega_z \\ v_x \\ v_y \end{bmatrix} \in \mathbb{R}^3$ where $\omega_z$ is angular velocity about z-axis and $(v_x, v_y)$ are planar linear velocities
\item \textbf{Body twist} $\mathcal{V}_a$: Twist expressed in the moving body frame $\lbrace a \rbrace$
\item \textbf{Spatial twist} ${}\mathcal{V}_s$: Twist expressed in the spatial frame $\lbrace s \rbrace$
\end{itemize}

\textbf{Wrench (Generalized Force):} A wrench $\mathcal{F}$ represents the generalized force acting on a rigid body, combining moment and force:
\begin{itemize}
\item \textbf{SE(3)}: $\mathcal{F} = \begin{bmatrix} m \\ f \end{bmatrix} \in \mathbb{R}^6$ where $m \in \mathbb{R}^3$ is moment (torque) and $f \in \mathbb{R}^3$ is force
\item \textbf{SE(2)}: $\mathcal{F} = \begin{bmatrix} m_z \\ f_x \\ f_y \end{bmatrix} \in \mathbb{R}^3$ where $m_z$ is moment about z-axis and $(f_x, f_y)$ are planar forces
\item Wrenches naturally pair with twists via power: $P = \mathcal{F}^T \mathcal{V} = m^T \omega + f^T v$
\end{itemize}

\textbf{Adjoint Transformation:} Transforms twists between coordinate frames:
\begin{equation*}
{}^a\mathcal{V}_c = [Ad_{T_{ab}}] {}^b\mathcal{V}_c
\end{equation*}
where the adjoint matrix is:
\begin{itemize}
\item \textbf{SE(3)}: $[Ad_T] = \begin{bmatrix} R & 0 \\ [p]_\times R & R \end{bmatrix} \in \mathbb{R}^{6 \times 6}$ for $T = \begin{bmatrix} R & p \\ 0 & 1 \end{bmatrix}$
\item \textbf{SE(2)}: $[Ad_T] = \begin{bmatrix}
1 & 0 & 0 \\
y & \cos\theta & -\sin\theta \\
-x & \sin\theta & \cos\theta
\end{bmatrix} \in \mathbb{R}^{3 \times 3}$ for pose $(x, y, \theta)$
\end{itemize}

Twists transform via the adjoint: ${}^a\mathcal{V}_c = [Ad_{T_{ab}}] {}^b\mathcal{V}_c$.
Wrenches transform via the dual adjoint: ${}^a\mathcal{F}_c = [Ad_{T_{ab}^{-1}}]^T {}^b\mathcal{F}_c$.

\subsection{Screw Theory}

\textbf{Screw Axis:} A screw $\mathcal{S}$ describes the instantaneous motion axis of a rigid body. For a unit twist $\mathcal{V}$ (i.e., $\|\omega\| = 1$ or $\omega = 0$), the screw is the twist itself.

\begin{itemize}
\item \textbf{SE(3) Screw}: $\mathcal{S} = \begin{bmatrix} \omega \\ v \end{bmatrix} \in \mathbb{R}^6$
\begin{itemize}
\item Rotational screw ($\|\omega\| = 1$): $\mathcal{S} = \begin{bmatrix} \omega \\ -\omega \times q \end{bmatrix}$ where $q$ is a point on the axis
\item Translational screw ($\omega = 0$): $\mathcal{S} = \begin{bmatrix} 0 \\ v \end{bmatrix}$ where $\|v\| = 1$
\end{itemize}

\item \textbf{SE(2) Screw}: $\mathcal{S} = \begin{bmatrix} \omega_z \\ v_x \\ v_y \end{bmatrix} \in \mathbb{R}^3$
\begin{itemize}
\item Pure rotation ($|\omega_z| = 1$): Center of rotation at $(c_x, c_y) = (-v_y/\omega_z, v_x/\omega_z)$
\item Pure translation ($\omega_z = 0$): $\mathcal{S} = \begin{bmatrix} 0 \\ v_x \\ v_y \end{bmatrix}$ where $\sqrt{v_x^2 + v_y^2} = 1$
\end{itemize}
\end{itemize}

\textbf{Exponential Coordinates:} Any rigid body displacement can be represented as screw motion:
\begin{equation*}
T = e^{[\mathcal{S}]\theta}
\end{equation*}
where $[\mathcal{S}] \in \mathfrak{se}(3)$ or $\mathfrak{se}(2)$ is the matrix representation of the screw, and $\theta$ is the magnitude.

\textbf{Rodrigues' Formula:}
\begin{itemize}
\item \textbf{SE(3)}: For $\mathcal{S} = \begin{bmatrix} \omega \\ v \end{bmatrix}$ with $\|\omega\| = 1$,
\begin{equation*}
e^{[\mathcal{S}]\theta} = \begin{bmatrix} e^{[\omega]\theta} & (I\theta + (1-\cos\theta)[\omega] + (\theta-\sin\theta)[\omega]^2)v \\ 0 & 1 \end{bmatrix}
\end{equation*}

\item \textbf{SE(2)}: For $\mathcal{S} = \begin{bmatrix} \omega_z \\ v_x \\ v_y \end{bmatrix}$,
\begin{equation*}
e^{[\mathcal{S}]\theta} = \begin{bmatrix}
\cos(\omega_z\theta) & -\sin(\omega_z\theta) & \frac{v_x\sin(\omega_z\theta) + v_y(1-\cos(\omega_z\theta))}{\omega_z} \\
\sin(\omega_z\theta) & \cos(\omega_z\theta) & \frac{v_y\sin(\omega_z\theta) - v_x(1-\cos(\omega_z\theta))}{\omega_z} \\
0 & 0 & 1
\end{bmatrix}
\end{equation*}
\end{itemize}

\textbf{Velocity from Exponential Coordinates:} The body twist is:
\begin{equation*}
\mathcal{V} = \mathcal{S}\dot{\theta}
\end{equation*}
This relationship connects the configuration space velocity $\dot{\theta}$ to the geometric velocity (twist) $\mathcal{V}$.

\textbf{Product of Exponentials (POE):} Forward kinematics can be expressed as:
\begin{equation*}
T(\theta) = e^{[\mathcal{S}_1]\theta_1} e^{[\mathcal{S}_2]\theta_2} \cdots e^{[\mathcal{S}_n]\theta_n} M
\end{equation*}
where $\mathcal{S}_i$ are the joint screws at zero configuration and $M$ is the home configuration.

\subsection{Summary of Key Notation}

\begin{table}[h]
\centering
\begin{tabular}{ll}
\hline
\textbf{Symbol} & \textbf{Description} \\
\hline
$SE(3)$, $SE(2)$ & Special Euclidean groups (spatial transformations) \\
$T_{ab}$ & Transformation from frame $\lbrace b \rbrace$ to frame $\lbrace a \rbrace$ \\
$\mathcal{V}$, ${}^b\mathcal{V}_a$ & Twist (body/spatial velocity) \\
$\mathcal{F}$ & Wrench (generalized force) \\
$\mathcal{S}$, $\mathcal{B}_i$ & Screw axis (spatial/body frame) \\
$[Ad_T]$ & Adjoint matrix for twist transformation \\
$\lbrace s \rbrace$, $\lbrace l \rbrace$, $\lbrace r \rbrace$ & Spatial, left, right reference frames \\
$\dot{q}_{\text{obj}}$ & Object's internal joint velocity \\
\hline
\end{tabular}
\caption{Summary of mathematical notation used throughout the paper.}
\end{table}

\textbf{Reference:} Notation follows Park \& Lynch (2017), \textit{Modern Robotics}.


% Appendix: SE(3) and SE(2) Trajectory Interpolation
\section{SE(3) and SE(2) Trajectory Interpolation}
\label{app:trajectory_interpolation}

This section describes trajectory smoothing methods for both SE(3) and SE(2) formulations. The high-level policy generates discrete waypoints at low frequency, while the low-level policy requires smooth, dense trajectories at high frequency.

\subsection{SE(3) Trajectory Interpolation}

\subsubsection{Rotation Representation Pipeline}

While the neural network uses \textbf{rotation\_6d} representation for training stability and continuity, trajectory interpolation requires conversion to quaternions for SLERP:

$$
\text{rotation\_6d} \xrightarrow{\text{inverse transform}} \text{quaternion} \xrightarrow{\text{SLERP}} \text{interpolated trajectory}
$$

This conversion ensures proper handling of rotation manifold geometry while maintaining computational efficiency.

\subsubsection{Geodesic Interpolation}

Given discrete waypoints $\lbrace T_{si}^{des}[\tau_k] \rbrace_{k=0}^{H}$ at times $\lbrace t_k \rbrace_{k=0}^{H}$, we construct a smooth trajectory $T_{si}^{des}(t)$ using SE(3) geodesics. For $t \in [t_k, t_{k+1}]$, the interpolated transformation is:

$$
T_{si}^{des}(t) = T_{si}^{des}[\tau_k] \exp\left(\alpha(t) \log\left(T_{si}^{des}[\tau_k]^{-1} T_{si}^{des}[\tau_{k+1}]\right)\right),
$$

where $\alpha(t) \in [0,1]$ is a smooth interpolation parameter. To ensure $C^1$ continuity in velocity, we use cubic interpolation:

$$
\alpha(t) = 3s^2 - 2s^3, \quad s = \frac{t - t_k}{t_{k+1} - t_k}.
$$

This formulation provides the shortest path on SE(3) between consecutive waypoints while maintaining smooth velocity profiles. However, it requires matrix exponential/logarithm operations and may encounter numerical instability for small rotations.

\subsubsection{Decoupled Interpolation (Recommended)}

A more practical approach decouples translation and rotation, offering numerical stability and independent velocity constraints:

\textbf{Translation:} For position $\mathbf{p}_{si}(t) \in \mathbb{R}^3$, we support two interpolation schemes:

\textit{Linear Interpolation} (computationally efficient):
$$
\mathbf{p}_{si}(t) = \mathbf{p}_{si}[\tau_k] + s \cdot (\mathbf{p}_{si}[\tau_{k+1}] - \mathbf{p}_{si}[\tau_k]), \quad s = \frac{t - t_k}{t_{k+1} - t_k}
$$

\textit{Cubic Spline Interpolation} (smoother velocity profiles):
$$
\mathbf{p}_{si}(t) = \mathbf{a}_3 s^3 + \mathbf{a}_2 s^2 + \mathbf{a}_1 s + \mathbf{a}_0,
$$

where coefficients $\lbrace \mathbf{a}_j \rbrace$ are determined by boundary conditions (positions and velocities at waypoints).

\textbf{Rotation:} Use Spherical Linear Interpolation (SLERP) for quaternions $\mathbf{q}_{si}(t)$:

$$
\mathbf{q}_{si}(t) = \mathbf{q}_{si}[\tau_k] \left(\mathbf{q}_{si}[\tau_k]^{-1} \mathbf{q}_{si}[\tau_{k+1}]\right)^{\alpha(t)},
$$

where:
\begin{itemize}
\item Quaternion exponentiation is defined via the exponential map on SO(3)
\item $\alpha(t) = s$ for linear SLERP, or $\alpha(t) = 3s^2 - 2s^3$ for cubic smoothing
\item SLERP ensures geodesic path (shortest rotation) on the SO(3) manifold
\item Preserves unit norm: $\|\mathbf{q}_{si}(t)\| = 1$ for all $t$
\item Numerically stable for all rotation magnitudes
\end{itemize}

\textbf{Combined Transformation:} Construct $T_{si}^{des}(t)$ from $\mathbf{p}_{si}(t)$ and $R_{si}(t) = \text{quat2mat}(\mathbf{q}_{si}(t))$:

$$
T_{si}^{des}(t) = \begin{bmatrix} R_{si}(t) & \mathbf{p}_{si}(t) \\ \mathbf{0}^\top & 1 \end{bmatrix} \in \text{SE}(3).
$$

\subsubsection{Distance Metrics and Velocity Constraints}

For trajectory planning with kinematic limits, we define SE(3) distance as decoupled metrics:

\textbf{Position Distance:}
$$
d_{\text{pos}}(\mathbf{p}_1, \mathbf{p}_2) = \|\mathbf{p}_2 - \mathbf{p}_1\|_2 \quad \text{(meters)}
$$

\textbf{Rotation Distance:}
$$
d_{\text{rot}}(\mathbf{q}_1, \mathbf{q}_2) = \left\|(\mathbf{q}_2 \mathbf{q}_1^{-1})\right\|_{\text{angle}} \quad \text{(radians)}
$$

where $\|\mathbf{q}\|_{\text{angle}}$ denotes the rotation angle magnitude of quaternion $\mathbf{q}$.

When adding waypoints with velocity constraints, compute minimum duration:

$$
\Delta t_{\min} = \max\left(\frac{d_{\text{pos}}}{v_{\max}}, \frac{d_{\text{rot}}}{\omega_{\max}}\right),
$$

where $v_{\max}$ is maximum linear velocity (m/s) and $\omega_{\max}$ is maximum angular velocity (rad/s). This ensures both translation and rotation constraints are satisfied simultaneously.

\subsubsection{Implementation Notes}

\begin{itemize}
\item \textbf{Frequency:} High-Level Policy generates waypoints at $f_{HL} = 5$ Hz (action chunks of $H = 10$ steps); Low-Level Policy requires trajectories at $f_{LL} = 50$ Hz, requiring interpolation ratio $f_{LL}/f_{HL} = 10$.
\item \textbf{Horizon:} For action chunks of $H = 10$ waypoints, the smoothed trajectory contains $H_{LL} = H \cdot (f_{LL}/f_{HL}) = 100$ samples.
\item \textbf{Time Clipping:} Queries outside $[t_0, t_H]$ are clipped to boundary values.
\item \textbf{Numerical Stability:} SLERP handles all rotation magnitudes robustly, including near-identity rotations, avoiding singularities present in matrix logarithm approaches.
\end{itemize}

\subsubsection{Comparison: Coupled vs. Decoupled Approaches}

\textbf{Coupled SE(3) Geodesic:}
\begin{itemize}
\item True geodesic on SE(3) manifold with mathematically elegant formulation
\item Requires computationally expensive matrix exponential/logarithm
\item Numerical instability for small rotations (log singularity)
\item Cannot independently constrain translation/rotation velocities
\end{itemize}

\textbf{Decoupled Interpolation:}
\begin{itemize}
\item Numerically stable and computationally efficient
\item Independent velocity constraints for translation and rotation
\item Flexible interpolation schemes (linear, cubic, etc.)
\item Geodesic on SO(3) subgroup (though not on full SE(3))
\item Widely adopted in practical robotic applications
\end{itemize}

\subsection{SE(2) Trajectory Smoothing}

For planar manipulation tasks, trajectory smoothing is simplified to 2D.

\subsubsection{Planar Position Interpolation}

Given discrete waypoints $\{T_{si}^{des}[\tau]\}_{\tau=0}^H = \{(x[\tau], y[\tau], \theta[\tau])\}_{\tau=0}^H$ from the high-level planner at 10 Hz, we generate dense trajectories at 50 Hz (Low-Level Policy frequency).

\textbf{Position Interpolation:}
Cubic spline interpolation through position waypoints $(x[\tau], y[\tau])$:

$$
\begin{bmatrix} x(t) \\ y(t) \end{bmatrix} = \sum_{j=0}^{3} a_j s^j, \quad s = \frac{t - t_k}{t_{k+1} - t_k}
$$

where coefficients $\{a_j\}$ satisfy boundary conditions (positions and velocities at waypoints).

\subsubsection{Planar Orientation Interpolation}

Circular interpolation on SO(2) ensuring shortest path:

$$
\theta(t) = \theta_k + \text{wrap}(\theta_{k+1} - \theta_k) \cdot \phi(s)
$$

where:
\begin{itemize}
\item $\phi(s) = 3s^2 - 2s^3$ (cubic smoothing for $C^1$ continuity)
\item $\text{wrap}(\Delta\theta) = \text{atan2}(\sin\Delta\theta, \cos\Delta\theta)$ ensures $|\Delta\theta| \leq \pi$ (shortest angular path)
\end{itemize}

\subsubsection{Body Twist Computation}

From smooth trajectory $T_{si}^{des}(t)$, compute desired body twist via time differentiation:

$$
\mathcal{V}_i^{des}(t) = \begin{bmatrix} \dot{\theta}(t) \\ \dot{x}(t)\cos\theta(t) + \dot{y}(t)\sin\theta(t) \\ -\dot{x}(t)\sin\theta(t) + \dot{y}(t)\cos\theta(t) \end{bmatrix} \in \mathbb{R}^3
$$

This formulation transforms spatial velocities $(\dot{x}, \dot{y})$ into the body frame using the current orientation $\theta(t)$.

\subsubsection{Implementation Details}

\textbf{Frequency Matching:}
\begin{itemize}
\item High-level planner: 10 Hz (10 waypoints per chunk)
\item Low-level policy: 50 Hz (100 interpolated poses per chunk)
\item Interpolation ratio: 5× upsampling
\end{itemize}

\textbf{Smoothness Guarantees:}
\begin{itemize}
\item Position: $C^2$ continuous (cubic splines)
\item Orientation: $C^1$ continuous (cubic blending function)
\item Velocity: $C^0$ continuous at waypoints
\end{itemize}

\textbf{Edge Cases:}
\begin{itemize}
\item \textbf{Orientation wrapping:} Handle $\theta$ discontinuities at $\pm\pi$ using atan2
\item \textbf{Zero velocity waypoints:} Use natural spline boundary conditions
\item \textbf{Stationary goals:} Exponential decay to final pose or zero-velocity boundary condition
\end{itemize}

% Appendix: Impedance Control on SE(3)
\section{Impedance Control on SE(3)}
\label{app:impedance}

This appendix presents a systematic derivation of geometrically consistent impedance control on SE(3) for the compliance controller used in SWIVL's low-level execution. Starting from the definition of an inner product on the Lie algebra $\mathfrak{se}(3)$ representing the kinetic energy of an isotropic rigid body, we extend this to a Riemannian metric on SE(3), derive geodesics through variational principles, and construct a virtual mass-spring-damper system that respects the manifold structure. Finally, we couple this virtual system with the robot's operational space dynamics to derive the controller implementation.

\subsection{Notation Conventions}

We adopt the following notation for impedance control derivation:
\begin{itemize}
\item ${}^a\mathcal{V}_b$: Twist of frame $b$ expressed in frame $a$
\item $T_b = (R_b, p_b)$: Current end-effector pose (body frame $b$)
\item $T_d = (R_d, p_d)$: Desired end-effector pose (desired frame $d$)
\item $T_{bd} = T_b^{-1} T_d$: Relative transformation from current to desired
\item ${}^b\mathcal{V}_b = (\omega_b, v_b)$: Current body twist
\item ${}^d\mathcal{V}_d = (\omega_d, v_d)$: Desired body twist
\item $\alpha \in \mathbb{R}^+$: Characteristic length weighting rotational cost
\end{itemize}

\subsection{Virtual System Design on SE(3)}

\subsubsection{Inner Product on $\mathfrak{se}(3)$ as Kinetic Energy}

We begin by defining an inner product on the Lie algebra $\mathfrak{se}(3)$, corresponding to the tangent space at identity. Let $\hat{\mathcal{V}}_1, \hat{\mathcal{V}}_2 \in \mathfrak{se}(3)$ be twist elements with coordinates $\mathcal{V}_1 = (\omega_1, v_1)$ and $\mathcal{V}_2 = (\omega_2, v_2)$.

To provide a clear physical interpretation, we use an inner product representing an isotropic rigid body's kinematic energy. Using characteristic length scale $\alpha$, we define the metric coefficients as $\alpha^2$ for rotation and $1$ for translation (normalizing the mass term):
\begin{equation}
\langle \hat{\mathcal{V}}_1, \hat{\mathcal{V}}_2 \rangle_I = \frac{\alpha^2}{2} \mathrm{tr}([\omega_1]^\top [\omega_2]) + v_1^\top v_2 = \alpha^2 \omega_1^\top \omega_2 + v_1^\top v_2 = \mathcal{V}_1^\top G \mathcal{V}_2
\end{equation}
where $G = \mathrm{diag}(\alpha^2 I_3, I_3)$ is the inertia matrix.

This naturally defines the kinetic energy $K$ of an isotropic rigid body with body twist ${}^b\mathcal{V}_b$:
\begin{equation}
K = \frac{1}{2} \langle \hat{\mathcal{V}}_b, \hat{\mathcal{V}}_b \rangle_I = \frac{1}{2} {}^b\mathcal{V}_b^\top G \,{}^b\mathcal{V}_b
\end{equation}

\subsubsection{Riemannian Metric Extension}

We extend the inner product to a left-invariant Riemannian metric on SE(3). For tangent vectors $\dot{T}_1, \dot{T}_2 \in T_T \mathrm{SE}(3)$:
\begin{equation}
\langle \dot{T}_1, \dot{T}_2 \rangle_T = \langle T^{-1}\dot{T}_1, T^{-1}\dot{T}_2 \rangle_I
\end{equation}

\subsubsection{Geodesics and Action Minimization}

A geodesic minimizes the action integral along the manifold. For a curve $T(t) \in \mathrm{SE}(3)$ with body twist ${}^b\mathcal{V}_b(t) = (\omega(t), v(t))$, the action integral is:
\begin{equation}
S = \int_{t_0}^{t_f} \langle {}^b\mathcal{V}_b(t), {}^b\mathcal{V}_b(t) \rangle_{T(t)} \, dt = \int_{t_0}^{t_f} {}^b\mathcal{V}_b^\top G \,{}^b\mathcal{V}_b \, dt = \int_{t_0}^{t_f} \left( \alpha^2 \|\omega(t)\|^2 + \|v(t)\|^2 \right) dt
\end{equation}

The Euler-Poincaré equations for the decoupled metric $G = \mathrm{diag}(\alpha^2 I, I)$ are:
\begin{align}
\alpha^2 \dot{\omega} + \omega \times (\alpha^2 \omega) &= 0 \quad \Rightarrow \quad \dot{\omega} = 0 \\
\dot{v} + \omega \times v &= 0 \quad \Rightarrow \quad \dot{v} = -\omega \times v
\end{align}

Solving with initial conditions $\omega(0) = \omega_0$ and $v(0) = v_0$ yields:

\textbf{Rotational component:}
\begin{equation}
\omega(t) = \omega_0 \quad \text{(constant angular velocity)}
\end{equation}

\textbf{Translational component:}
\begin{equation}
v(t) = e^{-[\omega_0]t} v_0 = R(t)^\top v_0
\end{equation}
where $R(t) = e^{[\omega_0]t}$ is the rotation matrix. These solutions describe motion of an isotropic rigid body: constant angular velocity about a fixed axis in the body frame, with linear velocity maintaining constant direction in the spatial frame.

\subsubsection{Geodesic Distance and Weighted Pose Error}

The geodesic distance between poses $T_b = (R_b, p_b)$ and $T_d = (R_d, p_d)$ is computed by integrating the Riemannian metric along the geodesic path. The squared geodesic distance is:
\begin{equation}
d^2(T_b, T_d) = \alpha^2 \big\| \log(R_b^\top R_d)^\vee \big\|^2 + \big\| p_d - p_b \big\|^2
\end{equation}

This represents the minimum action required to move from $T_b$ to $T_d$ under the Riemannian metric.

We define unweighted pose error components in the body frame:
\begin{equation}
\begin{aligned}
e_p &= R_b^\top(p_d - p_b) \in \mathbb{R}^3 \quad \text{(translation error)} \\
e_R &= \log(R_b^\top R_d)^\vee \in \mathbb{R}^3 \quad \text{(rotation error)}
\end{aligned}
\end{equation}

where $e_p$ is the position difference vector expressed in body frame coordinates, and $e_R$ is the rotation vector representing the required rotation from $R_b$ to $R_d$ in body frame.

The \textbf{weighted pose error vector} incorporates the characteristic length $\alpha$:
\begin{equation}
\mathcal{E} = \begin{pmatrix} \alpha e_R \\ e_p \end{pmatrix} \in \mathbb{R}^6
\end{equation}

With this definition, $\|\mathcal{E}\|^2 = \alpha^2 \|e_R\|^2 + \|e_p\|^2 = d^2(T_b,T_d)$, so $\mathcal{E}$ is a Euclidean representation whose norm equals the SE(3) geodesic distance.

\subsubsection{Potential Energy and Elastic Wrench}

We define the potential energy using a symmetric positive semi-definite stiffness matrix $K \in \mathbb{R}^{6 \times 6}$:
\begin{equation}
P(\mathcal{E}) = \frac{1}{2} \mathcal{E}^\top K \, \mathcal{E}
\end{equation}

For regulation tasks with static desired pose ($\dot{T}_d = 0$), we derive the error time derivatives in terms of body twist ${}^b\mathcal{V}_b = (\omega_b, v_b)$.

\textbf{Translation Error Rate:} With $e_p = R_b^\top(p_d - p_b)$, using $\dot{R}_b = R_b[\omega_b]$ and $\dot{p}_b = R_b v_b$:
\begin{equation}
\begin{aligned}
\dot{e}_p &= \frac{d}{dt}(R_b^\top)(p_d - p_b) + R_b^\top(\dot{p}_d - \dot{p}_b) \\
&= (R_b[\omega_b])^\top(p_d - p_b) - R_b^\top R_b v_b \\
&= [\omega_b]^\top e_p - v_b = -[\omega_b] e_p - v_b = [e_p] \omega_b - v_b
\end{aligned}
\end{equation}
where we used $[\omega_b]^\top = -[\omega_b]$ and the identity $-\omega_b \times e_p = [e_p] \omega_b$.

\textbf{Rotation Error Rate:} Let $R_{err} = R_b^\top R_d$ so that $e_R = \log(R_{err})^\vee$. Differentiating for static $\dot{R}_d = 0$:
\begin{equation}
\dot{R}_{err} = \dot{R}_b^\top R_d = (R_b[\omega_b])^\top R_d = -[\omega_b] R_b^\top R_d = -[\omega_b] R_{err}
\end{equation}

From Lie group theory, if $\dot{R} = [\omega_s]R$ then $\dot{\theta} = J_l^{-1}(\theta)\,\omega_s$, where $J_l$ is the left Jacobian of SO(3):
\begin{equation}
J_l(\theta) = I + \frac{1 - \cos \|\theta\|}{\|\theta\|^2} [\theta] + \frac{\|\theta\| - \sin \|\theta\|}{\|\theta\|^3} [\theta]^2
\end{equation}

Here $\omega_s = -\omega_b$, so:
\begin{equation}
\dot{e}_R = -J_l^{-1}(e_R) \omega_b
\end{equation}

The weighted error rate is:
\begin{equation}
\dot{\mathcal{E}} = \begin{pmatrix} \alpha \dot{e}_R \\ \dot{e}_p \end{pmatrix} = \begin{pmatrix} -\alpha J_l^{-1}(e_R) \, \omega_b \\ [e_p] \, \omega_b - v_b \end{pmatrix} = -J_{\mathcal{E}} \, {}^b\mathcal{V}_b
\end{equation}

with the \textbf{weighted error Jacobian}:
\begin{equation}
J_{\mathcal{E}} = \begin{pmatrix} \alpha J_l^{-1}(e_R) & 0_{3 \times 3} \\ -[e_p] & I_{3} \end{pmatrix} \in \mathbb{R}^{6 \times 6}
\end{equation}

By power duality, the elastic wrench satisfies $\dot{P} = {}^b\mathcal{V}_b^\top \mathcal{F}_{\mathrm{elastic}}$:
\begin{equation}
\dot{P} = \frac{\partial P}{\partial \mathcal{E}}^\top \dot{\mathcal{E}} = (K \mathcal{E})^\top \dot{\mathcal{E}} = (-J_{\mathcal{E}}^\top K \mathcal{E})^\top {}^b\mathcal{V}_b
\end{equation}

yielding:
\begin{equation}
\boxed{\mathcal{F}_{\mathrm{elastic}} = -J_{\mathcal{E}}^\top K \mathcal{E}}
\end{equation}

Expanding with $\mathcal{E} = \begin{pmatrix} \alpha e_R \\ e_p \end{pmatrix}$ and $K = \begin{pmatrix} K_{RR} & K_{Rp} \\ K_{pR} & K_{pp} \end{pmatrix}$:
\begin{equation}
\begin{aligned}
m_{\mathrm{elastic}} &= -\alpha J_l^{-\top}(e_R) \, (K_{RR} \, \alpha e_R + K_{Rp} \, e_p) - e_p \times (K_{pR} \, \alpha e_R + K_{pp} \, e_p) \\
f_{\mathrm{elastic}} &= -K_{pR} \, \alpha e_R - K_{pp} \, e_p
\end{aligned}
\end{equation}

\subsubsection{Twist Error and Kinetic Energy}

Given current body twist ${}^b\mathcal{V}_b$ and desired body twist ${}^d\mathcal{V}_d$, we compute their difference in the current body frame using the Adjoint map. Let $T_{bd} = T_b^{-1} T_d$ with $R_{bd} = R_b^\top R_d$ and $p_{bd} = R_b^\top(p_d - p_b)$. The Adjoint transformation is:
\begin{equation}
\Ad_{T_{bd}} = \begin{pmatrix} R_{bd} & 0 \\ [p_{bd}]R_{bd} & R_{bd} \end{pmatrix}
\end{equation}

The twist error in the body frame is:
\begin{equation}
\xi = {}^b\mathcal{V}_d - {}^b\mathcal{V}_b = \Ad_{T_{bd}} {}^d\mathcal{V}_d - {}^b\mathcal{V}_b
\end{equation}

The kinetic energy of the virtual system is defined as:
\begin{equation}
K_{\mathrm{virtual}}(\xi) = \frac{1}{2} \xi^\top M \xi
\end{equation}
where $M \in \mathbb{R}^{6 \times 6}$ is the positive-definite virtual mass (inertia) matrix.

Assuming $M$ is constant in the body frame, the rate of change of kinetic energy is:
\begin{equation}
\dot{K}_{\mathrm{virtual}} = \frac{d}{dt} \left( \frac{1}{2} \xi^\top M \xi \right) = \xi^\top M \dot{\xi}
\end{equation}

The inertial wrench is $M\dot{\xi}$, analogous to $ma$ in Newton's second law.

\subsubsection{Complete Virtual Dynamics}

The complete virtual system follows the power balance equation. The total energy $E = K_{\mathrm{virtual}} + P$ evolves according to:
\begin{equation}
\dot{E} = P_{\mathrm{ext}} - P_{\mathrm{diss}}
\end{equation}

where external power is $P_{\mathrm{ext}} = \mathcal{F}_{\mathrm{ext}}^\top \xi$ and dissipated power is $P_{\mathrm{diss}} = \xi^\top D \xi$ with symmetric positive-definite damping matrix $D \in \mathbb{R}^{6 \times 6}$.

Expanding the power balance:
\begin{equation}
\frac{d}{dt}\left(\frac{1}{2} \xi^\top M \xi\right) + \frac{d}{dt}\left(\frac{1}{2} \mathcal{E}^\top K \mathcal{E}\right) = \mathcal{F}_{\mathrm{ext}}^\top \xi - \xi^\top D \xi
\end{equation}

\begin{equation}
\xi^\top \left[ M \dot{\xi} + J_{\mathcal{E}}^\top K \mathcal{E} \right] = \xi^\top \left[ \mathcal{F}_{\mathrm{ext}} - D \xi \right]
\end{equation}

This yields the virtual mass-spring-damper system:
\begin{equation}
\boxed{M \dot{\xi} + D \xi + J_{\mathcal{E}}^\top K \mathcal{E} = \mathcal{F}_{\mathrm{ext}}}
\end{equation}

where $M \dot{\xi}$ is the inertial term, $D \xi$ is the damping term, $J_{\mathcal{E}}^\top K \mathcal{E}$ is the elastic wrench, and $\mathcal{F}_{\mathrm{ext}}$ is external excitation.

\subsection{Impedance Controller Implementation}

\subsubsection{Operational Space Dynamics}

The robot's joint space dynamics are described by the Euler-Lagrange equations:
\begin{equation}
M(q)\ddot{q} + C(q,\dot{q})\dot{q} + g(q) = \tau - J_b^\top \mathcal{F}_{\mathrm{ext}}
\end{equation}

where $q \in \mathbb{R}^n$ are joint positions, $M(q)$ is the joint space inertia matrix, $C(q,\dot{q})$ includes Coriolis and centrifugal effects, $g(q)$ is gravity, $\tau$ are joint torques, and $\mathcal{F}_{\mathrm{ext}}$ is external wrench.

The body twist ${}^b\mathcal{V}_b = (\omega_b, v_b)$ relates to joint velocities through the body Jacobian $J_b(q) \in \mathbb{R}^{6 \times n}$:
\begin{equation}
{}^b\mathcal{V}_b = J_b(q) \dot{q}, \quad \dot{\mathcal{V}}_b = J_b(q) \ddot{q} + \dot{J}_b(q,\dot{q}) \dot{q}
\end{equation}

Projecting to operational space with operational space inertia $\Lambda_b(q) = (J_b M^{-1} J_b^\top)^{-1}$:
\begin{equation}
\Lambda_b(q) \dot{\mathcal{V}}_b + \mu_b(q, \dot{q}) + \gamma_b(q) = \mathcal{F}_{\mathrm{cmd}} - \mathcal{F}_{\mathrm{ext}}
\end{equation}

where:
\begin{itemize}
\item $\Lambda_b(q) = (J_b M^{-1} J_b^\top)^{-1}$: Operational space inertia (symmetric positive-definite)
\item $\mu_b(q, \dot{q}) = \Lambda_b(q) J_b M^{-1} C \dot{q} - \Lambda_b \dot{J}_b \dot{q}$: Coriolis and centrifugal wrench
\item $\gamma_b(q) = \Lambda_b(q) J_b M^{-1} g(q)$: Gravity wrench in body frame
\item $\mathcal{F}_{\mathrm{cmd}} \in \mathbb{R}^6$: Control wrench related to joint torques by $\tau = J_b^\top \mathcal{F}_{\mathrm{cmd}}$
\end{itemize}

\subsubsection{Controller Design by Virtual-Robot Coupling}

To achieve desired impedance behavior, we couple the virtual system dynamics with the robot's operational space dynamics. The key is to match the closed-loop robot behavior to the virtual mass-spring-damper system.

We have two dynamic systems:
\begin{itemize}
\item \textbf{Virtual System:} $M \dot{\xi} + D \xi + J_{\mathcal{E}}^\top K \mathcal{E} = \mathcal{F}_{\mathrm{ext}}$
\item \textbf{Robot Dynamics:} $\Lambda_b(q) \dot{\mathcal{V}}_b + \mu_b(q, \dot{q}) + \gamma_b(q) = \mathcal{F}_{\mathrm{cmd}} - \mathcal{F}_{\mathrm{ext}}$
\end{itemize}

Recall $\xi = {}^b\mathcal{V}_d - {}^b\mathcal{V}_b$, so $\dot{\xi} = {}^b\dot{\mathcal{V}}_d - {}^b\dot{\mathcal{V}}_b$.

\textbf{Step 1:} Solve for $\dot{\xi}$ from the virtual system:
\begin{equation}
\dot{\xi} = -M^{-1}\left(D \xi + J_{\mathcal{E}}^\top K \mathcal{E} - \mathcal{F}_{\mathrm{ext}}\right)
\end{equation}

\textbf{Step 2:} Substitute into $\dot{\xi} = {}^b\dot{\mathcal{V}}_d - {}^b\dot{\mathcal{V}}_b$ and solve for ${}^b\dot{\mathcal{V}}_b$:
\begin{equation}
{}^b\dot{\mathcal{V}}_b = {}^b\dot{\mathcal{V}}_d + M^{-1}\left(D \xi + J_{\mathcal{E}}^\top K \mathcal{E} - \mathcal{F}_{\mathrm{ext}}\right)
\end{equation}

\textbf{Step 3:} Substitute into robot dynamics:
\begin{equation}
\Lambda_b \left[{}^b\dot{\mathcal{V}}_d + M^{-1}\left(D \xi + J_{\mathcal{E}}^\top K \mathcal{E} - \mathcal{F}_{\mathrm{ext}}\right)\right] + \mu_b + \gamma_b = \mathcal{F}_{\mathrm{cmd}} - \mathcal{F}_{\mathrm{ext}}
\end{equation}

\textbf{Step 4:} Solve for the control wrench:
\begin{equation}
\boxed{\mathcal{F}_{\mathrm{cmd}} = \Lambda_b M^{-1}\left(D \xi + J_{\mathcal{E}}^\top K \mathcal{E}\right) + \Lambda_b {}^b\dot{\mathcal{V}}_d + \mu_b + \gamma_b + \left(I - \Lambda_b M^{-1}\right)\mathcal{F}_{\mathrm{ext}}}
\end{equation}

This is the general impedance controller with components:
\begin{itemize}
\item $\Lambda_b M^{-1}(D \xi + J_{\mathcal{E}}^\top K \mathcal{E})$: Impedance feedback with weighted error Jacobian
\item $\Lambda_b {}^b\dot{\mathcal{V}}_d$: Feedforward acceleration term
\item $\mu_b + \gamma_b$: Compensation for Coriolis and gravity
\item $(I - \Lambda_b M^{-1})\mathcal{F}_{\mathrm{ext}}$: External force compensation (inertia-dependent)
\end{itemize}

\textbf{Simplified Cases:}
\begin{itemize}
\item \textbf{When $M = \Lambda_b$} (matching virtual and robot inertia):
\begin{equation}
\mathcal{F}_{\mathrm{cmd}} = D \xi + J_{\mathcal{E}}^\top K \mathcal{E} + \Lambda_b {}^b\dot{\mathcal{V}}_d + \mu_b + \gamma_b
\end{equation}

\item \textbf{With $M = \Lambda_b$ and ${}^b\dot{\mathcal{V}}_d = 0$} (regulation):
\begin{equation}
\mathcal{F}_{\mathrm{cmd}} = D \xi + J_{\mathcal{E}}^\top K \mathcal{E} + \mu_b + \gamma_b
\end{equation}

\item \textbf{Small rotation errors} ($\|e_R\| \ll 1$ so $J_l^{-1} \approx I$) with isotropic stiffness ($K=kI_6$):
\begin{equation}
J_{\mathcal{E}} \approx \begin{pmatrix}  \alpha I_3 & 0 \\ -[e_p] & I_3 \end{pmatrix}, \quad \mathcal{F}_{\mathrm{cmd}} = D \xi + K_{\alpha} \, \mathcal{E} + \mu_b + \gamma_b
\end{equation}
where $K_{\alpha}=\mathrm{diag}(\alpha k I_3,k I_3)$ recovers the familiar linear impedance form.
\end{itemize}

This completes the geometrically consistent impedance controller derivation for SE(3) manipulation tasks.
% Appendix: SWIVL Instantiation in SE(2)
\section{SWIVL Instantiation in SE(2)}
\label{app:se2}

While the SWIVL framework presented in Section~\ref{sec:method} is formulated for general bimanual manipulation of $k$-DoF articulated objects in SE(3), our experimental evaluation in Section~\ref{sec:experiments} focuses on SE(2) planar tasks with a \\textbf{single internal joint} ($k = 1$). This design choice allows systematic study of force coupling and constraint satisfaction while controlling for the additional complexity of full 3D manipulation. Here we detail how the SE(3) formulation naturally reduces to SE(2) in this 1-DoF setting and how each component of SWIVL is instantiated.

\subsection{SE(2) Geometric Formulation (1-DoF Object)}

\subsubsection{Configuration Space}

In SE(2), poses are represented as $(x, y, \theta) \in \mathbb{R}^2 \times SO(2)$, where $(x, y)$ is planar position and $\theta$ is orientation around the vertical z-axis. The homogeneous transformation matrix:

\[
T \in SE(2): \quad T = \begin{bmatrix} \cos\theta & -\sin\theta & x \\ \sin\theta & \cos\theta & y \\ 0 & 0 & 1 \end{bmatrix}
\]

\subsubsection{Twist Space}

The Lie algebra $\mathfrak{se}(2)$ consists of planar twists:

\[
\mathcal{V} = \begin{bmatrix} \omega_z \\ v_x \\ v_y \end{bmatrix} \in \mathbb{R}^3
\]

where $\omega_z \in \mathbb{R}$ is angular velocity around z-axis and $(v_x, v_y) \in \mathbb{R}^2$ is linear velocity in the plane.

\subsubsection{Screw Axis in SE(2)}

For planar articulated objects with a \\textbf{single kinematic joint}, the screw axis $\mathcal{S} = \begin{bmatrix} s_\omega \\ s_v \end{bmatrix}$ reduces to:

\[
\mathcal{S} = \begin{bmatrix} s_\omega \\ s_{v,x} \\ s_{v,y} \end{bmatrix} \in \mathbb{R}^3
\]

\textbf{Joint Type Examples:}
\begin{itemize}
\item \textbf{Revolute joint} (rotation around z-axis): $\mathcal{S} = \begin{bmatrix} 1 \\ 0 \\ 0 \end{bmatrix}$ (pure rotation)
\item \textbf{Prismatic joint} (translation along direction $\hat{d}$): $\mathcal{S} = \begin{bmatrix} 0 \\ d_x \\ d_y \end{bmatrix}$ where $(d_x, d_y)$ defines sliding direction
\end{itemize}

In the general SE(3) formulation (Section~\ref{sec:problem_formulation}), the object Jacobian $J_s(\mathbf{q}_{obj}) \in \mathbb{R}^{6 \times k}$ relates internal joint velocities to relative end-effector motion. In our SE(2), 1-DoF setting, this reduces to a single spatial screw axis $\mathcal{S} \in \mathbb{R}^3$ and the kinematic constraint becomes:

$$
{}^s\mathcal{V}_l - {}^s\mathcal{V}_r = \mathcal{S} \, \dot{q}_{obj}, \quad {}^s\mathcal{V}_i \in \mathbb{R}^3, \quad \mathcal{S} \in \mathbb{R}^3, \quad \dot{q}_{obj} \in \mathbb{R}.
$$

For each grasp, the corresponding \textbf{body-frame joint screw axes} $\mathcal{B}_l, \mathcal{B}_r \in \mathbb{R}^3$ are obtained by transforming $\mathcal{S}$ into the left and right end-effector frames via the SE(2) adjoint (Appendix~\ref{app:notation}). Because the object has a single joint and grasps remain fixed, these body-frame screw axes are \textbf{constant in time and independent of the joint configuration} $q_{obj}$:
$$
\mathcal{B}_i = [Ad_{T_{ib}}] \, {}^b\mathcal{S}, \quad i \in \{l,r\}, \quad \mathcal{B}_i \text{ fixed for a given object}.
$$
Thus, the object Jacobians in each body frame collapse to
$$
J_l(q_{obj}) = \mathcal{B}_l \in \mathbb{R}^{3 \times 1}, \qquad J_r(q_{obj}) = \mathcal{B}_r \in \mathbb{R}^{3 \times 1},
$$
which no longer depend on $q_{obj}$.

\subsubsection{Wrench Space}

Forces and moments in SE(2) are dual to twists:

$$
\mathcal{F} = \begin{bmatrix} m_z \\ f_x \\ f_y \end{bmatrix} \in \mathbb{R}^3
$$

where $m_z$ is moment around z-axis and $(f_x, f_y)$ are planar forces.

\subsubsection{Adjoint Representation}

The adjoint transformation for frame changes in SE(2):

$$
[Ad_T] = \begin{bmatrix}
1 & 0 & 0 \\
y & \cos\theta & -\sin\theta \\
-x & \sin\theta & \cos\theta
\end{bmatrix} \in \mathbb{R}^{3 \times 3}
$$

Twist transformation between frames:

$$
{}^s\mathcal{V} = [Ad_{T_{si}}] \mathcal{V}_i, \quad \mathcal{V}_i = [Ad_{T_{si}^{-1}}] {}^s\mathcal{V}
$$

\subsection{Reference Twist Field Generator in SE(2)}

\subsubsection{SE(2) Trajectory Smoothing}

Given discrete waypoints $\{T_{si}^{des}[\tau]\}_{\tau=0}^H = \{(x[\tau], y[\tau], \theta[\tau])\}_{\tau=0}^H$ from the high-level planner at 10 Hz, we generate dense trajectories at 100 Hz (Low-Level Policy frequency).

\textbf{Position Interpolation:}
Cubic spline interpolation through position waypoints $(x[\tau], y[\tau])$:

$$
\begin{bmatrix} x(t) \\ y(t) \end{bmatrix} = \sum_{j=0}^{3} a_j s^j, \quad s = \frac{t - t_k}{t_{k+1} - t_k}
$$

where coefficients $\{a_j\}$ satisfy boundary conditions (positions and velocities at waypoints).

\textbf{Orientation Interpolation:}
Circular interpolation on SO(2) ensuring shortest path:

$$
\theta(t) = \theta_k + \text{wrap}(\theta_{k+1} - \theta_k) \cdot \phi(s)
$$

where $\phi(s) = 3s^2 - 2s^3$ (cubic smoothing), and wrap ensures $|\theta_{k+1} - \theta_k| \leq \pi$.

\subsubsection{Body Twist Computation}

From smooth trajectory $T_{si}^{des}(t)$, compute desired body twist via time differentiation:

$$
\mathcal{V}_i^{des}(t) = \begin{bmatrix} \dot{\theta}(t) \\ \dot{x}(t)\cos\theta(t) + \dot{y}(t)\sin\theta(t) \\ -\dot{x}(t)\sin\theta(t) + \dot{y}(t)\cos\theta(t) \end{bmatrix} \in \mathbb{R}^3
$$

\subsubsection{Stable Imitation Vector Field}

Following Method Eq.~\eqref{eq:vector_field}, the reference twist combines imitation and stability components:

$$
\mathcal{V}_i^{\text{ref}}(t, T_{sb_i}) = \mathrm{Ad}_{T_{b_id_i}} \mathcal{V}_i^{\text{des}}(t) + k_{p_i} \mathcal{E}_i
$$

where $\mathrm{Ad}_T$ denotes the SE(2) adjoint transformation that maps twists between frames. Since the desired twist $\mathcal{V}_i^{\text{des}}(t)$ is computed in the desired frame $\{d_i\}$, we must transform it to the current body frame $\{b_i\}$ where the controller operates. The transformation $T_{b_id_i} = T_{b_is} T_{sd_i} = (T_{sb_i})^{-1} T_{sd_i}$ represents the relative transformation from the desired frame to the current body frame.

For SE(2), the adjoint transformation is:

$$
\mathrm{Ad}_{T_{b_id_i}} = \begin{bmatrix}
1 & 0 & 0 \\
\Delta y & \cos\Delta\theta & -\sin\Delta\theta \\
-\Delta x & \sin\Delta\theta & \cos\Delta\theta
\end{bmatrix} \in \mathbb{R}^{3 \times 3}
$$

where $(\Delta x, \Delta y, \Delta\theta)$ are the components of $T_{b_id_i}$.

The pose error term $\mathcal{E}_i \in \mathbb{R}^3$ is given by:

\textbf{SE(2) Logarithm Map:}
For pose error $\Delta T = T_{si}^{des}(t^*)^{-1} T_{si}$:

$$
\left[\log(\Delta T)\right]^\vee = \begin{bmatrix}
\Delta\theta \\
\Delta x \cos\theta_{des} + \Delta y \sin\theta_{des} \\
-\Delta x \sin\theta_{des} + \Delta y \cos\theta_{des}
\end{bmatrix}
$$

where $\Delta x = x - x_{des}$, $\Delta y = y - y_{des}$, $\Delta\theta = \text{wrap}(\theta - \theta_{des})$.

\subsection{Bulk-Internal Decomposition via Projection Operators in SE(2)}

Following Method Section 3.2.4, SWIVL uses projection operators based on the learned metric tensor $G = \mathrm{diag}(\alpha^2, 1, 1)$ to decompose twists into bulk and internal motion components. This approach enables independent impedance modulation for each component.

\subsubsection{Metric Tensor and Inner Product}

The SE(2) inner product on $\mathfrak{se}(2)$ is defined using the metric tensor $G \in \mathbb{R}^{3 \times 3}$:

$$
\langle \mathcal{V}_1, \mathcal{V}_2 \rangle_G = \mathcal{V}_1^\top G \mathcal{V}_2 = \alpha^2 \omega_{1,z} \omega_{2,z} + v_{1,x} v_{2,x} + v_{1,y} v_{2,y}
$$

where $\alpha \in \mathbb{R}^+$ is the **learnable characteristic length scale** (part of the RL action space) that weights rotational versus translational components. By learning $\alpha$, the policy discovers task-appropriate notions of orthogonality for separating bulk versus internal motions.

\subsubsection{Projection Operators}

For each end-effector $i \in \{l, r\}$ with constant body-frame screw axis $\mathcal{B}_i \in \mathbb{R}^{3 \times 1}$ (1-DoF object), we construct orthogonal projection operators:

$$
\begin{aligned}
P_{i,\parallel} &= \mathcal{B}_i (\mathcal{B}_i^\top G \mathcal{B}_i)^{-1} \mathcal{B}_i^\top G \in \mathbb{R}^{3 \times 3} \quad \text{(project onto internal motion)}, \\
P_{i,\perp} &= I_3 - P_{i,\parallel} \in \mathbb{R}^{3 \times 3} \quad \text{(project onto bulk motion)}.
\end{aligned}
$$

These operators satisfy:
\begin{itemize}
\item $P_{i,\parallel}^\top G = G P_{i,\parallel}$ (G-self-adjoint for internal projection)
\item $P_{i,\perp}^\top G = G P_{i,\perp}$ (G-self-adjoint for bulk projection)
\item $P_{i,\parallel} + P_{i,\perp} = I_3$ (partition of identity)
\item $P_{i,\parallel} P_{i,\perp} = 0$ (orthogonal subspaces under G-metric)
\end{itemize}

\subsubsection{Twist Decomposition}

Given reference body twist $\mathcal{V}_i^{\text{ref}} \in \mathbb{R}^3$, decompose into bulk and internal components:

$$
\begin{aligned}
\mathcal{V}_{i,\parallel}^{\text{ref}} &= P_{i,\parallel} \mathcal{V}_i^{\text{ref}} \in \mathbb{R}^3 \quad \text{(internal motion: range of } \mathcal{B}_i \text{)}, \\
\mathcal{V}_{i,\perp}^{\text{ref}} &= P_{i,\perp} \mathcal{V}_i^{\text{ref}} \in \mathbb{R}^3 \quad \text{(bulk motion: orthogonal complement)}.
\end{aligned}
$$

This decomposition satisfies G-orthogonality: $\langle \mathcal{V}_{i,\parallel}^{\text{ref}}, \mathcal{V}_{i,\perp}^{\text{ref}} \rangle_G = 0$.

\textbf{Physical Interpretation:}
\begin{itemize}
\item $\mathcal{V}_{i,\parallel}^{\text{ref}}$: Motion component aligned with the object's kinematic constraint (drives joint articulation)
\item $\mathcal{V}_{i,\perp}^{\text{ref}}$: Motion component orthogonal to constraint (drives overall object transport/reorientation)
\end{itemize}

The Low-Level Policy receives both the full reference twists $\{\mathcal{V}_l^{\text{ref}}, \mathcal{V}_r^{\text{ref}}\}$ and their decomposed components, enabling it to learn task semantics from trajectory structure.

\subsubsection{Wrench Decomposition}

By duality, wrenches decompose using transposed projection operators. For measured wrench $\mathcal{F}_i \in \mathbb{R}^3$:

$$
\begin{aligned}
\mathcal{F}_{i,\parallel} &= P_{i,\parallel}^\top \mathcal{F}_i \in \mathbb{R}^3 \quad \text{(productive wrench)}, \\
\mathcal{F}_{i,\perp} &= P_{i,\perp}^\top \mathcal{F}_i \in \mathbb{R}^3 \quad \text{(internal wrench)}.
\end{aligned}
$$

Internal wrench $\mathcal{F}_{i,\perp}$ represents non-productive contact forces that stress the grasp without contributing to joint motion. The RL reward explicitly penalizes $\|\mathcal{F}_{i,\perp}\|_2^2$ to minimize harmful internal forces (Section 3.3.2).

\textbf{Output Summary:}
The Reference Twist Field Generator produces at each timestep (100 Hz):

$$
\{\mathcal{V}_l^{\text{ref}}, \mathcal{V}_r^{\text{ref}}, \mathcal{B}_l, \mathcal{B}_r\}
$$

where individual reference twists $\mathcal{V}_i^{\text{ref}} = \mathrm{Ad}_{T_{b_id_i}} \mathcal{V}_i^{\text{des}} + k_{p_i} \mathcal{E}_i$ are computed from the stable imitation vector field (Eq.~\eqref{eq:vector_field} in Method Section 3.2.2). The Low-Level Policy applies projection operators $P_{i,\parallel}$ and $P_{i,\perp}$ (parameterized by learned $\alpha$) to decompose these into bulk and internal components. Constant screw axes $\mathcal{B}_l, \mathcal{B}_r$ encode the 1-DoF constraint structure.

\subsection{Low-Level Policy Architecture for SE(2)}

\subsubsection{Observation Space}

Following Method Section 3.2.3, the SE(2) policy observes:

\textbf{1. Reference Twists} ($\mathbb{R}^{6}$):
\begin{itemize}
\item $\mathcal{V}_l^{\text{ref}}, \mathcal{V}_r^{\text{ref}} \in \mathbb{R}^3$: Reference motions computed by the Reference Twist Field Generator (Layer 2) at the current time $t$ and current end-effector poses $T_{sb_l}, T_{sb_r}$ (6-dim)
\end{itemize}

\textbf{2. Object Constraints} ($\mathbb{R}^{6}$):
\begin{itemize}
\item $\mathcal{B}_l, \mathcal{B}_r \in \mathbb{R}^3$: Body-frame screw axes defining the object's allowable internal motion directions at each end-effector (6-dim)
\end{itemize}

\textbf{3. Wrench Feedback} ($\mathbb{R}^{6}$):
\begin{itemize}
\item $\mathcal{F}_l, \mathcal{F}_r \in \mathbb{R}^3$: Body wrenches measured at the end-effectors (6-dim)
\end{itemize}

\textbf{4. Proprioception} ($\mathbb{R}^{12}$):
\begin{itemize}
\item End-effector poses: $T_{sb_l}, T_{sb_r} \in \mathrm{SE}(2)$, represented as $(x_i, y_i, \theta_i) \in \mathbb{R}^3$ × 2 (6-dim)
\item End-effector body twists: $\mathcal{V}_l, \mathcal{V}_r \in \mathbb{R}^3$, represented as $(\omega_{z,i}, v_{x,i}, v_{y,i}) \in \mathbb{R}^3$ × 2 (6-dim)
\end{itemize}

\textbf{Total: $o_t \in \mathbb{R}^{30}$ (6+6+6+12=30-dim)}

\textbf{Note:} The policy receives the core physical observations that directly parameterize the impedance controller. The policy network internally computes bulk-internal decomposition of reference twists and measured wrenches using projection operators $P_{i,\parallel}$ and $P_{i,\perp}$ parameterized by the learned metric tensor $G = \mathrm{diag}(\alpha^2, 1, 1)$.

\subsubsection{Action Space}

Following the SE(3) formulation in Method Section 3.2.3, the SE(2) policy outputs impedance modulation variables adapted for the planar, 1-DoF setting:

\textbf{Action Space:}

$$
a_t = (d_{l,\parallel}, d_{r,\parallel}, d_{l,\perp}, d_{r,\perp}, k_{p_l}, k_{p_r}, \alpha) \in \mathbb{R}^7
$$

where:
\begin{itemize}
\item $d_{l,\parallel}, d_{r,\parallel} \in \mathbb{R}^+$: Per-arm damping coefficients for internal motion (parallel to screw axis)
\item $d_{l,\perp}, d_{r,\perp} \in \mathbb{R}^+$: Per-arm damping coefficients for bulk motion (orthogonal to screw axis)
\item $k_{p_l}, k_{p_r} \in \mathbb{R}^+$: Per-arm stiffness gains for the stability term $k_{p_i} \mathcal{E}_i$ in the reference vector field (Eq.~\eqref{eq:vector_field})
\item $\alpha \in \mathbb{R}^+$: Learnable characteristic length scale that defines the metric tensor $G = \mathrm{diag}(\alpha^2, 1, 1)$ for the SE(2) inner product, enabling task-appropriate orthogonal decomposition of twists and wrenches
\end{itemize}

\textbf{Note:} This maintains the full SE(3) action space structure $(d_{l,\parallel}, d_{r,\parallel}, d_{l,\perp}, d_{r,\perp}, k_{p_l}, k_{p_r}, \alpha) \in \mathbb{R}^7$, preserving the ability to independently modulate compliance for each arm. Gripper commands are omitted as grippers remain closed throughout episodes.

\subsubsection{SE(2) Screw-decomposed Controller}

Following the SE(3) controller formulation in Method Section 3.2.4 (Eq.~\eqref{eq:projection_operators}--\eqref{eq:impedance_control_main}), the impedance variables parameterize an SE(2) twist-driven impedance controller:

\textbf{Orthogonal Projection Operators:}

Using the metric tensor $G = \mathrm{diag}(\alpha^2, 1, 1) \in \mathbb{R}^{3 \times 3}$ and body-frame screw axes $\mathcal{B}_i \in \mathbb{R}^{3 \times 1}$:

$$
\begin{aligned}
P_{i,\parallel} &= \mathcal{B}_i (\mathcal{B}_i^\top G \mathcal{B}_i)^{-1} \mathcal{B}_i^\top G \in \mathbb{R}^{3 \times 3}, \\
P_{i,\perp} &= I_3 - P_{i,\parallel} \in \mathbb{R}^{3 \times 3}
\end{aligned}
$$

where $P_{i,\parallel}$ projects onto internal motion (range of $\mathcal{B}_i$) and $P_{i,\perp}$ projects onto bulk motion (orthogonal complement).

\textbf{Damping Matrix Construction:}

$$
K_{d_i} = G (P_{i,\parallel} d_{i,\parallel} + P_{i,\perp} d_{i,\perp}) \in \mathbb{R}^{3 \times 3}
$$

where $d_{i,\parallel}$ and $d_{i,\perp}$ denote the per-arm damping coefficients ($d_{l,\parallel}, d_{l,\perp}$ for left arm, $d_{r,\parallel}, d_{r,\perp}$ for right arm). This allows independent damping modulation for each arm: $d_{i,\parallel}$ controls compliance along the object's kinematic constraint (internal motion), while $d_{i,\perp}$ controls compliance orthogonal to it (bulk motion).

\textbf{Commanded Wrench:}

$$
\mathcal{F}_{\mathrm{cmd}, i} = K_{d_i} (\mathcal{V}_i^{\text{ref}} - \mathcal{V}_i) + \mu_{b,i} \in \mathbb{R}^3
$$

where $\mathcal{V}_i^{\text{ref}} = \mathrm{Ad}_{T_{b_id_i}} \mathcal{V}_i^{\text{des}} + k_{p_i} \mathcal{E}_i$ is the reference twist from Layer 2 (Eq.~\eqref{eq:vector_field}), $\mu_{b,i} = C_{b,i}(q_i, \dot{q}_i) \dot{q}_i$ accounts for Coriolis/centrifugal terms, and gravity $\gamma_{b,i} = 0$ in planar settings.

\textbf{Execution:}

In our SE(2) simulation environment with direct body wrench control, the commanded wrenches $\mathcal{F}_{\mathrm{cmd}, i} \in \mathbb{R}^3$ are directly applied as control inputs to the end-effectors. The simulation environment integrates these wrench commands to update end-effector poses, consistent with the impedance-based control framework.

\textbf{Kinematic Constraint Satisfaction:}

By construction, the projection-based structure ensures the holonomic constraint is satisfied. The reference twists $\mathcal{V}_i^{\text{ref}}$ already respect the constraint through the Reference Twist Field Generator, and the damping matrix $K_{d_i}$ preserves the constraint subspace through its construction from $P_{i,\parallel}$ and $P_{i,\perp}$.

\subsection{Controller Implementation}

\subsubsection{Direct End-Effector Control}

In the SE(2) simulation environment, we use direct end-effector wrench control without intermediate joint-space representations. The commanded body wrenches $\mathcal{F}_{\mathrm{cmd}, i} \in \mathbb{R}^3$ (computed from the impedance controller above) are directly applied as control inputs to the end-effectors. The simulation environment integrates these wrench commands through forward dynamics to update end-effector poses at each control step. This wrench-based control scheme is appropriate for the planar manipulation tasks and allows us to focus on the core challenges of force coupling and constraint satisfaction while maintaining full consistency with the impedance control framework in Method Section 3.2.4.

\textbf{Control Frequency:} 100 Hz (policy and controller run at the same frequency).

\subsection{Reward Function in SE(2)}

The reward function for SE(2) specializes the general formulation from Method Section 3.3.2, adapted for planar manipulation with the learned metric tensor $G = \mathrm{diag}(\alpha^2, 1, 1)$:

$$
r_t = r_{\text{track}} + r_{\text{safety}} + r_{\text{reg}}
$$

\subsubsection{Motion Tracking Reward}

Following Method Eq.~\eqref{eq:reward_tracking}, the tracking reward uses the G-metric to measure velocity error:

$$
r_{\text{track}} = -w_{\text{track}} \sum_{i \in \{l,r\}} \|\mathcal{V}_i - \mathcal{V}_i^{\text{ref}}\|_G^2 = -w_{\text{track}} \sum_{i \in \{l,r\}} (\mathcal{V}_i - \mathcal{V}_i^{\text{ref}})^T G (\mathcal{V}_i - \mathcal{V}_i^{\text{ref}})
$$

Expanding with $G = \mathrm{diag}(\alpha^2, 1, 1)$ and $\mathcal{V}_i = [\omega_{z,i}, v_{x,i}, v_{y,i}]^T \in \mathbb{R}^3$:

$$
\|\mathcal{V}_i - \mathcal{V}_i^{\text{ref}}\|_G^2 = \alpha^2 (\omega_{z,i} - \omega_{z,i}^{\text{ref}})^2 + (v_{x,i} - v_{x,i}^{\text{ref}})^2 + (v_{y,i} - v_{y,i}^{\text{ref}})^2
$$

This ensures tracking error is measured consistently with the impedance control framework, with adaptive weighting between rotational and translational components via the learned parameter $\alpha$.

\subsubsection{Safety Reward}

Following Method Eq.~\eqref{eq:reward_safety}, the safety reward minimizes internal wrenches---wrench components orthogonal to the object's allowable motion direction:

$$
r_{\text{safety}} = -w_{\text{int}} \sum_{i \in \{l,r\}} \|\mathcal{F}_{i,\perp}\|_2^2
$$

\textbf{Wrench Decomposition via Projection Operators.} Consistent with Method Section 3.3.2 and the twist decomposition in Layer 4, wrenches decompose using the transpose of twist projection operators. For measured wrench $\mathcal{F}_i = [m_{z,i}, f_{x,i}, f_{y,i}]^T \in \mathbb{R}^3$:

$$
\mathcal{F}_{i,\parallel} = P_{i,\parallel}^T \mathcal{F}_i, \quad \mathcal{F}_{i,\perp} = P_{i,\perp}^T \mathcal{F}_i = (I_3 - P_{i,\parallel})^T \mathcal{F}_i
$$

where $P_{i,\parallel} = \mathcal{B}_i (\mathcal{B}_i^T G \mathcal{B}_i)^{-1} \mathcal{B}_i^T G$ and $P_{i,\perp} = I_3 - P_{i,\parallel}$ are the SE(2) projection operators defined in Section~\ref{app:se2}.3.2.

This decomposition exploits the duality between twist and wrench spaces under the reciprocal product (virtual power). For any $\mathcal{V} \in \text{range}(P_{i,\perp})$, we have $\mathcal{V} = P_{i,\perp} \mathcal{V}'$, and:

$$
\mathcal{F}_{i,\parallel}^T \mathcal{V} = (P_{i,\parallel}^T \mathcal{F}_i)^T (P_{i,\perp} \mathcal{V}') = \mathcal{F}_i^T P_{i,\parallel} P_{i,\perp} \mathcal{V}' = 0
$$

where the last equality follows from $P_{i,\parallel} P_{i,\perp} = 0$ (orthogonal projections). Similarly, $\mathcal{F}_{i,\perp}^T \mathcal{V} = 0$ for all $\mathcal{V} \in \text{range}(P_{i,\parallel})$.

\textbf{Physical Interpretation:}
\begin{itemize}
\item $\mathcal{F}_{i,\parallel}$: Productive wrench that performs work along the object's internal degree of freedom (joint articulation)
\item $\mathcal{F}_{i,\perp}$: Internal wrench orthogonal to the kinematic constraint that:
\begin{itemize}
    \item Does not contribute to desired object motion (zero virtual power along $\text{range}(P_{i,\parallel})$)
    \item Arises from coordination errors between the two arms
    \item Represents constraint forces (bearing loads, friction, contact stresses) unrelated to joint actuation
    \item Increases unnecessary contact stress and grasp instability
    \item Wastes energy and risks hardware damage
\end{itemize}
\end{itemize}

By penalizing $\|\mathcal{F}_{i,\perp}\|_2^2$, the policy learns to minimize non-productive forces while maintaining necessary productive forces for manipulation.

\subsubsection{Regularization Reward}

Following Method Eq.~\eqref{eq:reward_regularization}, the regularization reward encourages smooth motion:

$$
r_{\text{reg}} = -w_{\text{reg}} \sum_{i \in \{l,r\}} \|\dot{\mathcal{V}}_i\|_2^2
$$

where $\dot{\mathcal{V}}_i = [\ddot{\theta}_i, \dot{v}_{x,i}, \dot{v}_{y,i}]^T \in \mathbb{R}^3$ is the SE(2) twist acceleration. This reduces energy consumption, joint jerkiness, and Cartesian jerkiness, promoting natural and efficient movements.

\subsubsection{Termination Conditions}

Following Method Section 3.3.2, grasp stability is enforced through early termination rather than reward penalties. Episodes terminate immediately (task failure) when grasp drift exceeds safety thresholds:

$$
\text{Terminate if: } \exists i \in \{l, r\} \text{ such that } \left\|\left[\log\left((T_{\text{grip},i}^{\text{init}})^{-1} T_{\text{grip},i}\right)\right]^\vee\right\|_2 > d_{\max}
$$

where $T_{\text{grip},i}^{\text{init}}$ is the initial grasp pose, $T_{\text{grip},i}$ is the current grasp pose, and $d_{\max}$ is the maximum allowable drift threshold. For SE(2), the logarithm map computes planar geodesic distance:

$$
\left[\log(\Delta T)\right]^\vee = \begin{bmatrix}
\Delta\theta \\
\Delta x \cos\theta_{init} + \Delta y \sin\theta_{init} \\
-\Delta x \sin\theta_{init} + \Delta y \cos\theta_{init}
\end{bmatrix}
$$

This ensures grasp stability throughout manipulation without explicit reward shaping.

\subsection{SE(2) $\to$ SE(3) Extension Path}

The SE(2) experimental validation serves as a controlled study of SWIVL's core principles. Extension to SE(3) is straightforward:

\textbf{Mathematical Framework:}
\begin{itemize}
\item All SE(3) formulations in Section 3 directly apply
\item Twist space: $\mathfrak{se}(2) \subset \mathfrak{se}(3)$ (3-dim $\to$ 6-dim)
\item Metric tensor: $G = \mathrm{diag}(\alpha^2, 1, 1) \in \mathbb{R}^{3 \times 3} \to G = \mathrm{diag}(\alpha^2 I_3, I_3) \in \mathbb{R}^{6 \times 6}$ (scalar rotation $\to$ 3D rotation weighting)
\item Action space: $(d_{l,\parallel}, d_{r,\parallel}, d_{l,\perp}, d_{r,\perp}, k_{p_l}, k_{p_r}, \alpha) \in \mathbb{R}^7$ (same structure for both SE(2) and SE(3))
\item Object Jacobian: $\mathcal{B}_i \in \mathbb{R}^{3 \times 1} \to J_i \in \mathbb{R}^{6 \times k}$ (single screw axis $\to$ multi-DoF Jacobian)
\item Network architecture scales with input/output dimensions
\end{itemize}

\textbf{Engineering Requirements:}
\begin{itemize}
\item 6-axis F/T sensors (already available on Franka FR3)
\item 7-DoF differential IK controller (standard in Franka SDK) for joint-space control
\item SE(3) trajectory smoothing (geodesic interpolation, Appendix C)
\item Robot proprioception (end-effector poses and twists, object tracking, gripper feedback)
\end{itemize}

\textbf{Validation Strategy:}
\begin{enumerate}
\item SE(2) experiments (current work): Isolate force coupling and constraint satisfaction with impedance-based control
\item SE(3) simulation: Validate 6-DoF extension with gravity, collisions, and per-arm impedance modulation
\item Real-world deployment: Franka FR3 dual-arm setup
\end{enumerate}

The SE(2) results provide strong evidence that SWIVL's principles---learned impedance variables, projection-based motion decomposition, FiLM-based object conditioning, and screw-decomposed control---will transfer to full SE(3) manipulation.

% Appendix: Learning Settings for SE(2) Implementation
\section{Learning Settings for SE(2) Implementation}
\label{app:learning_settings}

This appendix provides comprehensive implementation details for the SWIVL Low-Level Policy in the SE(2) planar manipulation setting, including network architecture, training configuration, and simulation environment specifications. All experiments are conducted in the BiarT (Bimanual Articulated manipulation) environment described in Section~\ref{sec:exp_setup}.

\subsection{Network Architecture}

The Low-Level Policy $\pi_\theta: \mathcal{O} \to \Delta(\mathcal{A})$ is implemented as a neural network with object-conditioned multi-stream architecture, employing Feature-wise Linear Modulation (FiLM) to inject object geometric structure throughout all feature processing stages.

\subsubsection{Input and Output Specifications}

\textbf{Observation Space:} $o_t \in \mathbb{R}^{30}$ (SE(2) planar setting)
\begin{itemize}
\item \textbf{Reference Twists} (6-dim): $\mathcal{V}_l^{\text{ref}}, \mathcal{V}_r^{\text{ref}} \in \mathbb{R}^3$
\item \textbf{Object Constraints} (6-dim): Body-frame screw axes $\mathcal{B}_l, \mathcal{B}_r \in \mathbb{R}^3$
\item \textbf{Wrench Feedback} (6-dim): Body wrenches $\mathcal{F}_l, \mathcal{F}_r \in \mathbb{R}^3$
\item \textbf{Proprioception} (12-dim): End-effector poses $(x_i, y_i, \theta_i) \in \mathbb{R}^3$ × 2 (6-dim), body twists $\mathcal{V}_i = (\omega_{z,i}, v_{x,i}, v_{y,i}) \in \mathbb{R}^3$ × 2 (6-dim)
\end{itemize}

Note: This corresponds to the SE(2) observation space detailed in Method Section 3.2.3 and Appendix~\ref{app:se2}.

\textbf{Input Normalization:} Each modality is normalized before being fed to its respective encoder to ensure balanced gradients and stable learning:

\begin{itemize}
\item \textbf{Reference Twists}: Twist components clipped to $[-v_{\max}, v_{\max}]$ then scaled by $s_{ref}$
\item \textbf{Object Constraints}: Screw axes are already unit-normalized
\item \textbf{Wrench Feedback}: Running normalization with exponential moving average: $\hat{\mathcal{F}} = (\mathcal{F} - \mu_{\mathcal{F}}) / (\sigma_{\mathcal{F}} + \epsilon)$ where $\mu_{\mathcal{F}}, \sigma_{\mathcal{F}}$ are updated online with decay $\alpha_{wrench}$
\item \textbf{Proprioception}: Poses clipped to workspace bounds $[-p_{\max}, p_{\max}] \times [-p_{\max}, p_{\max}] \times [-\pi, \pi]$ then scaled by $s_{pose}$; body twists clipped to $[-\dot{p}_{\max}, \dot{p}_{\max}]$ then scaled by $s_{vel}$
\end{itemize}

\textbf{Action Space:} $a_t \in \mathbb{R}^7$ (SE(2) planar setting)
\begin{itemize}
\item Per-arm damping coefficients for internal motion: $d_{l,\parallel}, d_{r,\parallel} \in \mathbb{R}$
\item Per-arm damping coefficients for bulk motion: $d_{l,\perp}, d_{r,\perp} \in \mathbb{R}$
\item Stiffness gains: $k_{p_l}, k_{p_r} \in \mathbb{R}$
\item Characteristic length scale: $\alpha \in \mathbb{R}$
\end{itemize}

These impedance variables parameterize the SE(2) screw-decomposed controller as detailed in Appendix~\ref{app:se2}.

\subsubsection{Multi-Stream Encoder Architecture}

\textbf{Object Structure Encoder (Conditioning Generator):}

The object encoder processes kinematic constraint information and generates a shared embedding that is then projected to stream-specific FiLM parameters:

$$
\begin{aligned}
h_{obj}^{(1)} &= \text{SiLU}(\text{LayerNorm}(W_{obj}^{(1)} x_{obj} + b_{obj}^{(1)})) \in \mathbb{R}^{64}, \\
e_{obj} &= \text{SiLU}(\text{LayerNorm}(W_{obj}^{(2)} h_{obj}^{(1)} + b_{obj}^{(2)})) \in \mathbb{R}^{128}
\end{aligned}
$$

where $x_{obj} \in \mathbb{R}^{6}$ contains body-frame screw axes $\mathcal{B}_l, \mathcal{B}_r \in \mathbb{R}^3$. The shared object embedding $e_{obj}$ is projected to layer-specific FiLM parameters via lightweight affine transformations:

$$
[\gamma_{s}^{(l)}, \beta_{s}^{(l)}] = W_{FiLM,s}^{(l)} e_{obj} + b_{FiLM,s}^{(l)} \in \mathbb{R}^{d_s} \times \mathbb{R}^{d_s}
$$

where $s \in \{\text{ref}, \text{wrench}, \text{proprio}, \text{fuse}\}$ denotes the stream, $l$ is the layer index, and $d_s$ is the feature dimension of that layer. This ensures dimensional compatibility between FiLM parameters and target features.

\textbf{Reference Motion Encoder:}

Processes reference twists with object-aware feature transformation:

$$
\begin{aligned}
h_{ref}^{(0)} &= \text{SiLU}(\text{LayerNorm}(W_{ref}^{(1)} x_{ref} + b_{ref}^{(1)})) \in \mathbb{R}^{128}, \\
h_{ref} &= \text{FiLM}(\text{LayerNorm}(W_{ref}^{(2)} h_{ref}^{(0)} + b_{ref}^{(2)}); \gamma_{ref}^{(1)}, \beta_{ref}^{(1)}) \in \mathbb{R}^{128}
\end{aligned}
$$

where $x_{ref} \in \mathbb{R}^{6}$ contains reference twists $\mathcal{V}_l^{\text{ref}}, \mathcal{V}_r^{\text{ref}}$. The policy network internally computes bulk-internal decomposition using projection operators.

\textbf{Wrench Encoder:}

Processes force-torque sensor feedback with object-aware feature transformation:

$$
\begin{aligned}
h_{wrench}^{(0)} &= \text{SiLU}(\text{LayerNorm}(W_{wrench}^{(1)} x_{wrench} + b_{wrench}^{(1)})) \in \mathbb{R}^{128}, \\
h_{wrench} &= \text{FiLM}(\text{LayerNorm}(W_{wrench}^{(2)} h_{wrench}^{(0)} + b_{wrench}^{(2)}); \gamma_{wrench}^{(1)}, \beta_{wrench}^{(1)}) \in \mathbb{R}^{128}
\end{aligned}
$$

where $x_{wrench} \in \mathbb{R}^{6}$ contains body wrenches $\mathcal{F}_l, \mathcal{F}_r$. The policy network internally computes productive-internal wrench decomposition using projection operators.

\textbf{Proprioception Encoder:}

Processes robot state information with higher capacity for rich state representation:

$$
\begin{aligned}
h_{proprio}^{(0)} &= \text{SiLU}(\text{LayerNorm}(W_{proprio}^{(1)} x_{proprio} + b_{proprio}^{(1)})) \in \mathbb{R}^{128}, \\
h_{proprio} &= \text{FiLM}(\text{LayerNorm}(W_{proprio}^{(2)} h_{proprio}^{(0)} + b_{proprio}^{(2)}); \gamma_{proprio}^{(1)}, \beta_{proprio}^{(1)}) \in \mathbb{R}^{128}
\end{aligned}
$$

where $x_{proprio} \in \mathbb{R}^{12}$ contains end-effector poses and velocities.

\subsubsection{Multi-Modal Fusion and Policy Head}

\textbf{Feature Fusion:}

Encoded features from all streams are concatenated and fused through object-conditioned layers:

$$
\begin{aligned}
\tilde{h} &= [h_{ref}, h_{wrench}, h_{proprio}] \in \mathbb{R}^{384}, \\
h_{fused}^{(1)} &= \text{SiLU}(\text{FiLM}(W_{fuse}^{(1)} \tilde{h} + b_{fuse}^{(1)}; \gamma_{fuse}^{(1)}, \beta_{fuse}^{(1)})) \in \mathbb{R}^{256}, \\
h_{context} &= \text{FiLM}(W_{fuse}^{(2)} h_{fused}^{(1)} + b_{fuse}^{(2)}; \gamma_{fuse}^{(2)}, \beta_{fuse}^{(2)}) \in \mathbb{R}^{256}
\end{aligned}
$$

\textbf{Action Decoder:}

The fused context is decoded into action distribution parameters:

$$
\begin{aligned}
h_{action} &= \text{SiLU}(W_{action}^{(1)} h_{context} + b_{action}^{(1)}) \in \mathbb{R}^{128}, \\
[\mu, \log\sigma] &= W_{action}^{(2)} h_{action} + b_{action}^{(2)} \in \mathbb{R}^{14}
\end{aligned}
$$

where $\mu \in \mathbb{R}^7$ and $\log\sigma \in \mathbb{R}^7$ parameterize a diagonal Gaussian action distribution $\pi_\theta(a|o) = \mathcal{N}(a; \mu(o), \text{diag}(\exp(\log\sigma(o))))$ for the 7-dimensional SE(2) impedance action space $(d_{l,\parallel}, d_{r,\parallel}, d_{l,\perp}, d_{r,\perp}, k_{p_l}, k_{p_r}, \alpha)$. The log standard deviation is clipped to $[\log(0.01), \log(10)]$ to prevent numerical instability.

\textbf{Positivity Constraint:} Since all impedance parameters must be strictly positive ($d_{i,\parallel}, d_{i,\perp}, k_{p_i}, \alpha \in \mathbb{R}^+$) for physical stability, the sampled actions from the Gaussian distribution are passed through a Softplus activation function:
$$
a_{\text{final}} = \text{Softplus}(a_{\text{sampled}}) = \log(1 + \exp(a_{\text{sampled}}))
$$
This ensures $a_{\text{final}} > 0$ for all components while maintaining differentiability for policy gradient updates. The Softplus function provides smooth gradients near zero, avoiding the non-differentiability issues of ReLU or absolute value, and naturally prevents negative damping or stiffness coefficients that would destabilize the impedance controller.

\subsubsection{Architectural Components}

\textbf{FiLM Layer:} Feature-wise Linear Modulation applies affine transformation based on object conditioning:
$$
\text{FiLM}(h; \gamma^{(obj)}, \beta^{(obj)}) = \gamma^{(obj)} \odot h + \beta^{(obj)}
$$

where $\gamma^{(obj)}, \beta^{(obj)} \in \mathbb{R}^{d}$ are stream- and layer-specific parameters projected from the shared object embedding $e_{obj}$ and modulate features element-wise. This enables object-specific feature transformation throughout the network while maintaining dimensional compatibility.

\textbf{Activation:} SiLU (Swish) for smooth gradients: $\text{SiLU}(x) = x \cdot \sigma(x)$

\textbf{Normalization:} LayerNorm with $\epsilon = 10^{-5}$: $\text{LayerNorm}(x) = \frac{x - \mu}{\sqrt{\sigma^2 + \epsilon}} \odot \gamma_{LN} + \beta_{LN}$

Note: The learnable parameters $\gamma_{LN}$ and $\beta_{LN}$ in LayerNorm are distinct from the FiLM parameters $\gamma^{(obj)}$ and $\beta^{(obj)}$.

\subsection{Training Configuration}

\subsubsection{Reinforcement Learning Algorithm}

We train the Low-Level Policy using Proximal Policy Optimization (PPO) with clipped objective:

$$
L^{CLIP}(\theta) = \mathbb{E}_t \left[\min\left(r_t(\theta) \hat{A}_t, \text{clip}(r_t(\theta), 1-\epsilon, 1+\epsilon) \hat{A}_t\right)\right]
$$

where $r_t(\theta) = \frac{\pi_\theta(a_t|o_t)}{\pi_{\theta_{old}}(a_t|o_t)}$ is the probability ratio and advantages are computed via Generalized Advantage Estimation (GAE).

\subsubsection{Hyperparameters}

\textbf{Optimization:}
\begin{itemize}
\item \textbf{Optimizer:} Adam with $\beta_1 = 0.9$, $\beta_2 = 0.999$
\item \textbf{Learning rate:} $3 \times 10^{-4}$ with linear decay over training
\item \textbf{Gradient clipping:} Maximum norm 0.5
\item \textbf{Weight decay:} $10^{-4}$
\end{itemize}

\textbf{PPO Configuration:}
\begin{itemize}
\item \textbf{Rollout horizon:} 256 steps per worker
\item \textbf{Batch size:} 4096 transitions per iteration
\item \textbf{Mini-batch size:} 256 transitions per update
\item \textbf{Update epochs:} 10 epochs per batch
\item \textbf{Clip range:} $\epsilon = 0.2$
\item \textbf{Value loss coefficient:} 0.5
\item \textbf{Entropy coefficient:} $0.01 \to 0.001$ (linear annealing)
\end{itemize}

\textbf{GAE Configuration:}
\begin{itemize}
\item \textbf{Discount factor:} $\gamma = 0.99$
\item \textbf{GAE lambda:} $\lambda = 0.95$
\end{itemize}

\textbf{Policy Distribution:}
\begin{itemize}
\item \textbf{Type:} Diagonal Gaussian with state-dependent standard deviation
\item \textbf{Initial log std:} $\log\sigma_0 = -0.5$
\item \textbf{Action bounds:} $[-10, 10]$ for raw Gaussian samples before Softplus transformation (ensuring final positive actions in practical range $[\text{Softplus}(-10), \text{Softplus}(10)] \approx [4.5 \times 10^{-5}, 10.00]$)
\end{itemize}

\subsubsection{Initialization Strategy}

\textbf{Linear Layers:}
Xavier initialization with fan-averaging:
$$
W \sim \mathcal{U}\left(-\sqrt{\frac{6}{n_{in} + n_{out}}}, \sqrt{\frac{6}{n_{in} + n_{out}}}\right)
$$

\textbf{FiLM Generators:}
Initialize scale parameters near identity and shift parameters near zero to ensure stable initial conditioning:
$$
W_{\gamma} \sim \mathcal{N}(0, 0.01^2), \quad b_{\gamma} = 1, \quad W_{\beta} \sim \mathcal{N}(0, 0.01^2), \quad b_{\beta} = 0
$$

This ensures that FiLM conditioning initially approximates identity transformation, preventing disruption of gradient flow during early training.

\textbf{Action Head:}
Small-scale initialization to encourage near-zero initial actions:
$$
W_{action}^{(2)} \sim \mathcal{N}(0, 0.01^2), \quad b_{action}^{(2)} = 0
$$

% Appendix: SE(2) Simulation Environment Settings
\section{SE(2) Simulation Environment Settings}
\label{app:environment_settings}

This appendix specifies the simulation environment configuration, task specifications, and evaluation protocol for the SE(2) planar manipulation experiments.

\subsection{Simulation Environment}
\label{app:simulation}

\textbf{Workspace Configuration:}
\begin{itemize}
\item Total space: 512 × 512 pixels
\item Effective space: 501 × 501 pixels
\item Walls: Located at (5, 5) $\sim$ (506, 506) with 2-pixel thickness
\end{itemize}

\textbf{Physics and Control:}
\begin{itemize}
\item Physics timestep: 0.002s (500 Hz simulation)
\item Control frequency: 50 Hz (policy and controller)
\item Robot platform: Dual end-effectors with direct wrench control
\item Force-torque sensing: 3-axis $(m_z, f_x, f_y)$ per end-effector at 50 Hz
\end{itemize}

\textbf{Object Dataset:}
\begin{itemize}
\item Total objects: 9 (3 joint types × 3 variants per type)
\item Joint types: Fixed (rigid transport), Revolute (angular articulation), Prismatic (linear articulation)
\item Material: Rigid body with Coulomb friction ($\mu = 0.6$--0.9)
\item Contact model: Soft contact with 5000 N/m stiffness
\end{itemize}

\subsection{Task Specifications}
\label{app:tasks}

\textbf{Task Objective:}
Manipulate an articulated object from a randomized initial configuration $({}^s T_o^{init}, q_{obj}^{init})$ to a fixed goal configuration $({}^s T_o^{goal}, q_{obj}^{goal})$ through coordinated bimanual control.

\textbf{Object Configuration:}
\begin{itemize}
\item Object frame pose: ${}^s T_o \in SE(2)$ (3 DoF: position $(x_o, y_o)$, orientation $\theta_o$)
\item Joint state: $q_{obj} \in \mathbb{R}$ (1 DoF: angle for revolute, extension for prismatic)
\end{itemize}

\textbf{Task Structure:}
\begin{itemize}
\item Objects: 9 total, each with one fixed goal configuration
\item Episode initialization: Object spawns at random configuration within safe workspace region
\item Episode duration: 8 seconds (400 timesteps at 50 Hz)
\item Visual feedback: Goal configuration marked with distinct pixel color in 512 × 512 workspace image
\end{itemize}

\textbf{Joint Type Categories:}
\begin{enumerate}
\item \textbf{Fixed Joint:} $q_{obj}$ constant; manipulate ${}^s T_o$ to ${}^s T_o^{goal}$
\item \textbf{Revolute Joint:} Manipulate both ${}^s T_o$ and $q_{obj}$ (rotation) to goal
\item \textbf{Prismatic Joint:} Manipulate both ${}^s T_o$ and $q_{obj}$ (extension) to goal
\end{enumerate}

\textbf{Success Criteria (unified):}
\begin{itemize}
\item Object position error: $\|p_o - p_o^{goal}\|_2 < 10$ pixels
\item Object orientation error: $|\theta_o - \theta_o^{goal}| < 5°$
\item Joint error: $|q_{obj} - q_{obj}^{goal}| < 5°$ (revolute) or $< 5$ pixels (prismatic)
\item Grasp stability: Maintained throughout episode (no drift exceeding $d_{\max}$)
\end{itemize}

\textbf{Termination Conditions:}
\begin{itemize}
\item \textbf{Success:} All success criteria satisfied at episode end
\item \textbf{Failure (grasp loss):} Grasp drift exceeds threshold: $\left\|\left[\log\left((T_{\text{grip},i}^{\text{init}})^{-1} T_{\text{grip},i}\right)\right]^\vee\right\|_2 > d_{\max}$ for any $i \in \{l, r\}$
\item \textbf{Timeout:} Maximum episode duration (8 seconds / 400 timesteps) reached without success
\end{itemize}

\subsection{Training Randomization}
\label{app:randomization}

To improve policy robustness, we apply the following randomization during training:
\begin{itemize}
\item Object spawn: Uniform random within safe workspace region (with wall margin)
\item Object orientation: Uniform random $\theta_o \in [0, 2\pi)$
\item Joint configuration: Uniform random within joint limits
\item End-effector positions: Random grasp configurations on object
\end{itemize}

\subsection{Evaluation Protocol}
\label{app:evaluation}

\textbf{Test Configuration:}
\begin{itemize}
\item Objects: 9 (3 per joint type)
\item Goal configurations: 1 fixed goal per object
\item Trials: 100 rollouts per object with randomized initial configurations
\end{itemize}

\textbf{Primary Metrics:}

\begin{enumerate}
\item \textbf{Success Rate (\%):} Binary task completion metric
\begin{equation}
\text{Success Rate} = \frac{1}{N} \sum_{i=1}^{N} \mathbb{1}[\text{task}_i \text{ succeeded}]
\end{equation}
Confidence intervals computed via bootstrap (10,000 resamples).

\item \textbf{Constraint Violation (pixels/s):} Time-averaged kinematic constraint violation
\begin{equation}
\text{CViol} = \frac{1}{T} \sum_{t=1}^{T} \left\| {}^s\mathcal{V}_l(t) - {}^s\mathcal{V}_r(t) - \mathcal{S} \dot{q}_{obj}(t) \right\|_2
\end{equation}
where $\mathcal{S}$ is the spatial screw axis and $T=400$ timesteps.

\item \textbf{Internal Force (N):} Time-averaged non-productive contact forces
\begin{equation}
F_{\text{int}} = \frac{1}{T} \sum_{t=1}^{T} \sum_{i \in \{l,r\}} \left\| \mathcal{F}_{i,\perp}(t) \right\|_2
\end{equation}
where $\mathcal{F}_{i,\perp}$ is the wrench component orthogonal to screw axis.
\end{enumerate}

\textbf{Secondary Metrics:}
\begin{itemize}
\item \textbf{Peak Contact Force (N):} $F_{\text{peak}} = \max_{t,i} \|\mathcal{F}_i(t)\|_2$
\item \textbf{Tracking RMSE (pixels):} End-effector tracking error
\begin{equation}
\text{RMSE} = \sqrt{\frac{1}{2T} \sum_{t=1}^{T} \sum_{i \in \{l,r\}} d_{SE(2)}^2(T_{si}^{actual}(t), T_{si}^{des}(t))}
\end{equation}
\item \textbf{Motion Smoothness (pixels/s³):} Jerk magnitude
\end{itemize}

\textbf{Statistical Testing:}
\begin{itemize}
\item Method: Welch's t-test ($p < 0.05$) with Bonferroni correction for multiple comparisons
\item Effect size (Cohen's d): negligible ($|d| < 0.2$), small ($0.2 \leq |d| < 0.5$), medium ($0.5 \leq |d| < 0.8$), large ($|d| \geq 0.8$)
\end{itemize}

% Appendix: Pure Imitation Learning Baseline (OWIL)
\section{Pure Imitation Learning Baseline (OWIL)}
\label{app:owil}

\subsection{Overview}

   extbf{OWIL (Object-Wrench conditioned Imitation Learning)} is a pure imitation learning baseline designed to evaluate the necessity of SWIVL's explicit kinematic decomposition and constraint-based action space. OWIL learns to satisfy kinematic constraints \textbf{implicitly} through behavior cloning on expert demonstrations, without explicit bulk--internal motion decomposition or kinematic-constrained action parameterization.

   extbf{Purpose:} Answer Q1 by comparing SWIVL's physics-aware approach against state-of-the-art imitation learning that receives the same object and wrench information but relies on implicit constraint learning.

\subsection{Architecture (SE(2) Version)}

\subsubsection{Observation Space}

OWIL receives similar object and wrench information as SWIVL's low-level policy, but \textbf{excludes reference motion inputs}:

\begin{enumerate}
\item \textbf{Object Constraints} (6-dim):
   \begin{itemize}
   \item Body-frame screw axes: $\mathcal{B}_l, \mathcal{B}_r \in \mathbb{R}^3$ (6-dim)
   \end{itemize}

\item \textbf{Wrench Feedback} (6-dim):
   \begin{itemize}
   \item Body wrenches: $\mathcal{F}_l, \mathcal{F}_r \in \mathbb{R}^3$ ($(m_z, f_x, f_y)$ × 2 arms = 6-dim)
   \item Filtered with exponential smoothing ($\alpha$=0.3) to reduce sensor noise
   \end{itemize}

\item \textbf{Proprioception} (12-dim):
   \begin{itemize}
   \item End-effector poses: $(x_i, y_i, \theta_i) \in \mathbb{R}^3$ × 2 arms (6-dim)
   \item End-effector body twists: $\mathcal{V}_i = (\omega_{z,i}, v_{x,i}, v_{y,i}) \in \mathbb{R}^3$ × 2 arms (6-dim)
   \end{itemize}
\end{enumerate}

\textbf{Total observation dimension: $o_t \in \mathbb{R}^{24}$}

\textbf{Key Difference from SWIVL:} No reference twist inputs $\{\mathcal{V}_l^{\text{ref}}, \mathcal{V}_r^{\text{ref}}\}$ (6-dim excluded). OWIL learns manipulation directly from demonstrations without explicit reference motion guidance from the vector field.

\subsubsection{Action Space}

OWIL uses a \textbf{direct arm twist action space} in SE(2) without kinematic constraint parameterization:

$$
a_t = ([\mathcal{V}_l], [\mathcal{V}_r]) \in \mathfrak{se}(2) \times \mathfrak{se}(2)
$$

Policy output: $(\mathcal{V}_l, \mathcal{V}_r) \in \mathbb{R}^3 \times \mathbb{R}^3$ (6-dim)
\begin{itemize}
\item Left arm twist: $(\omega_{z,l}, v_{x,l}, v_{y,l})$ (3-dim)
\item Right arm twist: $(\omega_{z,r}, v_{x,r}, v_{y,r})$ (3-dim)
\end{itemize}

\textbf{Constraint Handling:} The kinematic constraint $\mathcal{V}_l - \mathcal{V}_r = \mathcal{S} \dot{q}_{obj}$ is \textbf{not enforced structurally}. OWIL must learn to satisfy constraints implicitly through demonstrations.

\subsubsection{Network Architecture}

\textbf{Encoders:}
\begin{itemize}
\item \textbf{Object encoder (FiLM conditioning):} [64, 64] $\to$ 128-dim $\to$ generates $(\gamma, \beta) \in \mathbb{R}^{64} \times \mathbb{R}^{64}$
\item \textbf{Wrench encoder:} [64, 64] $\to$ 64-dim
\item \textbf{Proprioception encoder:} [128, 128] $\to$ 128-dim
\end{itemize}

\textbf{Policy Head:}
\begin{itemize}
\item Fusion layer: Concatenate [wrench(64) + proprio(128)] $\to$ 192-dim
\item FiLM modulation: $\text{FiLM}(x) = \gamma \odot x + \beta$
\item Residual MLP: [256, 256] $\to$ 6-dim twist outputs (3-dim per arm)
\end{itemize}

\subsection{Training}

\subsubsection{Data Collection}

\begin{itemize}
\item \textbf{Source:} Human teleoperation demonstrations
\item \textbf{Tasks:} All 9 objects (3 fixed, 3 revolute, 3 prismatic) with diverse initial configurations
\item \textbf{Demonstrations:} 5,000 expert trajectories collected across all objects
\item \textbf{Augmentation:} Random spawn within safe workspace region, uniform orientation $\theta_o \in [0, 2\pi)$, random joint configurations within limits
\end{itemize}

\subsubsection{Training Protocol}

\begin{itemize}
\item \textbf{Algorithm:} Behavior cloning with MSE loss
\item \textbf{Optimizer:} Adam (lr=1e-4, $\beta$=(0.9, 0.999))
\item \textbf{Batch size:} 256 trajectories
\item \textbf{Training steps:} 100k gradient updates
\item \textbf{Loss function:}
  $$
  \mathcal{L}_{\text{BC}} = \mathbb{E}_{(o,a) \sim \mathcal{D}} \left[ \|\pi_{\theta}(o) - a^{\text{expert}}\|_2^2 \right]
  $$
\end{itemize}

\subsection{Key Differences from SWIVL}

\begin{table}[h]
\centering
\begin{tabular}{lll}
\hline
\textbf{Component} & \textbf{OWIL} & \textbf{SWIVL} \\
\hline
\textbf{Learning Paradigm} & Behavior cloning (offline) & Reinforcement learning (online) \\
\textbf{Reference Motion} & None (direct from demonstrations) & Stable imitation vector field \\
\textbf{Action Space} & Direct arm twists $(\mathcal{V}_l, \mathcal{V}_r)$ & Impedance variables $(d_{l,\parallel}, d_{r,\parallel}, d_{l,\perp}, d_{r,\perp}, k_{p_l}, k_{p_r}, \alpha)$ \\
\textbf{Constraint Satisfaction} & Implicit learning from data & Structural guarantee via projection operators \\
\textbf{Observation Dimension} & 24-dim (no reference) & 30-dim (reference included) \\
\textbf{Action Dimension} & 6-dim (unconstrained twists) & 7-dim (impedance modulation) \\
\hline
\end{tabular}
\end{table}

\subsection{Expected Limitations}

\begin{enumerate}
\item \textbf{Constraint Violations:} Without structural constraint enforcement, OWIL may violate kinematic constraints, especially when deviating from training distribution.

\item \textbf{Internal Forces:} Implicit coordination learning may lead to higher internal forces compared to SWIVL's explicit force decomposition and compliance.

\item \textbf{Generalization:} Pure imitation may overfit to demonstration characteristics, limiting adaptation to novel objects or task variations.

\item \textbf{Data Efficiency:} Requires large amounts of expert data to implicitly capture constraint satisfaction patterns.

\item \textbf{Robustness:} No corrective feedback mechanism (no vector field stability) when execution deviates from demonstrated trajectories.
\end{enumerate}

\subsection{Evaluation Protocol}

OWIL is evaluated using the same metrics as SWIVL:

\textbf{Primary Metrics:}
\begin{enumerate}
\item \textbf{Success Rate (\%):} Task completion within error thresholds (position $< 10$ pixels, orientation $< 5°$, joint $< 5°$ or $< 5$ pixels)
\item \textbf{Constraint Violation (pixels/s):} $\text{CViol} = \frac{1}{T} \sum_t \|\mathcal{V}_l - \mathcal{V}_r - \mathcal{S}\dot{q}_{obj}\|_2$
\item \textbf{Internal Force (N):} $F_{int} = \frac{1}{T} \sum_{t,i} \|\mathcal{F}_{i,\perp}\|_2$ where $\mathcal{F}_{i,\perp}$ is the wrench component orthogonal to screw axis
\end{enumerate}

\textbf{Evaluation:} 9 objects × 100 trials per object = 900 rollouts per method.

	extbf{Hypothesis:} OWIL's implicit constraint learning will show higher constraint violations and internal forces compared to SWIVL's explicit physics-aware approach, especially on novel objects and task variations not well-represented in demonstrations.


\end{document}
